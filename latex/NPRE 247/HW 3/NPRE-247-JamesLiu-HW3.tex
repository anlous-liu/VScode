\documentclass{article}
\usepackage{graphicx}
\usepackage{amsmath}
\usepackage{array}
\usepackage[font=small, labelfont={sf,bf}, margin=1cm]{caption}
\usepackage{tabularx}
\usepackage{amssymb}
\usepackage[colorlinks,linkcolor=blue]{hyperref}
\title{\textbf{Homework \#2 }}
\author{James Liu}
\date{\ }

\begin{document}
\maketitle
\section*{Problem 1}
\subsection*{a)}

The resonance escape probability will increase.\\
\[p=\text{exp}\left[-\left(\frac{V_f}{V_m}\right)\frac{N_f}{\xi \Sigma_s}I\right]\]
Changing moderator from heavy water to light water only increasing \(\xi \Sigma_s\) and the exponent 
increases and thus the resonance escape probability increases.
\subsection*{b)}
The thermal utilization decraeses.\\
\[f = \frac{1}{1+\varsigma (V_m/V_f)(N_m/N_f)(\bar{\sigma}_{aT}^m/\bar{\sigma}_{aT}^f)}\]\\
Switching moderator from heavy water to light water increase \(\bar{\sigma}_{aT}^m\) and thus increases the denominator
. Therefore, the thermal utilization facotr decraeses.
\subsection*{c)}
\(k_\infty\) will decrease.\\
\[k_\infty = \frac{\nu \overline{\Sigma}_f}{\overline{\Sigma}_a}\]
While changing moderator will not affact fuel therefore the numerator will remain the same.
However, as the absorption crossection for moderator is increased and thus the absorption crossectionfor all 
component in the system increases, resulting the denominator to increase and thus the fraction decreases as well as 
\(k_\infty\).
\section*{Problem 2}
Here, as we are talking about pressurized water reactor, water is both coolant and moderator.
\subsection*{a)}
The fast fission factor will remain unchanged.\\
\[\varepsilon=1+\frac{1-e}{e}\times \frac{\nu^{fe}\overline{\sigma}_{fF}^{fe}}{\nu^{fi}\overline{\sigma}^{fi}_{fT}}\]
It is clear that fuel-moderator (fuel-coolant) ratio is not in the calculation of fast fission factor which means that it remains unchange.
\subsection*{b)}
The resonance escape probability will increase.\\
\[p=\text{exp}\left[-\left(\frac{V_f}{V_m}\right)\frac{N_f}{\xi \Sigma_s}I\right]\]
Here, the increase in volume ratio of coolant and fuel \(\left(\displaystyle\frac{V_m}{V_f}\right)\) will decrease \(\left(\displaystyle\frac{V_f}{V_m}\right)\) abd thus increase the 
exponent which will finally increase the escape probability.
\subsection*{c)}
The thermal utilization factor decraeses.\\
\[f = \frac{1}{1+\varsigma (V_m/V_f)(N_m/N_f)(\bar{\sigma}_{aT}^m/\bar{\sigma}_{aT}^f)}\]\\
ere, the increase in volume ratio of coolant and fuel \(\left(\displaystyle\frac{V_m}{V_f}\right)\) will cause the denominator to become larger and the fraction to become smaller as well as 
the thermal utilization factor.
\subsection*{d)}
It will remain the same.\\
\[\eta =\frac{\nu\sigma^f_f}{\sigma^f_a}\]
\(\eta \) means how many neutrons are produced if one neutron is absorbed by fuel, while all of the properties are only related to fuel and thus changing fuel-moderator ratio will not change it.
\section*{Problem 3}
\subsection*{a)}

\begin{table*}[h]
    \caption*{Table 4.3 in the text book:}
    \centering
    \begin{tabular}{l l}
        \hline \hline
        \(I=2.95+25.8\cdot \sqrt{4/\rho D}\)& for U metal\\
        \(I=4.45+26.6\cdot \sqrt{4/\rho D}\)& for \(\text{UO}_2\)\\
        \hline \hline        
    \end{tabular}
\end{table*}
Resonance escape probability will be increasing.\\ 
According to table 4.3 in the text book and preduced above, the resonance integral for UO\(_2\) is larger than the integral for U, also, as 
extra oxygen in included in the fuel, the density will also d
ecrease. Thus, the resonance integral will decrease after the modification.
\[p=\text{exp}\left[-\left(\frac{V_f}{V_m}\right)\frac{N_f}{\xi \Sigma_s}I\right]\]
According to the formula above, while all other variables remains the same except the resonance integral. While the resonance integral is increasing for UO\(_2\) the 
resonance escape probability shall also increase.
\subsection*{b)}
Neutron utilization factor will decrease.
\[f = \frac{1}{1+\varsigma (V_m/V_f)(N_m/N_f)(\bar{\sigma}_{aT}^m/\bar{\sigma}_{aT}^f)}\]\\
The change will affect \(N_f\) and \(\overline{\sigma}^f_{aT}\). However, \(\overline{\sigma}^f_{aT}\) according to Table E.3 of the text book, is similar for 
\(U\) and \(UO_2\) of 6.623 \(barns\). According to same table of E.3, the density of \(UO_2\) is 10.0 \(g/cm^3\) while for \(U\) it is \(18.9\ g/cm^3\). Thus, \(N_f\)
would decrease after the change, resulting a bigger denominator and a decreased utilization factor.
\newpage
\subsection*{c)}
\(\eta\) decreases.
\[\eta =\frac{\nu\sigma^f_f}{\sigma^f_a}\]
\begin{table*}[h]
    \caption*{crossection used, from Table 3.2 and Table E.3 in the text book}
    \centering
    \begin{tabular}{c c}
        \hline \hline
        &(barns)\\
        \(\sigma_a^{235}\)& 591\\
        \(\sigma_a^{238}\)& 2.42\\
        \(\sigma_a^O\)& \(2\times 10^{-4}\)\\
        \hline \hline
    \end{tabular}
\end{table*}

For fission crossection (\(\sigma_f^f\)), as the enrichment is unchanged, is unchanged.\\
For absorption crossection:
\begin{align*}
    \sigma_a^f(U)&=e \cdot \sigma_a^{235}+(1-e)\sigma_a^{238}\\
    \sigma_a^f(UO_2)&=e \cdot \sigma_a^{235}+(1-e)\sigma_a^{238}+2\cdot \sigma_a^O
\end{align*}
As shown, the absorption crossection will increase due to presence of oxygen, and thus the fraction decreases together with \(\eta\)
\section*{Problem 4}
\subsection*{a)}
\[P_{abs}^{300K}=\frac{2\Delta E}{(1-\alpha)E}\frac{200}{200+40}=1.667\]
\subsection*{b)}
\[P_{abs}^{600K}=\frac{4\Delta E}{(1-\alpha)E}\frac{100}{100+40}=2.857\]
\subsection*{c)}
Yes, clearly \(2.857>1.667\)  meaning that as temperature increase, neutrons are more likly to be absorbed meaning that the escaped portion is lesser.Thus, increase in temperature will lead to decreased resonance escape probability.
\newpage
\section*{Problem 5}

\[p=\text{exp}\left[-\left(\frac{V_f}{V_m}\right)\frac{N_f}{\xi \Sigma_s}I\right]\]
Here increasing moderator is increasing \(\displaystyle \frac{V_m}{V_f}\), decreasing \(\displaystyle \frac{V_f}{V_m}\), thus, 
the resonance escape probability increases meaning less neutrons captured in ransonance.

\section*{Problem 6}
\begin{figure}[h]
    \includegraphics*[scale = 0.5]{figures/Figure_1.png}
\end{figure}
\newpage
\section*{Problem 7}
Take the density of \(UO_2\) to be 270 \(g/cm^3\)\\
\(\xi_{_{H_2O}}\) to be 0.93\\
\(\sigma_s^{^{H_2O}}\) to be 99.52 barns\\
\(\nu\) to be 2.43\\
\(\displaystyle N^{235}=\frac{3\%\times 11}{270}\times A_L=0.0012A_L (cm^{-3})\)\\
\(\displaystyle N^{238}=\frac{97\%\times 11}{270}\times A_L=0.0395A_L (cm^{-3})\)\\
\(\displaystyle \eta = \frac {\nu N^{235} \sigma^{235}_{f}}{N^{235}\sigma_{a}^{235}+N^{238}\sigma_{a}^{238}}
=\frac{2.43\times 0.0012\times 580}{0.0012\times (580+107)+0.0395\times 2.75}=1.812\)\\

\(I = 4.45+26.6\sqrt{\frac{4}{\rho D}}=4.45+26.6\sqrt{\frac{4}{11\times 1}}=20.49 \ barn \)\\
\(\displaystyle N_M= \frac{1}{18}\times A_L=0.055556\times A_L\), \(\displaystyle N_F = \frac{11}{270}\times A_L=0.0407407\times A_L\)\\
\(\displaystyle \frac {N_M}{N_F}=\frac{0.0555556}{0.04074074}=1.3636\)\\
\(\displaystyle \frac{V_MN_M}{V_FN_F}= \frac{V_M}{V_F}\cdot \frac{N_M}{N_F}=2\times 1.3636=2.7272\)
\\\(\displaystyle p =\frac{(1-e)}{(V_MN_M/V_FN_F)\xi^M\sigma^M_s }\ 
I=\frac{1-0.03}{2.7272\times 0.93\times 99.52}\times 20.49=0.921\)\\
\(N_a^f=\displaystyle \frac{11}{270}\times A_L=0.0407A_L\)\\
\(N_a^m=\displaystyle \frac{1}{18}\times A_L=0.0556 A_L\)\\
\(\sigma_a^f=3\%\times (580+107)+97\%\times 2.75=23.278\)\\
\begin{align*}
f&=\frac{V_f \Sigma_a^f \phi^f}{V_f \Sigma_a^f \phi^f + V_m \Sigma^m_a\phi^m}\\
&=\frac{V_f \Sigma_a^f}{V_f \Sigma_a^f+ V_m \Sigma^m_a\frac{\phi^m}{\phi^f}}\\
&=\frac{1}{1+\frac{V_mN_m}{V_fN_f}\cdot\frac{\sigma_a^m}{\sigma_a^f}\times \varsigma }\\
&=\frac{1}{1+2\cdot \frac{0.0556}{0.0407}\cdot \frac{0.5896}{23.278}\times 1.16}\\
&=0.92569
\end{align*}
Thus, \(k_\infty = \varepsilon \eta p f = 1.24\times 1.812\times 0.921\times 0.92569=1.9156\)\\
\newpage
\section*{Problem 8}
\subsection*{a)}

\(\displaystyle N_M= \frac{1.1}{20}\times A_L=0.055\times A_L\), \(\displaystyle N_F = \frac{11}{270}\times A_L=0.0407407\times A_L\)\\
\(\displaystyle \frac {N_M}{N_F}=\frac{0.055}{0.04074074}=1.35\)\\
\(\displaystyle p =\frac{(1-e)}{(V_MN_M/V_FN_F)\xi^M\sigma^M_s }I=0.921=\frac{0.97}{\frac{V_M}{V_F} \cdot 1.35\cdot 0.51
\cdot14.765}\cdot 20.49\)\\
Therefore: \(\displaystyle \frac{V_M}{V_F}=\frac{0.97}{0.921 \cdot 1.35\cdot 0.51\cdot 14.765}\cdot 20.49=2.12\)\\
\subsection*{b)}

\(N_a^f=\displaystyle \frac{11}{270}\times A_L=0.0407A_L\)\\
\(N_a^m=\displaystyle \frac{1.1}{20}\times A_L=0.055 A_L\)\\
\(\sigma_a^f=3\%\times (580+107)+97\%\times 2.75=23.278\)\\
\begin{align*}
f&=\frac{V_f \Sigma_a^f \phi^f}{V_f \Sigma_a^f \phi^f + V_m \Sigma^m_a\phi^m}=0.92569\\
&=\frac{V_f \Sigma_a^f}{V_f \Sigma_a^f+ V_m \Sigma^m_a\frac{\phi^m}{\phi^f}}\\
&=\frac{1}{1+\frac{V_mN_m}{V_fN_f}\cdot\frac{\sigma_a^m}{\sigma_a^f}\times \varsigma }\\
0.92569&=\frac{1}{1+\frac{V_m}{V_f}\cdot \frac{0.05}{0.0407}\cdot \frac{0.0013}{23.278}\times 1.16}\\
1+\frac{V_m}{V_f}\cdot \frac{0.05}{0.0407}\cdot \frac{0.0013}{23.278}\times 1.16&=\frac{1}{0.92569}\\
\frac{V_m}{V_f} &=\frac{\frac{1}{0.92569}-1}{\frac{0.05}{0.0407}\cdot \frac{0.0013}{23.278}\times 1.16}=1008.67
\end{align*}















\end{document}