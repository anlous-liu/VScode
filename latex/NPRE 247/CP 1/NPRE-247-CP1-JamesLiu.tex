\documentclass{article}
\usepackage{graphicx}
\usepackage{amsmath}
\usepackage{array}
\usepackage[font=small, labelfont={sf,bf}, margin=1cm]{caption}
\usepackage{tabularx}
\usepackage{amssymb}
\usepackage[margin=0.5in]{geometry}
\usepackage[colorlinks,linkcolor=blue]{hyperref}
\title{\textbf{Computer Projcet \#1 }}
\author{James Liu}
\date{\today }
\pagenumbering{gobble}
\begin{document}
% \maketitle
% \newpage
\pagenumbering{arabic}
\section{Theory}
\subsection{Analytical}
From the book, there is the basic formula given
\begin{align}
    \frac{dN}{dt} = -\lambda N(t) \label{base}
\end{align}
Therefore, by substituting parameters, we have the defferential equation for substance A's atom number with respect to time.
\begin{align}
    \frac{dN_A}{dt} &= -\lambda_AN_A(t)\label{NAtd}
\end{align}
\begin{align*}
    (D+\lambda_A)N_A(t)&=0\\
    N_A(t)&=C\cdot e^{-\lambda_A t}
\end{align*}
Introducing initial values \(N_A(t)=N_{0_A}\)
\begin{align*}
    c\cdot 1 = N_{0_A}
\end{align*}
Thus, \(N_A(t)=N_{0_A}\cdot e^{-\lambda_A t}\)\\
Thus, given that \(\lambda_A\) and \(N_A\) are going to be positive,
the number of atom of substance A will be decreasing, and the amount of decreased 
atom of substance a becomes atoms of substance B. While substance B also decays, we have a new differencial euqation.
\begin{align}
    \frac{dN_B}{dt} = \lambda_AN_A(t)-\lambda_BN_B(t)\label{NBtd}
\end{align}
Puluging in \(N_A(t)\), it gives:
\begin{align*}
    (D+\lambda_B)N_B(t)=\lambda_A \cdot N_{0_A}\cdot e^{-\lambda_A t}
\end{align*}
There whould be a general solution for the homogenous equation:
\begin{align*}
    N_{B_c} = C e^{-\lambda _B t}
\end{align*}
Assuming that there are 2 terms in a particular solution, we have:
\begin{align*}
    N_{B_p}=C_1\cdot e^{-\lambda _B t}+C_2\cdot e^{-\lambda _A t}
\end{align*}
Plug it back in equation \ref{NBtd}, and solve for \(C_1\) and \(C_2\) gives \(C_1 = 0\) and \(C_2 = \displaystyle \frac{\lambda_A N_{0_A}}{\lambda_B-\lambda_A}\).
Thus the general solution will be \(N_{B_c}+N_{B_p}\) which is \(N_B(t)=\frac{\lambda_A N_{0_A}}{\lambda_B-\lambda_A}e^{-\lambda_At}+C\cdot e^{-\lambda_Bt}\). 
Appling initial value of \(N_B(0)=N_{0_B}\), we have \(C=N_{0_B}-\frac{\lambda_A N_{0_A}}{\lambda_B-\lambda_A}\). Thus, for \(N_B(t)\) we have solution:
\begin{align}
    N_B(t)=\frac{\lambda_A N_{0_A}}{\lambda_B-\lambda_A}e^{-\lambda_At}+\left(N_{0_B}-\frac{\lambda_A N_{0_A}}{\lambda_B-\lambda_A}\right)e^{-\lambda_Bt} \label{NBt}
\end{align}
While that substance C is not decaying, we have:
\begin{align}
    \frac{dN_C}{dt}=-\frac{dN_B(t)}{dt}=\lambda_BN_B(t) \label{NCtd}
\end{align}
Plugging in equation \ref{NBt}, there is:
\begin{align*}
    \frac{dN_C}{dt} = \lambda_B\left(\frac{\lambda_A N_{0_A}}{\lambda_B-\lambda_A}e^{-\lambda_At}+\left(N_{0_B}-\frac{\lambda_A N_{0_A}}{\lambda_B-\lambda_A}\right)e^{-\lambda_Bt}\right)
\end{align*}
Integrating both side give:
\begin{align*}
    N_C(t)&=\frac{\lambda_B\lambda_A N_{0_A}}{\lambda_B-\lambda_A} \int e^{-\lambda_A t}dt +(\lambda_BN_{0_B}-\frac{\lambda_B\lambda_A N_{0_A}}{\lambda_B-\lambda_A})\int e^{-\lambda_Bt}dt\\
    N_C(t)&=\left[-\frac{\lambda_B}{\lambda_A}\frac{\lambda_A N_{0_A}}{\lambda_B-\lambda_A}e^{-\lambda_At}-\left(N_{0_B}-\frac{\lambda_A N_{0_A}}{\lambda_B-\lambda_A}\right)e^{-\lambda_Bt}\right]+C
\end{align*}
Applying initial condition \(N_C(0)=N_{0_C}\) gives:
\begin{align*}
    \left[\frac{-\lambda_B N_{0_A}}{\lambda_B-\lambda_A}-N_{0_B}+\frac{\lambda_A N_{0_A}}{\lambda_B-\lambda_A}\right]+C&=N_{0_C}\\
    -\frac{(\lambda_B-\lambda_A)\cdot N_{0_A}}{\lambda_B-\lambda_A} +C&= N_{0_C}+N_{0_B}\\
    C&=N_{0_C}+N_{0_B}+N_{0_A}
\end{align*}
Thus, there is:
\begin{align}
    N_C(t)=\left[-\frac{\lambda_B N_{0_A}}{\lambda_B-\lambda_A}e^{-\lambda_At}-\left(N_{0_B}-\frac{\lambda_A N_{0_A}}{\lambda_B-\lambda_A}\right)e^{-\lambda_Bt}\right]+N_{0_A}+N_{0_B}+N_{0_C} \label{NCt}
\end{align}
Substituting equation \ref{NBt} back into equation \ref{NBtd}, the full equation of \(N_B(t)'\). Thus, the maximum of \(N_B(t)\) accours when \(N_B(t)'=0\).
\begin{align}
    N_B(t)'= \lambda_AN_{0_A}e^{-\lambda_At}-\lambda_B\left[\frac{\lambda_A N_{0_A}}{\lambda_B-\lambda_A}e^{-\lambda_At}+\left(N_{0_B}-\frac{\lambda_A N_{0_A}}{\lambda_B-\lambda_A}\right)e^{-\lambda_Bt}\right] \label{Nb'}
\end{align}
\begin{align*}
    0&= \lambda_AN_{0_A}e^{-\lambda_At}-\lambda_B\left[\frac{\lambda_A N_{0_A}}{\lambda_B-\lambda_A}e^{-\lambda_At}+\left(N_{0_B}-\frac{\lambda_A N_{0_A}}{\lambda_B-\lambda_A}\right)e^{-\lambda_Bt}\right]
    \\0&=\lambda_AN_{0_A}e^{-\lambda_At} +\left[-\frac{\lambda_B\lambda_A N_{0_A}}{\lambda_B-\lambda_A}e^{-\lambda_At}-\lambda_BN_{0_B}e^{-\lambda_Bt}+\frac{\lambda_B\lambda_A N_{0_A}}{\lambda_B-\lambda_A}e^{-\lambda_Bt}\right]\\
    0&=\left(\lambda_AN_{0_A}-\frac{\lambda_B\lambda_A N_{0_A}}{\lambda_B-\lambda_A}\right)e^{-\lambda_At}-\left(\lambda_BN_{0_B}-\frac{\lambda_B\lambda_A N_{0_A}}{\lambda_B-\lambda_A}\right)e^{-\lambda_Bt}
    \\ \frac{e^{-\lambda_At}}{e^{-\lambda_Bt}}&=\frac{\lambda_BN_{0_B}-\displaystyle \frac{\lambda_B\lambda_AN_{0_A}}{\lambda_B-\lambda_A}}{\lambda_AN_{0_A}-\displaystyle \frac{\lambda_B\lambda_AN_{0_A}}{\lambda_B-\lambda_A}}\\
    \\ e^{(\lambda_B-\lambda_A)t}&=\frac{\lambda_BN_{0_B}-\displaystyle \frac{\lambda_B\lambda_AN_{0_A}}{\lambda_B-\lambda_A}}{\lambda_AN_{0_A}-\displaystyle \frac{\lambda_B\lambda_AN_{0_A}}{\lambda_B-\lambda_A}}
\end{align*}
Thus, there is:
\begin{align}
    t_{max_{N_B}}=\frac{ln\left(\displaystyle \frac{\lambda_BN_{0_B}-\displaystyle \frac{\lambda_B\lambda_AN_{0_A}}{\lambda_B-\lambda_A}}{\lambda_AN_{0_A}-\displaystyle \frac{\lambda_B\lambda_AN_{0_A}}{\lambda_B-\lambda_A}}\right)}{\lambda_B-\lambda_A}
\end{align}
\subsection{Numerical}
Based on equation \ref{base}, there is:
\begin{align}
    \frac{dN(t_0)}{dt}= \lim_{h\to  0} \frac{N(t_0+h)-N(t_0)}{h} = -\lambda N (t)
\end{align}
Thus, when modifying \(h\) there gives a numerical solution step by step, which gives:
\begin{align}
    N(t_0+\Delta t)=N(t_0)-\lambda N(t_0)\Delta t
\end{align}
\newpage
\section{Results}

\begin{table}[h]
    \centering
    \caption{Parameters Used in This Section} \label{parameters}
    \begin{tabular}{c c}
        \hline
        \(t_{A,1/2}\)&2.53h\\
        \(t_{B,1/2}\)&11.05h\\
        \(t_{C,1/2}\)&stable\\
        \(N_{0_A}\)&100\\
        \(N_{0_B}\)&0\\
        \(N_{0_C}\)&0\\
        \(t_{\text{final}}\)&60h\\
        \hline        
    \end{tabular}
\end{table}
\begin{align}
    \lambda = \frac{\ln (2)}{t_{1/2}}\label{halflife conversion}
\end{align}
Also, by using equation \ref{halflife conversion}, the halflife mentioned here can be converted into decay constants.


\subsection{}


\begin{table*}[h]
    \centering
    \caption*{$\Delta t$ used}
\begin{tabular}{c c}
    \hline 
    Type& \(\Delta t\)(h)\\
    Coarse& 3\\
    Medium& 1.5\\
    Fine& 0.75\\ \hline
\end{tabular}

\end{table*}
\begin{figure}[h]
    \centering
    \includegraphics{figures/result_2.2.1.png}
    \caption{} \label{1}
\end{figure}
From figure \ref{1}, it there is a clear pattern that the smaller \(\Delta t\) used in performing analytical solutions, the better the output curve fits the analytical curve.
\newpage
\subsection{}
\begin{figure}[h]
    \centering
    \includegraphics{figures/result_2.2.2.png}
    \caption{Numerical Solution with \(\Delta t = 0.05 h\)} \label{2}
\end{figure}
From figure \ref{2}, there is a clear pattern that the \(N_A\) dropes expotentially with \(N_B\) first increase  then decrease. All in all, the total number of particles stays unchanged and euqals to the sum of initial
values. Here, as displayed in table \ref{parameters}, the system starts with 100 particles in total giving the total number of particles staying the same 
thoughout the time range displayed to be 100.
\subsection{}
\begin{table}[h]
    \centering
    \caption{\(\Delta t\) used in section 2.2.3} 
    \begin{tabular}{c c c c c c}
        \hline
        \(\Delta t(h)\)&4&1&0.6&0.4&0.1\\
        \hline
    \end{tabular}
\end{table}
\begin{figure}[h]
    \centering
    
    
    \includegraphics{figures/result_2.2.3.png}
    \caption{}
    \label{1/dt}
\end{figure}
From figure \ref{1/dt}, there is a clear pattern that, the lower \(\Delta t\) used in the numerical solution, the closer that results of the time \(N_B\) gets maximized will be compared to the analytical true value.














\end{document}