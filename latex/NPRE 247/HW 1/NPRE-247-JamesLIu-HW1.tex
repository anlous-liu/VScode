\documentclass{article}
\usepackage{graphicx}
\usepackage{amsmath}
\usepackage{array}
\usepackage[font=small, labelfont={sf,bf}, margin=1cm]{caption}
\usepackage{tabularx}
\usepackage{amssymb}
\usepackage[colorlinks,linkcolor=blue]{hyperref}
\title{\textbf{Homework \#1 }}
\author{James Liu}
\date{\ }

\begin{document}
\maketitle

\section*{Problem 1}
    Macroscopic cross-section follows that:\\
    \(\Sigma = N\cdot \sigma= 
    (0.048\times 10^{24}) \times  4.5 \times 10^{-24}
    = 0.216\ \text{cm}^{-1}\).
\section*{Problem 2}
    \(\displaystyle{\sigma_{D_2O}=2\sigma_D+\sigma_O=2\times 2.6+1.6=6.8\ \text{barn}}\)\\
    \(\Sigma = N\cdot \sigma_{D_2O}=0.3323\times 10^{24}\times 6.8\times 10^{-24}=2.259\ \text{cm}^{-1}\)
\section*{Problem 3}
    \begin{table}[h]
        \centering
        \begin{tabular}{c c c c c}
            \multicolumn{5}{c}{Used Microscopic Abosoption Cross Section}\\ \hline \hline
            Element&C&Cr&Ni&Fe\\\hline
            $\sigma(barn)$ &3.530(mb)&15.38&4.505&2.463\\ \hline
        \end{tabular}
    \end{table}

    \begin{align*}
    N_C &=\displaystyle{\frac {\rho \cdot N_A}{M_C}=\frac{0.08\% \cdot 7.86 \cdot 6.023\times 10^{23}}{12.011}=3.153\times 10^{20}\ \frac {\text{atom}}{\text{cm}^{3}}}\\
    N_{Cr}&=\displaystyle{\frac{19\% \cdot 7.86 \cdot 6.023\times 10^{23}}{51.996}=1.89\times 10^{21}}\ \frac {\text{atom}}{\text{cm}^{3}}\\
    N_{Ni}&=\displaystyle{\frac{10\% \cdot 7.86 \cdot 6.023\times 10^{23}}{58.71}=8.823\times 10^{20}}\ \frac {\text{atom}}{\text{cm}^{3}}\\
    N_{Fe}&=\displaystyle{\frac{70.92\% \cdot 7.86 \cdot 6.023\times 10^{23}}{55.847}=6.5\times 10^{21}}\ \frac {\text{atom}}{\text{cm}^{3}}\\
        %
        \Sigma_a    &= \displaystyle {\sum N\cdot \sigma_a}=N_{C}\cdot \sigma_{C}+N_{Cr}\cdot \sigma_{Cr}+N_{Ni}\cdot \sigma_{Ni}+N_{Fe}\cdot \sigma_{Fe}\\
                    &=3.153\cdot 3.530\times 10^{-3} \times 10^{-4}+15.38\cdot 1.89\times 10^{-3}+4.505\cdot 8.823\times 10^{-4}+2.463\cdot 6.5\times 10^{-3}\\
                    &=0.0491 \ \text{cm}^{-1}                    
        %
    \end{align*}
\section*{Problem 4}
    \begin{table}[h]
        \centering
        \begin{tabular}{c c c c}
            \multicolumn{4}{c}{Used Microscopic Abosoption Cross Section}\\ \hline \hline
            Element&$U_{235}$ & $U_{238}$& O               \\\hline
            $\sigma(barn)$ & 683.81&   2.683&189.9($\mu$b)                         \\ \hline
        \end{tabular}
    \end{table}
    \begin{align*}
        \sigma_U    &= e\cdot \sigma_{U_{235}}+(1-e)\cdot \sigma_{U_{238}} \\
                    &=3\% \times 683.81+97\% 2.683\\
                    &=23.1168\  (barn)                                                  \\
        \sigma_{UO_2}&=2\times \sigma_O + \sigma_U\\\
                    &= 2\times 189.9\times 10^{-6}+ 23.1168\\
                    &=23.1172\ \text{barn}\\
        \Sigma_{UO_2}&=N \cdot \sigma = \frac{\rho_{UO_2} \cdot N_A}{M_{UO_2}}\cdot \sigma_{UO_2}\\
                    &=\frac{10.5\times 6.023\times 10^{23}}{2\cdot15.999+(3\%\cdot 235+97\%\cdot 238)}\times 23.1172\times 10^{-24}\\
                    &=0.5417\ \text{cm}^{-1}
    \end{align*}
\section*{Problem 5}
    \begin{align*}
        e&=\frac {\rho_{U_{235}}}{\rho_{U_{235}}+\rho_{U_{238}}}  \\
            &= \frac{5\times 10^{21}}{4.35\times 10^{22}+5\times 10^{21}}\\
            &=10.3\%                                                     \\
        \eta &= \frac{e\cdot v \sigma_{f1}+(1-e)\cdot v \sigma_{f2}}{e\cdot \sigma_{a1}+(1-e)\cdot \sigma_{a2}}\\
            &=\frac {10.3\% \times2.42\times 582 \times 10^{-24}}{10.3\% \times 700 \times 10^{-24}+89.7\%\times 2.71\times 10^{-4}}\\
            &=1.946
    \end{align*}
\newpage
\section*{Problem 6}
    \begin{align*}
        %
        N_{Al}=\frac{\rho_{Al}\cdot N_A}{M_{Al}}=\frac{55\%\times 2.66\times 6.023\times 10^{23}}{26.9815}=3.266\times 10^{22}\  \frac{\text{atom}}{\text{cm}^3}\\
        N_{Si}=\frac{\rho_{Si}\cdot N_A}{M_{Si}}=\frac{45\%\times 2.66\times 6.023\times 10^{23}}{28.0855}=2.567\times 10^{22}\  \frac{\text{atom}}{\text{cm}^3}\\
        %
    \end{align*}
    \begin{align*}
        %
        \Sigma &= \sum N\cdot \sigma =N_{Al}\cdot \sigma_{Al}+N_{Si}\cdot \sigma_{Si}\\
                &=3.266\times 10^{22}\times 0.23\times 10^{-24} + 2.567\times 10^{22}\times 0.16\times 10^{-24}\\
                &=0.011619\ \text{cm}^{-1}\\
        \lambda &= \frac{1}{\Sigma}=\frac{1}{0.011619}=86.0659\  \text{cm}
        %
    \end{align*}
\section*{Problem 7}
    1. \ B\\
    2. \ C\\
    3. \ B\\
    4. \ A\\
    5. \ B\\
    6. \ A\\
    7. \ A\\
    8. \ C\\
    9. \ B\\
    10.\ C\\ 
\section*{Problem 8}
    A. Cause that fact neutrons tends to be scattered.\\
    B. Fertile materials cannot directly do fission while fissile materials can.\\
    C. Means that produced neutrons equals absorbed neutrons.\\
    D. They slows down fast neutrons into thermal newtrons.\\
    E. Increased amount of fissile materials to allow reactors to start.\\
    F. It's a parameter discribing the probability that certain reaction occurs.\\
    G. By making fertile atoms unstable and let is decay into a fissile atom.\\
    H. As atom's nucleus can easily split apart if it is struck by a neutron.\\
    I. Fast neutrons can be easily slowed down.\\
    J. Cause that the resonance is hard to model.\\



\newpage
\section*{Reference}
    All crossection data are retreived from 
    \href{https://wwwndc.jaea.go.jp/jendl/j33/j33.html}{Japan Atomic Energy Agency,
    Nuclear Data Center}.
\end{document}