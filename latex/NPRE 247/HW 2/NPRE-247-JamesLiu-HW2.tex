\documentclass{article}
\usepackage{graphicx}
\usepackage{amsmath}
\usepackage{array}
\usepackage[font=small, labelfont={sf,bf}, margin=1cm]{caption}
\usepackage{tabularx}
\usepackage{amssymb}
\usepackage[colorlinks,linkcolor=blue]{hyperref}
\title{\textbf{Homework \#2 }}
\author{James Liu}
\date{\ }

\begin{document}
\maketitle
\section*{Problem 1}
% \(R_X =\overline{\Sigma}_x\phi=N\overline{\sigma}_x\phi \). Therefore,\\ 
% \(\displaystyle{\overline{\sigma}_x=\frac{R_x}{N\phi}=\frac{R_x}{N\int^\infty_0\varphi(E)dE}}\)

\(R_x = \overline{\Sigma}_x\phi =\displaystyle{{\int^\infty_0 \Sigma_x(E)\varphi(E)dE}}\)\\
Substitude \(N\sigma \) for \(\Sigma\), there is \(R_x = N\overline{\sigma}_x \phi=N\displaystyle{{\int^\infty_0 \sigma_x(E)\varphi(E)dE}}\).\\
Therefore, there is \(\overline{\sigma}_x=\displaystyle{\frac{\int^\infty_0 \sigma_x(E)\varphi(E)dE}{\int^\infty_0\varphi(E)dE}}\)
\section*{problem 2}



\end{document}