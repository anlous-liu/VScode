\documentclass{article}
\usepackage{graphicx}
\usepackage{amsmath}
\usepackage{array}
\usepackage[font=small, labelfont={sf,bf}, margin=1cm]{caption}
\usepackage{tabularx}
\usepackage{amssymb}



\date{Due: Oct 25 Edit: \today}
\title{MATH 461 Homework 7}
\author{James Liu}

\begin{document}
\maketitle
\begin{itemize}
    \item [5.37]
    \begin{itemize}
        \item [a)]
        \[P(x<-\frac{1}{2})+P(x>\frac{1}{2})=\frac{1}{4}+\frac{1}{4}=\frac{1}{2}\]
        \item [b)] \(f(x)=\frac{1}{1-(-1)}=\frac{1}{2}\)
        \begin{align*}
            P(|x|\leq y)&=P(-y\leq x \leq y)\\
            &=\int_{y}^{-y}\frac{1}{2}\text{d}x\\
            &=y\\
            \frac{\text{d}}{\text{d}y}P(|x|\leq y)&=1\\
        \end{align*}
    \end{itemize}
    \item [5.39]
    \begin{align*}
        f(x)&=1e^{-1\times x}\\
        &=e^{-x}\\
        P(X\leq x)&=\int_{0}^{x}e^{-s}\text{d}s\\
        &=\left.(-e^{-s})\right|^{x}_{0}\\
        &=-e^{-x}-(-e^0)\\
        &=1-e^{-x}\\
        P(\log{X}<y)&=P(e^{\log(X)}\leq e^y)\\
        &=P(X\leq e^y)\\
        &=1-e^{-e^{y}}\\
        \frac{\text{d}}{\text{d}y} P(X\leq e^y)&=\frac{\text{d}}{\text{d}y} (1-e^{-e^{y}})\\
        &=(e^y)(e^{-e^y})
    \end{align*}
    \newpage
    \item [5.40]
    \begin{align*}
        f(x)&=\frac{1}{1-0}=1\\
        P(e^X\leq y)&= P(X\leq \log (y))\\
        &=\int_{0}^{\log(y)} 1 \text{ d}x\\
        &=\log(y)\\
        \frac{\text{d}}{\text{d}y} \log(y)&= \frac{1}{y} (1<y<e)
    \end{align*}
    \item [5.41]
    \begin{align*}
        f(\theta)&=\frac{1}{\frac{\pi}{2}+\frac{\pi}{2}}\\
        &=\frac{1}{\pi}\\
        \int_{-\frac{\pi}{2}}^{\theta} \frac{1}{\pi} \text{d} x&=\frac{1}{\pi}\left.(x)\right|^{\theta}_{-\frac{\pi}{2}}\\
        &=\frac{\theta+\frac{\pi}{2}}{\pi}\\
        P(A\sin(\theta)\leq r)&=P(\sin(\theta)\leq \frac{r}{A})\\
        (\text{for} |r|<A) &= P(\theta \leq sin^{-1}(\frac{r}{A}))\\
        \frac{\text{d}}{\text{d}r}\frac{sin^{-1}(\frac{r}{A})+\frac{\pi}{2}}{\pi}&=\frac{1}{\pi}\left(\frac{1}{\sqrt{1-\frac{r^2}{A^2}}}\times \frac{1}{A}\right)\\
        &=\frac{1}{\pi\sqrt{A^2-r^2}} (-A<r<A)
    \end{align*}
    \item [6.2]
    \begin{itemize}
        \item [a)]\begin{align*}
            \left\{\begin{matrix}
                \frac{\binom{5}{2}}{\binom{13}{2}} & (\text{Both balls are white})\\
                \frac{\binom{5}{1}\binom{8}{1}}{\binom{13}{2}}& (\text{\(X_1\) White, \(X_2\) red})\\
                \frac{\binom{8}{1}\binom{5}{1}}{\binom{13}{2}}& (\text{\(X_1\) red, \(X_2\) white})\\
                \frac{\binom{8}{2}}{\binom{13}{2}}& (\text{Both red})
            \end{matrix}\right.
        \end{align*}
        \item [b)]\begin{align*}
            \left\{\begin{matrix}
                \frac{\binom{5}{3}}{\binom{13}{3}} & (\text{all balls are white})\\
                \frac{\binom{5}{2}\binom{8}{1}}{\binom{13}{3}}& (\text{1 red 2 white})\\
                \frac{\binom{8}{2}\binom{5}{1}}{\binom{13}{3}}& (\text{2 red 1 white})\\
                \frac{\binom{8}{3}}{\binom{13}{3}}& (\text{all red})
            \end{matrix}\right.
        \end{align*}
    \end{itemize}
    \item [6.7]
    \begin{align*}
        P(X_1=x_1,X_2=x_2)&=P(X_1=x_1)P(X_2=x_2)\\
        &=p(1-p)^{x_1}\times p(1-p)^{x_2}\\
        &=p^2(1-p)^{x_1+x_2}
    \end{align*}
    \item [6.8]
    \begin{itemize}
        \item [a)]
        \begin{align*}
            1&=\iint R^2 f dxdy\\
            &=\int_{0}^{\infty}(\int_{-y}^{y}c(y^2-x^2)e^{-y}dx)dy\\
            &=\frac{4c}{3}\int_{0}^{\infty} y^3e^{-y} dy\\
            &=\frac{4c}{3}\times 6c\\
            c&=\frac{1}{8}
        \end{align*}
        \item [b)]
        \begin{align*}
            f_X(x)&=\int_{x}^{\infty}\frac{1}{8}(y^2-x^2)e^{-y}dy\\
            &=\frac{1}{8}\left(\int_{x}^{\infty}y^2e^{-y}dy-\int_{x}^{\infty}x^2e^{-y}dy\right)\\
            &=\frac{1}{8}\left(x^2e^{-x}+2xe^{-x}+2e^{-x}-x^2e^{-x}\right)\\
            &=\frac{1}{4}e^{-|x|}(1+|x|)\\
            f_Y(y) &=\frac{1}{8}\int_{-y}^{y}(y^2-x^2)e^{-y} dx\\
            &=\frac{1}{8}\frac{4}{3}y^3e^{-y}\\
            &=\frac{1}{6}y^3e^{-y}
        \end{align*}
        \item[c)]
        \begin{align*}
            E(X) &= \int_{-\infty}^{\infty}x f(x) dx \\
            &=\int_{-\infty}^{\infty}x \frac{1}{4}e^{-|x|}(1+|x|)dx\\
            &=0
        \end{align*}
    \end{itemize}
    \item [6.9]
    \begin{itemize}
        \item [a)]
        \begin{align*}
            \int_{0}^{1}\int_{0}^{2}\left(\frac{6}{7}\left(x^2+\left(\frac{xy}{2}\right)\right)\right)dydx &=\frac{6}{7} \int_{0}^{1}\int_{0}^{2}\left(x^2+\left(\frac{xy}{2}\right)\right)dydx\\
            &=\frac{6}{7}\int_{1}^{0}\left.x^2y+\frac{xy^2}{2}\right|^{y=2}_{y=0}dx\\
            &=\frac{6}{7}\int_{1}^{0}2x^2+\frac{4x}{4}dx\\
            &=\frac{6}{7}\frac{7}{6}\\
            &=1
        \end{align*} Thus, it is a density function.
        \item [b)]
        \begin{align*}
            \int_{0}^{2}\frac{6}{7}\left(x^2+\frac{xy}{2}\right) dy&=\frac{6}{7}\left(x^2y+\frac{xy^2}{2}\right)|^{2}_{0}\\
            &=\frac{6}{7}(2x^2+x)
        \end{align*}
        \item [c)]
        \begin{align*}
            \int_{0}^{1}\int_{0}^{x}\frac{6}{7}\left(x^2+\frac{xy}{2}\right) dydx&=\int_{0}^{1}\frac{6}{7}(x^3+\frac{x^3}{4}) dx\\
            &=\frac{6}{7}\left(\frac{1^4}{4}+\frac{1^4}{16}\right)\\
            &=\frac{15}{56}
        \end{align*}
        \item [d)]
        \begin{align*}
            \int_{0}^{0.5}\int_{0.5}^{2}\frac{6}{7}\left(x^2+\frac{xy}{2}\right) dydx = \frac{69}{80}
        \end{align*}
        \item [e)]
        \[\int_{0}^{1}x\frac{6}{7}(2x^2+x) dx = \frac{5}{6}\]
        \item [f)]
        \begin{align*}
            \int_{0}^{1}\frac{6}{7}(2x^2+x) dx &= \frac{6}{7}(\frac{1}{3}+\frac{y}{4})\\
            \int_{0}^{2}y \frac{6}{7}(\frac{1}{3}+\frac{y}{4}) dy &= \frac{8}{7}
        \end{align*}
    \end{itemize}
    \item [6.10]
    \begin{itemize}
        \item [a)] \(e^{-(x+y)}=e^{-x}e^{-y}\) Thus, \(f_X(x)=e^{-x},f_Y(y)=e^{-y}\)
        \begin{align*}
            \int_{0}^{\infty}\int_{x}^{\infty}e^{-x}e^{-y} dx dy &=\frac{1}{2}
        \end{align*}
        \item [b)]
        \begin{align*}
            \int_{a}^{\infty} e^{-x} dx &=1-e^{-a}
        \end{align*}
    \end{itemize}
\end{itemize}

\end{document}