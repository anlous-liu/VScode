\documentclass{article}
\usepackage{graphicx}
\usepackage{amsmath}
\usepackage{array}
\usepackage[font=small, labelfont={sf,bf}, margin=1cm]{caption}
\usepackage{tabularx}
\usepackage{amssymb}



\date{Due: Oct 18 Edit: \today}
\title{MATH 461 Homework 6}
\author{James Liu}

\begin{document}
\maketitle
\begin{itemize}
    \item [5.6]
    \begin{itemize}
        \item [a)]
        \[\int_{0}^{\infty}\frac{1}{4}x^2e^{-\frac{x}{2}}\text{d}x=4\]
        \item [b)]
        \begin{align*}
            \int_{-1}^{1}cx(1-x^2)\text{d}x &= 0\\
            \text{As it is an odd function.}
        \end{align*}
        \item [c)]
        \begin{align*}
            \int_{5}^{\infty} x\times \frac{5}{x^2}\text{d}x&=5(\ln(\infty)-\ln(5))=\infty
        \end{align*}
    \end{itemize}
    \item [5.10]
    \begin{itemize}
        \item [a)]
        \begin{align*}
            \int_{5}^{15}\frac{1}{60}\text{d}x+\int_{20}^{30}\frac{1}{60}\text{d}x+\int_{35}^{40}\frac{1}{60}\text{d}x+\int_{50}^{60}\frac{1}{60}\text{d}x&= 4\times \frac{10}{60}=\frac{2}{3}
        \end{align*}
        \item [b)]
        \begin{align*}
            \int_{10}^{15}\frac{1}{60}\text{d}x+\int_{20}^{30}\frac{1}{60}\text{d}x+\int_{35}^{40}\frac{1}{60}\text{d}x+\int_{50}^{60}\frac{1}{60}\text{d}x+\int_{65}^{70}\frac{1}{60}\text{d}x&= 3\times \frac{10}{60}+2\times\frac{5}{60}=\frac{2}{3}
        \end{align*}
    \end{itemize}
    \item [5.12] \
    \begin{itemize}
        \item [Case 1:]
        \begin{align*}
        E(Y)&= \left(\left(\int_{0}^{25}|x-0|\text{d}x\right)+\left(\int_{25}^{75}|x-50|\text{d}x\right)+\left(\int_{75}^{100}|x-100|\text{d}x\right)\right)\times \frac{1}{100}\\
            &= 12.5
        \end{align*}
        \item [Case 2:] 
        \begin{align*}
            E(Y)&= \left(\left(\int_{0}^{37.5}|x-25|\text{d}x\right)+\left(\int_{37.5}^{62.5}|x-50|\text{d}x\right)+\left(\int_{62.5}^{100}|x-75|\text{d}x\right)\right)\times \frac{1}{100}\\
                &= 9.375 < 12.5
            \end{align*}
    \end{itemize}
    Thus, it is right.
    \item [5.13]
    \begin{itemize}
        \item [a)]
        \[\int_{10}^{30}\frac{1}{30}\text{d}x=\frac{2}{3}\]
        \item [b)]
        \[\frac{\int_{25}^{30}\frac{1}{30}\text{d}x}{\int_{0}^{15}\frac{1}{30}\text{d}x}=\frac{5}{15}=\frac{1}{3}\]
    \end{itemize}
    \item [5.15]
    \begin{itemize}
        \item [a)]\begin{align*}
            P(X > 5) &= P\left(\frac{X - \mu}{\sigma} > \frac{5 - 10}{6}\right) = P\left(z > \frac{5 - 10}{6}\right) \\
            &= 1 - P\left(z \leq -0.8333\right)\\
            &=1 - 0.2023 = 0.7977
        \end{align*}

        \item [b)]\begin{align*}
            P(4 < X < 16) &= P\left(\frac{4 - 10}{6} < \frac{X - \mu}{\sigma} < \frac{16 - 10}{6}\right) \\ &= P(-1 < z < 1)=P(z < 1) - P(z < -1) \\&= 0.8413 - 0.1587 = 0.6826
        \end{align*}
        \item [c)]
        \[P(X < 8) = P\left(\frac{X - \mu}{\sigma} < \frac{8 - 10}{6}\right) = P\left(z < -0.3333\right)=0.3695\]
        \item [d)]
        \[P(X < 20) = P\left(\frac{X - \mu}{\sigma} < \frac{20 - 10}{6}\right) = P(z < 1.6667)=0.9522\]
        \item [e)]
        \begin{align*}
            P(X > 16) &= P\left(\frac{X - \mu}{\sigma} > \frac{16 - 10}{6}\right) = P(z > 1)\\&=1 - P(z \leq 1) = 1 - 0.8413 = 0.1587
        \end{align*}
    \end{itemize}
    \item [5.18]
    \begin{align*}
        P(\frac{x-\mu}{\sigma}>\frac{9-5}{\sigma})=0.2\\
        1-P(z\leq \frac{4}{\sigma}) = 0.2\\
        P(z\leq \frac{4}{\sigma})=0.84\\
        \sigma = 4/0.84 = 4.761\\
        \sigma^2 = 22.675
    \end{align*}
    \item [5.21]
    \begin{itemize}
        \item [a)]
        \[P(X>74) = P(\frac{X-\mu}{\sigma})>\frac{74-\mu}{\sigma}=P(z>\frac{74-71}{2.5})=1-P(z\leq 1.2)=1-0.8849 = 0.115\]
        \item [b)]
        \[P(X>77) = P(\frac{X-\mu}{\sigma})>\frac{77-\mu}{\sigma}=P(z>\frac{77-71}{2.5})=1-P(z\leq 2.4)=1-0.9918 = 0.0082\]
        \[P(X>72) = P(\frac{X-\mu}{\sigma})>\frac{72-\mu}{\sigma}=P(z>\frac{72-71}{2.5})=1-P(z\leq 0.4)=1-0.6554 = 0.3466\]
    \end{itemize}
    \item [5.23]
    \begin{itemize}
        \item [a)]
        \begin{align*}
            \mu &= np = 166.6667\\
            \sigma &= \sqrt{np(1-p)} = 11.7851\\
            P\left(150 \leq X \leq 200\right) &= P\left(\frac{150 - \mu }{\sigma } \leq \frac{X - \mu }{\sigma } \leq \frac{200 - \mu }{\sigma }\right)\\ &= P\left(\frac{150 - 0. 5}{11. 785} \leq z \leq \frac{200 - 0. 5}{11. 7851}\right)\\ 
            &= P\left(\frac{150 - 0. 5 - 166. 6667}{11. 785} \leq z \leq \frac{200 + 0. 5 - 166. 6667}{11. 7851}\right) \\
            &= P\left(\frac{149. 5 - 166. 6667}{11. 785} \leq z \leq \frac{200. 5 - 166. 6667}{11. 7851}\right)\\ &= P\left( - 1. 46 \leq z \leq 2. 87\right) \\&= \varphi \left(2. 87\right) - \varphi \left( - 1. 46\right) = 0. 998 - 0. 072 = 0. 9258
        \end{align*}
        \item [b)]
        \begin{align*}
            \mu &= np = 800\times \frac{1}{5}=160\\
            \sigma &= \sqrt{np(1-p)} = 11.314\\
            P(X<150)&= p(z<\frac{150-0.5-160}{11.314})\\
            &=P(z<-0.92) = 0.17
        \end{align*}
    \end{itemize}
    \item [5.25]
    \begin{align*}
        \mu = np = 150\times 0.05 = 7.5\\
        \sigma = \sqrt{np(p-1)}=2.67\\
        z = \frac{10-7.5}{2.67}=0.938\\
        P(z\leq 0.938) = 0.826
    \end{align*}
    \item [5.28]
    \begin{align*}
        \mu = np &= 200\times 0.12 = 24\\
        \sigma &= \sqrt{np(p-1)}=4.595\\
        z = \frac{20-24}{4.595}&=-0.8703\\
        P(z\geq 0.8703) &= 1-P(z\leq -0.8703) \\ &= 1-0.273173 = 0.726827
    \end{align*}
    \item [5.32]
    \begin{itemize}
        \item [a)]    \begin{align*}
            P(X>2) = \exp(-2\lambda ) = e^{-1}= 0.3678
        \end{align*}
        \item [b)] 
        \begin{align*}
            \frac{P(X\geq10 \text{ and } X>9)}{P(X>9)} = \frac{\exp(-5)}{\exp(-4.5)} = 0.6065
        \end{align*}
    \end{itemize}
    \item [5.33]
    \[e^{-1}=0.36787\]
    \item [5.40]
    \begin{align*}
        F_Y(y)=P(Y\leq y)=P(e^X\leq y)=P(X\leq \ln(y))=F_X(\ln(y))\\
        F_Y(y)' = F_X(\ln(y))\frac{1}{y}\\
    \end{align*}
    As \(X\) is uniform on \((0,1)\), then \(f_Y = \frac{1}{y}\) when \(1<y<e\) and 0 other wise.
\end{itemize}

\end{document}