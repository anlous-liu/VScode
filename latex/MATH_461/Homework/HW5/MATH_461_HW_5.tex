\documentclass{article}
\usepackage{graphicx}
\usepackage{amsmath}
\usepackage{array}
\usepackage[font=small, labelfont={sf,bf}, margin=1cm]{caption}
\usepackage{tabularx}
\usepackage{amssymb}



\date{Due: Sep 27  Edit: \today}
\title{MATH 461 Homework 4}
\author{James Liu}

\begin{document}
\maketitle
\begin{itemize}
    \item [4.72]
    \begin{align*}
        P(i=4) & = \binom{4}{4} \times (0.6)^4 =0.1296\\
        P(i=5) & = (\binom{5}{4}-\binom{4}{4}) \times (0.6)^4(0.4) = 0.2592\\
        P(i=5) & = (\binom{6}{4}-\binom{5}{4}) \times (0.6)^4(0.4)^2 = 0.2903\\
        P(i=5) & = (\binom{7}{4}-\binom{5}{4}) \times (0.6)^4(0.4)^3= 0.28201\\
        P(2\text{in}2) &=0.6^2 = 0.36\\
        P(2\text{in}3) &=(\binom{3}{2}-\binom{2}{2}) 0.6^20.4 = 0.288\\
        P(4win) = \sum P(i) &= 0.9092> 0.648 = P(2win)
    \end{align*}
    The stronger team has a high rate of wining when it needs to win 4 times compared with 2. Subtraction is due to that it counts the combinations that the games ends before \(i\).
    \item [4.73]
    \begin{align*}
        P(4) &=\binom{2}{1}\binom{4}{4} 0.5^40.5^0 = 0.125\\
        P(5) &=\binom{2}{1}(\binom{5}{4}-\binom{4}{4}) 0.5^405^1 = 0.25\\
        P(6) &=\binom{2}{1}(\binom{6}{4}-\binom{5}{4}) 0.5^405^2  = 0.3125\\
        P(6) &=\binom{2}{1}(\binom{7}{4}-\binom{6}{4}) 0.5^405^3  = 0.3125\\
        E(X) &=4P(4)+5P(5)+6P(6)+7P(7) = 5.8125
    \end{align*}
    \item [4.77]
    For combinations of first \(N-(N-K)\) experiment, the last trial must be from the box needed to be emptied, thus, finding combination like \(\binom{2n-k-1}{n-k}\) give all the possible ways of arranging such experiments. And there are 2 boxes, the answer should be:\[2\cdot \binom{2n-k-1}{n-k}(0.5^{2n-k})\]
    \item [4.78]
    \begin{align*}
        P(win) &= \binom{4}{2}\frac{4\cdot3\cdot4\cdot3}{8\cdot7\cdot6\cdot5} = \frac{18}{35} = 0.514286\\
        P(n) & = (1-P(win))^{n-1}P(win) = \left(\frac{17}{35}\right)^{n-1}\frac{18}{35}
    \end{align*}
    \item [4.79]
    \begin{itemize}
        \item [a)]
        \begin{align*}
            P(X=0) = \frac{\binom{94}{10}}{\binom{100}{10}} = 0.5223\\
        \end{align*}
        \item [b)]
        \begin{align*}
            P(X=1) &= \frac{\binom{94}{9}\binom{6}{1}}{\binom{100}{10}}=0.3686\\
            P(X=2) &= \frac{\binom{94}{8}\binom{6}{2}}{\binom{100}{10}}=0.096458\\
            P(X>2) &=1-P(X=0,1,2) = 1 - 0.5223 - 0.3686 - 0.0964 = 0.012551
        \end{align*}
    \end{itemize}
    \item [4.84]
    \begin{itemize}
        \item [a)]
        For each box, the probability of it being empty after 10 balls is \(P_i(empty) = (1-P_i)^{10}\)\\
        Thus \(E(X) = \sum_{i=1}^{5}1\times(1-p_i)^{10}\)
        \item [b)]
        Similarly: \(E(X) = \sum_{i=1}^{5}\binom{10}{1}(1-p_i)^{9}p_i\)
    \end{itemize}
    \item [4.85] \(E(X) = \sum_{i=1}^{k}1-(1-p_i)^n\)
    \item [5.1] \(f(x)\) is a probability distribution function then \(\int f \text{d}x = 1\)
    \begin{itemize}
        \item [a)]    \begin{align*}
            \int_{-1}^{1} c(1-x^2) \text{d}x = 1\\
            \left.x-\frac{1}{3}x^3\right|^{1}_{-1}=\frac{1}{c}\\
            c = \frac{3}{4}
        \end{align*}
        \item [b)]
        \begin{align*}
            \int^x_{-1}\frac{3}{4}(1-s^2)\text{d}s&=\
            \frac{3}{4}\left.(s-\frac{1}{3}s^3)\right|^x_{-1}\\
            &=\frac{3}{4}(x-\frac{1}{3}x^3)-\frac{3}{4}(-1+\frac{1}{3})\\
            &=\frac{3}{4}(-\frac{1}{3}x^3+x+\frac{2}{3})\  (-1<x<1) \text{otherwise 0}
        \end{align*}
    \end{itemize}

    \item [5.2]
    \begin{align*}
        \int_0^{\infty} Cxe^{-x/2}\text{d}x&=1\\
        \left.\left((-2x-4)e^{-x/2}\right)\right|^\infty_0&=\frac{1}{c}\\
        0-(0-4) &= \frac{1}{c}\\
        c&= \frac{1}{4}\\
        \int_{0}^{5}\frac{xe^{-x/2}}{4}&=\frac{1}{4}\left.\left((-2x-4)e^{-x/2}\right)\right|^5_0\\
        &=0.7127
    \end{align*}
    \item [5.4]
    \begin{itemize}
        \item [a)]
        \begin{align*}
            \int_{20}^{\infty}\frac{10}{x^2}\text{d}x&=\left.-\frac{10}{x}\right|^{\infty}_{20}\\
            P(X>20)&=0.5
        \end{align*}
        \item [b)]
        \begin{align*}
            \int_{10}^{x}\frac{10}{s^2}\text{d}s&=\left.-\frac{10}{s}\right|^{x}_{10}\\
            &=1-\frac{10}{x} (10<x) \text{otherwise 0}
        \end{align*}
        \item [c)]If assume that the time of functioning is independent, then we have:
        \begin{align*}
            P(X>15) & = 0-\frac{10}{15} = \frac{2}{3}
        \end{align*}
        \[P(X\geq3) = \binom{6}{3}(\frac{1}{3})^3(\frac{2}{3})^3+\binom{6}{4}(\frac{1}{3})^2(\frac{2}{3})^4+\binom{6}{5}(\frac{1}{3})^1(\frac{2}{3})^5+\binom{6}{6}(\frac{1}{3})^0(\frac{2}{3})^6=0.89986\]
    \end{itemize}
    \item [5.5]
    \begin{align*}
        \int_{0}^{x}5(1-s)^4\text{d}s &= 0.99\\
        -(1-s)^5|^x_0&=0.99\\
        1-(1-x)^5&=0.99\\
        x&=0.6018\\
    \end{align*}
    Thus, 601.8 gallons.
\end{itemize}

\end{document}