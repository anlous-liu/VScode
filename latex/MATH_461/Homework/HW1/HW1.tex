\documentclass{article}
\usepackage{graphicx}
\usepackage{amsmath}
\usepackage{array}
\usepackage[font=small, labelfont={sf,bf}, margin=1cm]{caption}
\usepackage{tabularx}
\usepackage{amssymb}



\date{Due: Sep 6  Edit: \today}
\title{MATH 461 Homework 1}
\author{James Liu}

\begin{document}
\maketitle

4. John, Jim, Jay, and Jack have formed a band consisting of 4 instruments. 
If each of the boys can play all 4
instruments, how many different arrangements are possible? 
What if John and Jim can play all 4 instruments, but
Jay and Jack can each play only piano and drums?
\\
a) \(4!=24\)\\
b) \(2!\times2!=4\)
\\ \\
5. For years, telephone area codes in the United States and
Canada consisted of a sequence of three digits. The first
digit was an integer between 2 and 9, the second digit was
either 0 or 1, and the third digit was any integer from 1 to
9. How many area codes were possible? How many area
codes starting with a 4 were possible?\\
a) \(8\times2\times9=144\)\\
b) \(2\times9=18\)


\end{document}