\documentclass{article}
\usepackage{graphicx}
\usepackage{amsmath}
\usepackage{array}
\usepackage[font=small, labelfont={sf,bf}, margin=1cm]{caption}
\usepackage{tabularx}
\usepackage{amssymb}



\date{Due: Nov 1 Edit: \today}
\title{MATH 461 Homework 8}
\author{James Liu}

\begin{document}
\maketitle
\begin{itemize}
    \item [6.48]
    \begin{itemize}
        \item [a)]
        \begin{align*}
            P(\min(X_1,X_2,X_3,X_4,X_5)\leq a)&=1-P(\min(X_1,X_2,X_3,X_4,X_5)>a)\\
            &=1-(e^{-(a\lambda)})^5
        \end{align*}
        \item [b)]
        \begin{align*}
            P(\max(X_1,X_2,X_3,X_4,X_5)\leq a)&=1-P(X_1\leq a,X_2\leq a,X_3\leq a,X_4\leq a,X_5\leq a)\\
            &=(e^{-(a\lambda)})^5
        \end{align*}
    \end{itemize}
    \item [7.5]
    \begin{align*}
        E(|x|+|y|)&=E(|x|)+E(|y|)\\
        &=\int_{-1.5}^{1.5}\frac{|x|}{3}dx+\int_{-1.5}^{1.5}\frac{|y|}{3}dy\\
        &=1.5
    \end{align*}
    \item [7.6] \[\frac{1}{6}\times(1+2+3+4+5+6)=\frac{7}{2}\]
    \item [7.7]
    \begin{itemize}
        \item [a)]
        \[0.3\times0.3=0.09\]
        \[10\times 0.09 = 0.9\]
        \item [b)]
        \[10\times (1-0.3)^2=4.9\]
        \item [c)]
        \[2\times (0.3\times 0.7)\times 10 = 4.2\]
    \end{itemize}
    \item [7.8]
    By noteing in the way, the total table occupied will be: \(\sum X_i\), 
    \begin{align*}
        E[X_i] = (1-p)^{i-1}\\
        E = \sum_{i=1}^{n}(1-p)^{i-1}
    \end{align*}
    \item [7.11]
    any change over will have a probability of \(2p(1-p)\) as they are landing are different sides. And there are total of \(n-1\) slots that are possible for flips, then the total probability is \((n-1)2p(1-p)\)
    \item [7.18]
    \(52\times \frac{1}{13}=4\)
    \item [7.19]
    \begin{itemize}
        \item [a)]Probability of chatching \(j\) before getting type 1 is \((1-p_1)^jp_1\), then the \(E((1-p_1)^jp_1)=\sum_{j=0}^{\infty}j(1-p_1)^jp_1=p_1\frac{1-p_1}{p_1^2}=\frac{1-p_1}{p_1}\)
        \item [b)]It will \(\sum_{j=2}^{n}\frac{p_j}{p_j+p_1}\)
    \end{itemize}
    \item [7.21]
    \begin{itemize}
        \item [a)]\[365\times \binom{100}{3}\left(\frac{1}{365}\right)^3\left(\frac{364}{365}\right)^{97}\]
        \item [b)]\[365 \times \left(1-\left(\frac{364}{365}\right)^{100}\right)\]
    \end{itemize}
    \item [7.30]
    \begin{align*}
        E\left[\left(X-Y\right)^2\right]&=E\left[X^2-2XY+Y^2\right]\\
        &=E\left[X^2\right]-2E\left[X\right]E\left[Y\right]+E[Y^2]\\
        &=(\sigma^2+\mu^2)-2\mu\mu+(\sigma^2+\mu^2)\\
        &=\sigma^2
    \end{align*}
    \item [7.31]
    \begin{align*}
        \text{var}(X) &= E[X^2]-(E[X])^2\\
        &=10\times \frac{1}{6}(1+4+9+16+25+36)-10\times( \frac{1}{6}(1+2+3+4+5+6))^2\\
        &=\frac{175}{6}
    \end{align*}
    \item [7.33]
    \begin{itemize}
        \item [a)]    
        \begin{align*}
            E\left[(2+X)^2\right]&= E[4+4X+X^2]\\
            &=4+(5+1)+4\\
            &=14
        \end{align*}
        \item [b)]
        \begin{align*}
            \text{var}(4+3X)&=3^2\text{var}(X)\\
            &=9\times 5\\
            &=45
        \end{align*}
    \end{itemize}
    \item [7.38]
    \begin{align*}
        E(XY)&=\int_{0}^{\infty}\int_{0}^{x}xy\frac{2}{x}e^{-2x}dydx\\
        &=\frac{1}{4}\\
        E(X)&=\int_{0}^{\infty}\int_{0}^{x}x\frac{2}{x}e^{-2x}dydx\\
        &=\frac{1}{2}\\
        E(Y)&=\int_{0}^{\infty}\int_{0}^{x}y\frac{2}{x}e^{-2x}dydx\\
        &=\frac{1}{4}\\
        \text{cov}(X,Y)&=E(XY)-E(X)E(Y)=\frac{1}{4}-\frac{1}{2}\frac{1}{4}\\
        &=\frac{1}{8}
    \end{align*}
    \item [7.39]
    \begin{align*}
        j=0: \text{cov}(Y_n,Y_{n})&=\sigma^3\\
        j=1: \text{cov}(Y_n,Y_{n+1})&=\sigma^2\\
        j=2: \text{cov}(Y_n,Y_{n+2})&=\sigma\\
        j\geq 3: \text{cov}(Y_n,Y_{n+j})&=0
    \end{align*}
    \item [7.41]The assumptions made are 
    follows Hyper geometric distribution with m=30,n=20,N=100.
    \begin{align*}
        \mu &=0.3\times 20\\
        &=6\\
        \text{var}(X)&=\frac{mn}{N} \left[ \frac{(n-1)(m-1)}{N-1} + 1 - \frac{mn}{N} \right] \\
        &= \frac{30 \times 20}{100} \left[ \frac{(20 - 1)(30 - 1)}{100 - 1} + 1 - \frac{30 \times 20}{100} \right] \\
        &= \frac{600}{100} \left[ \frac{19 \times 29}{99} + 1 - \frac{600}{100} \right] \\
        &= \frac{600}{100} \left[ \frac{551}{99} + 1 - \frac{600}{100} \right] \\
        &= \frac{600}{100} \left[ \frac{56}{99} \right] \\
        &= \frac{112}{33}
    \end{align*}
    \item [7.42]
    \begin{align*}
        E(X)&=\frac{10}{19}
        \text{var}&=\frac{3240}{6137}
    \end{align*}
\end{itemize}

\end{document}