\documentclass{article}
\usepackage{graphicx}
\usepackage{amsmath}
\usepackage{array}
\usepackage[font=small, labelfont={sf,bf}, margin=1cm]{caption}
\usepackage{tabularx}
\usepackage{amssymb}



\date{Due: Sep 20  Edit: \today}
\title{MATH 461 Homework 3}
\author{James Liu}

\begin{document}
\maketitle
\subsection*{Part 1}
\begin{itemize}
    \item [3.57]
    \begin{itemize}
        \item [a)] \(\displaystyle
        P(2S) = 2\times p(1-p)
        \)
        \item [b)]\(\displaystyle
        P(3I) = 3\times p^2(1-p)
        \)
        \item [c)]\(\displaystyle
        P = \frac{2p^2(1-p)}{3p^2(1-p)}=\frac{2}{3}
        \)
    \end{itemize}
    \item [3.59]
    \begin{itemize}
        \item [a)]\(\displaystyle
        P = p^4
        \)
        \item [b)]\(\displaystyle
        P = p^3(1-p)
        \)
        \item [c)]\(\displaystyle
        1-p^4
        \) as something else appear in first 4 toses which means at least a T is produced
    \end{itemize}
    \item [3.64]
    \begin{itemize}
        \item [a)]\(\displaystyle
        P_1 = P(C) = p
        \)
        \item [b)]\(\displaystyle
        P_2 = P(C) = p^2+0.5\times 2p(1-p)
        \)
    \end{itemize}
    \(P_2-P_1 = p^2+p(1-p)-p=0\), thus both strategy shall have same probability of wining.
    \item [3.66]
    \begin{itemize}
        \item [a)]\(P = P(C1C2C\cup C3C4C5) = (p_1p_2+p_3p_3-p_1p_2p_3p_4)p_5\) 
        \item [b)]\begin{align*}
P(E) &= P(C_1C_4 \cup C_2C_5 \cup C_3C_1C_5 \cup C_3C_2C_4)\\
&= P\left[C_3^c(C_1C_4 \cup C_2C_5) \cup C_3(C_1C_4 \cup C_2C_5 \cup C_1C_5 \cup C_2C_4)\right]\\
&= P(C_3^c)P(C_1C_4 \cup C_2C_5) + P(C_3)P(C_1C_4 \cup C_2C_5 \cup C_1C_5 \cup C_2C_4)\\
&=p_1p_4 + p_2p_5 + p_3(p_1p_5 + p_2p_4) - (p_1p_2p_3p_4 + p_1p_2p_3p_5 \\&+ p_1p_3p_4p_5 + p_2p_3p_4p_5) + 2p_1p_2p_3p_4p_5
                \end{align*}
    \end{itemize}
    \item [3.78]
    \begin{itemize}
        \item [a)] \(P = 2\times p^3(1-p)+2\times p(1-p)^3\)
        \item [b)] \(P = \frac{p^2}{p^2+(1-p)^2}\)
    \end{itemize}
    \item [3.81]\(\displaystyle
    P = \frac{0.55^{15}}{0.45^{15}+0.55^{15}} = 95.30\%
    \)
    \item [3.83]\begin{itemize}
        \item [a)]\(P = 0.5\times \frac{4}{6}+0.5\times\frac{2}{6}=\frac{3}{6}=\frac{1}{2}\)
        \item [b)]\(P = \frac{3}{5}\)
        \item [c)]\(P = \frac{4}{5}\)
    \end{itemize}
    \item [3.84]
    \begin{itemize}
        \item [a)]\(\displaystyle P(A) = \frac{1}{3}\sum_{i=1}^{\infty} \left(\frac{2}{3}\right)^{3(i-1)} = \frac{1}{3}\times\frac{-1}{\frac{8}{27}-1}=\frac{9}{19} \\
        P(B) = \frac{2}{9}\sum_{i=1}^{\infty} \left(\frac{2}{3}\right)^{3(i-1)} = \frac{2}{9}\times\frac{-1}{\frac{8}{27}-1}=\frac{6}{19}\\
        P(C) = \frac{19-9-6}{19} = \frac{4}{19} 
        \)
        
        \item [b)]\(\displaystyle
        P(A) =  P(A1)+P(A2)+P(A3) = \frac{7}{15}\\
        P(B) =  P(B1)+P(B2)+P(A3) = \frac{68}{165}\\
        P(C) =  1-P(A)-P(B)= \frac{4}{33}
        \)
\end{itemize}
\end{itemize}
\subsection*{Part 2}
\begin{itemize}
    \item [4.1] \(\binom{3}{2} = 6\) types of X.
    \begin{itemize}
        \item [X=4:] \(P(X=4)= \frac{4}{14}\frac{3}{13} = \frac{6}{91}\approx 6.59\%\)
        \item [X=2:] \(P(X=2)= \frac{4}{14}\frac{2}{13}+\frac{2}{14}\frac{4}{13} = \frac{\binom{4}{1}\binom{2}{1}}{\binom{14}{2}} = \frac{8}{91} \approx 8.79\%\)
        \item [X=1:] \(P(X=1)= \frac{\binom{4}{1}\binom{8}{1}}{\binom{14}{2}} = \frac{32}{91}\approx 35.16\%\)
        \item [X=0:] \(P(X=0)= \frac{\binom{2}{2}}{\binom{14}{2}}=\frac{1}{91}=\approx 1.10\%\)
        \item [X=-1:]\(P(X=-1)= \frac{\binom{8}{1}\binom{2}{1}}{\binom{14}{2}}=\frac{16}{91}\approx 17.58\%\)
        \item [X=-2:]\(P(X=-2)=\frac{\binom{8}{2}}{\binom{14}{2}}=\frac{28}{91}=\approx 30.77\%\)
    \end{itemize}
    \item [4.4] \ 
    \begin{itemize}
        \item [\(X=1\):] \(P(X=1) = \frac{\binom{5}{1}9!}{10!} = \frac{1}{2}=50\%\)
        \item [\(X=2\):] \(P(X=2) = \frac{\binom{5}{1}\binom{5}{1}8!}{10!} = \frac{5}{18}\approx27.78\%\)
        \item [\(X=3\):] \(P(X=3) = \frac{2!\times \binom{5}{2}\binom{5}{1}\times 7!}{10!}=\frac{5}{36} \approx 13.89\%\)
        \item [\(X=4\):] \(P(X=4) = \frac{3!\times \binom{5}{3}\binom{5}{1}\times 6!}{10!}=\frac{5}{84} \approx 5.95\%\)
        \item [\(X=5\):] \(P(X=5) = \frac{4!\times \binom{5}{4}\binom{5}{1}\times 5!}{10!}=\frac{5}{252} \approx 1.98\%\)
        \item [\(X=6\):] \(P(X=6) = \frac{5!\times \binom{5}{5}\binom{5}{1}\times 4!}{10!}=\frac{1}{252} \approx 0.40\%\)
        \item [\(X>6\):] \(P(X>6) = 0\)
    \end{itemize}
    \item [4.5] \(X\in\{y|y=(n-2t),t\in\mathbb{Z},(n-2t)\geqslant0\}\)
\end{itemize}
\subsection*{Part 3}
\begin{itemize}
    \item [4.13] There are total of \(5\) possible variants of \(X\)'s value.
    \begin{itemize}
        \item [\(P(X=0)=\)] \(0.7\times0.4=0.28\)
        \item [\(P(X=500)=\)] \(0.5\times(0.3\times0.4+0.7\times0.6) =0.27 \)
        \item [\(P(X=1000)=\)] \(0.5\times(0.3\times0.4+0.7\times0.6)+0.5^2\times0.3\times0.6=0.315\)
        \item [\(P(X=1500)=\)] \(0.3\times0.6(0.5^2)\times 2! = 0.09\)
        \item [\(P(X=2000)=\)] \(0.3\times0.6(0.5^2) = 0.045\)
    \end{itemize}
    \item [4.14] \
    \begin{itemize}
        \item [\(P(X=4)=\)] \(\frac{1}{5} = 20\%\)
        \item [\(P(X=3)=\)] \(\frac{3!}{5!} = \frac{1}{20}=5\%\)
        \item [\(P(X=2)=\)] \(\frac{1\times\binom{3}{2}\times2!}{5!}+\frac{1\times \binom{2}{2}\times 2!\times 2!}{5!}=\frac{1}{12}\approx8.33\%\)
                            % 1 gets card of the 4th largest card        gets the third largest card           
        \item [\(P(X=1)=\)] \(\frac{1\times\binom{3}{1}\times1\times2!}{5!}+\frac{1\times \binom{2}{1}\binom{2}{1}\times 2!}{5!}+\frac{1\times1\times\binom{3}{1}\times 2!}{5!}=\frac{1}{6}\approx16.67\%\)
                            %1 gets 4                       1 gets 3                                1 gets 2
        \item [\(P(X=0)=\)] \(\frac{\binom{4}{1}\times 3!}{5!}+\frac{\binom{3}{1}\times 3!}{5!}+\frac{\binom{2}{1}\times 3!}{5!}+\frac{3!}{5!} = 0.5 = 50\%\)
    \end{itemize}
\end{itemize}
\subsection*{Part 4}
\begin{itemize}
    \item [4.17]\
    \begin{itemize}
        \item [a)]\
        \begin{itemize}
            \item [\(P(X=1)=\)] \(\frac{1}{2}+\frac{0}{4}-\frac{1}{4}=\frac{1}{4}\)
            \item [\(P(X=2)=\)] \(\frac{11}{12}-\frac{1}{2}-\frac{1}{4}=\frac{1}{6}\)
            \item [\(P(X=3)=\)] \(1-\frac{11}{12}=\frac{1}{12}\)
        \end{itemize}
            \item [b)] \[P(E) = P(\frac{3}{2})-P(\frac{1}{2})=\left(\frac{1}{2}+\frac{\frac{1}{2}}{4}\right)-\frac{\frac{1}{2}}{4}=\frac{1}{2}\]
    \end{itemize}
    \item [4.19] \[\left\{\begin{matrix}
        \frac{1}{2}&&x=0\\ \\
        \frac{1}{10}&&x=1\\ \\
        \frac{1}{5}&&x=2\\ \\
        \frac{1}{10}&&x=3\\ \\
        \frac{1}{10}&&x=3.5\\ \\
    \end{matrix}\right.\]
\end{itemize}

\end{document}