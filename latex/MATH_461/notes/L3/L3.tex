\documentclass{article}
\usepackage{graphicx}
\usepackage{amsmath}
\usepackage{array}
\usepackage[font=small, labelfont={sf,bf}, margin=1cm]{caption}
\usepackage{tabularx}
\usepackage{amssymb}



\date{Edit: \today, Class: Aug 28}
\title{MATH 461 Lecture 2 Note}
\author{James Liu}

\begin{document}
\maketitle
\section{Probability}
\subsection{Definitions}
\begin{itemize}
    \item [\textbf{Sample Space:}] A sample space is a set containing all possible out come of a experiment.
    \item [\textbf{Event:}] A event is a subspace of the Sample space
\end{itemize}
\subsection{Axioms of probabilities}
\(E\): Event set, \(S\): Sample Space, \(P(x)\): a map between a set to probability 
\begin{itemize}
    \item [1)]\(E\subseteq S\), \(0\leqslant P(E)\leqslant1\)
    \item [2)]\(P(S)=1\)
    \item [3)] For any sequence of \(E_1,E_2,E_3,\cdots,E_n\) of events that are disjoint
        \\ \(P(\bigcup _{i=1}^\infty E_i)=\sum_{i=1}^\infty P(E_i)\). Or, the probability of combined disjoint event equals to the sum of the probabilities when they are seperate.
\end{itemize}
\subsubsection*{Exaples}
Say we measure the liftime of a light builb. Assume that \(P(A) = \int_A e^{-t} \text{d}t\),
where the \(A\) is a time interval \([0,\infty)\) as \(\int_{0}^{\infty}e^{-t}\text{d}t=1\), also other 
function \(P\) can be chosed.
\subsection{properties}
\begin{itemize}
    \item [1.] \(P(\varnothing )=0\)
    \item [2.] If \(E_1,E_2,E_3,\cdots,E_n\) are disjoint then \(P(\bigcup _{i=1}^n E_i)=\sum_{i=1}^\infty P(E_i)\)
    \item [3.] If \(E\subseteq F\), then \(P(E)\leqslant P(F)\)
    \item [4.] \(P(E^c)=1-P(E)\)
    \item [5.] \(P(\bigcup _{i=1}^n E^c_i)=1-P(\bigcup _{i=1}^n E_i)\)
\end{itemize}














\end{document}