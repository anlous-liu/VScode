\documentclass{article}
\usepackage{graphicx}
\usepackage{amsmath}
\usepackage{array}
\usepackage[font=small, labelfont={sf,bf}, margin=1cm]{caption}
\usepackage{tabularx}
\usepackage{amssymb}



\date{Edit: \today, Class: Aug 28}
\title{MATH 461 Lecture 2 Note}
\author{James Liu}

\begin{document}
\maketitle
\section*{Combination}
\subsection*{notation}
\(\displaystyle\begin{pmatrix}
    n\\r
\end{pmatrix}=\frac{1}{r!}\times \frac{n!}{(n-r)!}\)
   As \(\displaystyle\frac{n!}{(n-r)!}\) shows howmany ways of selecting \(r\) items from \(n\) items considering sequence. Divied by \(r!\) gives the total ways not considering order.
\subsection*{Examples}
\begin{itemize}
    \item [1.]From a group of 5 women, 7 men, how many different committes of 5 people (2 women, 3 men) can be formed?\\
    \\
    \(\begin{pmatrix}
        5\\2
    \end{pmatrix}\cdot\begin{pmatrix}
        7\\3
    \end{pmatrix}\)\\ \\
    \\
    What if 2 men do not want to stay together?\\ \\ \\


\begin{itemize}
    \item [a.]\[\begin{pmatrix}
        5\\2
    \end{pmatrix}\cdot\left( \begin{pmatrix}
        5\\3
    \end{pmatrix}+\begin{pmatrix}
        2\\1
    \end{pmatrix}\cdot \begin{pmatrix}
        5\\2
    \end{pmatrix}\right)
\]
    \item[b.]\[\begin{pmatrix}
        5\\2
    \end{pmatrix}\cdot\left(\begin{pmatrix}
        7\\3
    \end{pmatrix}-\begin{pmatrix}
        2\\2
    \end{pmatrix}\cdot \begin{pmatrix}
        5\\1
    \end{pmatrix}\right)\]
    
\end{itemize}

a. is correct because of is choose the approch to add the special case , as \(\begin{pmatrix}5\\3\end{pmatrix}\)
is the combination number without the 2 special people. and the rest accounted for having only one of the 2 special ones chosed.
\\ b. is as it removes the unsatisfied ones by removing the amount similar with chossing only one from the normal men group.
    \item [2.] If there are \(n\) antenna with \(m\) of them defacted, which is \(m\leq n-m+1\), how many ways to arrange it that no 
    2 defacted antenna are sitting next to each other?
    \\ \\ \\
    \(\begin{pmatrix}
        n-m+1\\
        m
    \end{pmatrix}\)
    as there are \(n-m\) working antenna, there are \(n-m+1\) places we can put defacting ones in between. and wee need to put in total m 
    defacting ones in these positions.

\end{itemize}
\subsubsection*{Useful Identity}

\[
\begin{pmatrix}
    n\\r
\end{pmatrix}=
\begin{pmatrix}
    n-1\\r-1
\end{pmatrix}+
\begin{pmatrix}
    n-1\\r
\end{pmatrix}
\]
\[(x+y)^n=\sum_{k=0}^n \binom{n}{k}x^ky^{n-k}\]
\subsubsection*{Intuitive Prof}
\begin{align*}
    (x_1+y_1)(x_2+y_2)(x_3+y_3)\times\cdots\times(x_n+y_n)\\
    =\underbrace{\overbrace{x_1x_2x_3\cdots x_n}^{n \text{ terms}}+y_1x_2x_3x_4\cdots x_n+\cdots+y_1y_2y_3\cdots y_n}_{2^n \text{terms}}
\end{align*}
The terms that have \(k\) numbers of \(x_i\) is \(\binom{n}{k}\).
\subsection*{Multinomial Coefficient}
\[(x_1+x_2+x_3+x_4)^n=\sum\binom{n}{p_1,p_2,p_3,p_4}x_1^{p_1}x_2^{p_2}x_3^{p_3}x_4^{p_4}\]

where \(\displaystyle\binom{n}{p_1,p_2,p_3,p_4} \)means \(\displaystyle\binom{n}{p_1}\binom{n-p_1}{p_2}\binom{n-p_1-p_2}{p_1}\binom{n-p_1-p_2-p_3}{p_4}=\frac{n!}{p_1!p_2!p_3!p_4!}\)
\newpage
\subsection*{Examples}
suppose there are n balls to be put into r boxes, and no boxes are empty, how many ways?\\
consider there are \(n-1\) possible divider place that gives no empty boxes:
\[\Box _\land \Box _\land\Box _\land\Box _\land\Box _\land\Box _\land\Box _\land\Box \]
and we need to put \(r-1\) dividers inside to divide the 8 balls into \(r\) groups.
Therefore it is straight forward that th answer is \(\binom{n-1}{r-1}\)
\\
\\
what if allows empty?
\\
\\
we add extra \(r\) balls to be the one ball that fills every empty boxes, thus, the answer goes \(\binom{n+r-1}{r-1}\)



\end{document}
