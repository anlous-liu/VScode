\documentclass{article}
\usepackage{graphicx}
\usepackage{amsmath}
\usepackage{array}
\usepackage[font=small, labelfont={sf,bf}, margin=1cm]{caption}
\usepackage{tabularx}
\usepackage{amssymb}



\date{Edit: \today, Class: Aug 26}
\title{MATH 461 Lecture 1 Note}
\author{James Liu}

\begin{document}
\maketitle

\subsection*{multiplication rule}
suppose 2 experiments are to be performed, if experiment 1 can result in \(m\) possible out comes,
and for each outcomes in experiment 1, experiment 2 can result in \(n\) outcomes. Together, there will be \(m\times n \) total possible
out comes for this 2 experiment. Similarly, it can be easily generalized into \(n\) experiments: \(\prod_1^n m_k \).
\subsubsection*{Examples}
1. Suppose we need a 6 digit series number with first 3 digit being numbers and last 3 being letters. then how many series number we can make in total?
\\ \\
\(10^3\times 26^3\)
\\ \\
2. A mother want to give her son a total of 14 cards in a 7 day period, suppose she only gives her son cards once a day, how many ways
that she can choose to give her son all 14 cards?
\\ \\
\(7^{14}\) see each card as a seperate experiment then each card can be given in one of the seven days
\\ \\
\newpage
\subsection*{Permutaion}
suppose we have \(n\) distinct objects we can put them in a ordinal arrangement, each of their arrangements is called a permutaion,
if there are \(n\) objects, then there will be \(n!\) numbers of permutaions.
\subsubsection*{Examples}
1. John have 12 different books to be sorted on a shelf. 3 of them are math books, 3 of them are cs books, 4 of them are physics books and 2 of them are fictions.
how many different ways he have to sort all these books on shelf if he want books with similar topic sit together?
\\ \\
\begin{align*}
    4!&\cdot(3!\cdot3!\cdot4!\cdot2!)\\
    \text{permutaion of subjects }&\text{ permutaion of books}
\end{align*}
2. In how many ways can 6 people were to sit in a row and a,b needs to sit together?
\\ \\ 
\(2! \times 5!\)
\\ \\ 
3. In howmany ways can 4 male and 4 female be sitting in a role with no 2 people with similar gender sitting next to each other?
\\ \\ 
\(2!\cdot 4!\times 4!\)
\\ \\
4. Consider the word PEPPER, how many new words can such letter make including this one?
\\ \\
\(\frac{6!}{3!2!1!}\)
as p and p are similar and overcountings must be illiminated.












\end{document}
