\documentclass{article}
\usepackage{graphicx}
\usepackage{amsmath}
\usepackage{array}
\usepackage[font=small, labelfont={sf,bf}, margin=1cm]{caption}
\usepackage{tabularx}
\usepackage{amssymb}
\title{\textbf{Homework \#4 }}
\author{James Liu}
\date{\ }


\begin{document}
\maketitle

\section*{Problem 1}
    \subsection*{a)}

    \subsection*{b)}
        Base step:\newline 
        Because that $P(1)$ is true, $P(2\times 1)$ and $P(2\times 1+1)$ are also true\\
        Induction step:\newline
        $\forall n\in \mathbb{N}$ and $n\geqslant 4$ assume that all 
        for all $k\in \mathbb{N}$ and $ k \leqslant n$, $P(k)$ is true. \(\forall n\in \mathbb{N}\), \(n\) is either even or odd.
        \\
        1. Suppse $n+1$ is odd, then there is $n+1-1=n$ is even. Because that \(n\in \mathbb{N}\), 
        there is \(\frac{n}{2} < n\). As $\forall k\in \mathbb{N}$ and $ k \leqslant n$, $P(k)$ is true, there is $P(2\times \frac{n}{2}+1) = P(n+1)$ is true.\\
        2. Suppose $n+1$ is even, and there is $n\geqslant 4$, then there is $\displaystyle{\frac {n+1}{2}\in \mathbb{N}}$
        \\
        $\forall n\in \mathbb{N}\text{, and } n\geqslant4$ for $n=4$ there is $\displaystyle{\frac{4+1}{2}<4}$\\
        suppose that $\displaystyle{\frac{n+1}{2}<n}$ is true, then for $n+1$, there is 
        \begin{align*}
            \frac{n+1+1}{2}&<n+1\\
            \frac{n+1}{2}+\frac{1}{2}&<n+1\\
            \frac{n+1}{2}&<n+\frac{1}{2}
        \end{align*}
        Therefore, by induction, there is $\forall n\in \mathbb{N}\text{, and } n\geqslant4$,  $\displaystyle{\frac{n+1}{2}<n}$\\
        Therefore, as $\displaystyle{\frac{n+1}{2}<n}$ is true and $\forall k\in \mathbb{N}$ and $ k \leqslant n$, $P(k)$ is true,
        there is $P(\frac{n+1}{2})$ is ture, then $P(2\times\frac{n+1}{2})=P(n+1)$ is true.
        Therefore, by induction, \(P(n)\) is true for every $n\in \mathbb{N}$
\section*{Problem 2}
    \(\forall n\in \mathbb{N}\text{, and }n\geqslant 6, \), there is \(2^n\geqslant(n+1)^2\).
    Base step:\\
    For\(n=6\), there is \(2^6=64>49=(6+1)^2\) \\
    Indcution step:\\
    Assume that \(2^n\geqslant(n+1)^2\), for \(n+1\) there is:\\
    \begin{align*}
        \text{as  } 2^n\geqslant(n+1)^2,\  2^n+2^n\geqslant 2(n+1)^2&\\
        2(n+1)^2=2n^2+4n+2, \text{ and } 2n^2+4n+2-(n^2+4n+4)=n^2-2\\
        \text{as }n\geqslant 6,\ n^2-2>0\\
        \text {Therefore }  2^n+2^n=2^{n+1}\geqslant2(n+1)^2>(n+2)^2
    \end{align*}
    
    \begin{align*}
        \text {For}\ n=1, \ 2^1=2&\ngeq 4=(1+1)^2\\
        \text {For}\ n=2, \ 2^2=4&\ngeq 9=(1+1)^2\\
        \text {For}\ n=3, \ 2^3=8&\ngeq 16=(1+1)^2\\
        \text {For}\ n=4, \ 2^4=16&\ngeq 25=(1+1)^2\\
        \text {For}\ n=5, \ 2^5=32&\ngeq 36=(1+1)^2\\
    \end{align*}
    
    Therefore, by induction, every integer greater than or equal to 6 satisfies \(2^n\geqslant(n+1)^2\)
\section*{Problem 3}
    \subsection*{a)}
        Base step:\\
        As said, \(a_1\) and \(a_2\) are odd, and \(a_3=2a_{2}+3a_n{1}\), \(2\times a_{2}\) will be even and \(3 \times a_1\) will be odd as \(a_1\) is odd.
        Therfore, a odd number plus a even number is odd and thus \(a_3\) is odd.
        Induction step:\\
        Assume that \(a_k,\ \forall n,k\in \mathbb{N}, \text{ and } n\geqslant 3, k\leqslant n\) is odd. For \(a_{n+1}\),
        there is that \(a_{n+1}=2a_n+3a_{n-1}\), \(2\) is even and thus \(2\times a_{n}\) is even. As
        \(a_{n-1}\) is also odd as assumed, \(3\times a_{n-1}\) is odd. Thus, \(a_{n+1}\) equals to the sum
        of a even and a odd number. Therefore, \(a_{n+1}\) is also odd.\\
        Thus, by induction, if \(a_1,a_2\) are odd,\(\forall b\in \mathbb{N}\), \(a_n\) is also odd.
    \newpage
    \subsection*{b)}
        Base step:\\
        \(a_1=1=\frac{1}{2} (3^0-(-1))\) and \(a_2=\frac{1}{2} (3^1-(-1)^2)\)\\
        Induction step:\\
        Assume that \(a_k=\frac{1}{2}(3^{n-1}-(-1)^n)\), for \(a_{n+1}\), there is:\\
            \(a_{n+1}=2a_n+3a_{n-1}\) and \(a_{n+1}=\frac{1}{2}(3^{n}-(-1)^{n+1})\)

\end{document}
