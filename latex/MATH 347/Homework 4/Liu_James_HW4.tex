\documentclass{article}
\usepackage{graphicx}
\usepackage{amsmath}
\usepackage{array}
\usepackage[font=small, labelfont={sf,bf}, margin=1cm]{caption}
\usepackage{tabularx}
\usepackage{amssymb}
\title{\textbf{Homework \#4 }}
\author{James Liu}
\date{\ }


\begin{document}
\maketitle

\section*{Problem 1}
    \subsection*{a)}
        1. For \(n\geqslant0\):\\
        Base step:\\
        \(P(0)\) is true.\\
        Induction step:\\
        \(P(n)\) do imply \(P(n+1)\).\\
        Thus,by induction, \(\forall n\in \mathbb{Z}, \ n\geqslant 0\) the statement \(P(n)\)is true.\\
        2. for \(n<0\)\\
        Suppose statement \(Q(m)\) means that:the statement\(P(-m)\) is true.
        Then the statement: the statement\(P(m)\) is true. is equivelent to \(Q(-m)\).\\
        Base step:\\
        \(P(0)\) is true, then \(Q(-0)\) is ture. Then, as implied, \(P(0-1)=P(-1)\) is true and thus \(Q(1)\) is true.\\
        Induction step:\\
        Assume that \(Q(m),\ m\in\mathbb{N},\ m\geqslant 1\) is true, equivelently, \(P(-m)\) is true. Then, \(Q(m+1)\) is true if \(P(-m-1)\) is true.
        Because, when \(P(n)\) is true, it implies \(P(n-1)\) is also true, when \(n=-m\), it means that when \(p(-m)\) is ture, \\
        \(P(-m-1)\) is also true. Which means that \(Q(m)\) implies \(Q(m+1)\).\\
        Thus, by induction, \(\forall m\in \mathbb{N}, m\geqslant 1\), \(Q(m)\) is ture.\\
        Meaning that \(\forall m\in \mathbb{N}, m\geqslant 1\), \(P(-m)\) is true.\\
        Meaning that \(\forall n\in \mathbb{Z}, n\leqslant -1\), \(P(n)\) is true.\\
        \\
        Thus, because \(n\in\mathbb{Z}\) and \(P(n)\) is ture for \(n<0\), and \(n\geqslant 0\), \(P(n)\) is true for all integers \(n\).
    \subsection*{b)}
        Base step:\newline 
        Because that $P(1)$ is true, $P(2\times 1)$ and $P(2\times 1+1)$ are also true\\
        Induction step:\newline
        $\forall n\in \mathbb{N}$ and $n\geqslant 4$ assume that all 
        for all $k\in \mathbb{N}$ and $ k \leqslant n$, $P(k)$ is true. \(\forall n\in \mathbb{N}\), \(n\) is either even or odd.
        \\
        1. Suppse $n+1$ is odd, then there is $n+1-1=n$ is even. Because that \(n\in \mathbb{N}\), 
        there is \(\frac{n}{2} < n\). As $\forall k\in \mathbb{N}$ and $ k \leqslant n$, $P(k)$ is true, there is $P(2\times \frac{n}{2}+1) = P(n+1)$ is true.\\
        2. Suppose $n+1$ is even, and there is $n\geqslant 4$, then there is $\displaystyle{\frac {n+1}{2}\in \mathbb{N}}$
        \\
        $\forall n\in \mathbb{N}\text{, and } n\geqslant4$ for $n=4$ there is $\displaystyle{\frac{4+1}{2}<4}$\\
        suppose that $\displaystyle{\frac{n+1}{2}<n}$ is true, then for $n+1$, there is 
        \begin{align*}
            \frac{n+1+1}{2}&<n+1\\
            \frac{n+1}{2}+\frac{1}{2}&<n+1\\
            \frac{n+1}{2}&<n+\frac{1}{2}
        \end{align*}
        Therefore, by induction, there is $\forall n\in \mathbb{N}\text{, and } n\geqslant4$,  $\displaystyle{\frac{n+1}{2}<n}$\\
        Therefore, as $\displaystyle{\frac{n+1}{2}<n}$ is true and $\forall k\in \mathbb{N}$ and $ k \leqslant n$, $P(k)$ is true,
        there is $P(\frac{n+1}{2})$ is ture, then $P(2\times\frac{n+1}{2})=P(n+1)$ is true.
        Therefore, by induction, \(P(n)\) is true for every $n\in \mathbb{N}$
\section*{Problem 2}
    \(\forall n\in \mathbb{N}\text{, and }n\geqslant 6, \), there is \(2^n\geqslant(n+1)^2\).
    Base step:\\
    For\(n=6\), there is \(2^6=64>49=(6+1)^2\) \\
    Indcution step:\\
    Assume that \(2^n\geqslant(n+1)^2\), for \(n+1\) there is:\\
    \begin{align*}
        \text{as  } 2^n\geqslant(n+1)^2,\  2^n+2^n\geqslant 2(n+1)^2&\\
        2(n+1)^2=2n^2+4n+2, \text{ and } 2n^2+4n+2-(n^2+4n+4)=n^2-2\\
        \text{as }n\geqslant 6,\ n^2-2>0\\
        \text {Therefore }  2^n+2^n=2^{n+1}\geqslant2(n+1)^2>(n+2)^2
    \end{align*}
    
    \begin{align*}
        \text {For}\ n=1, \ 2^1=2&\ngeq 4=(1+1)^2\\
        \text {For}\ n=2, \ 2^2=4&\ngeq 9=(1+1)^2\\
        \text {For}\ n=3, \ 2^3=8&\ngeq 16=(1+1)^2\\
        \text {For}\ n=4, \ 2^4=16&\ngeq 25=(1+1)^2\\
        \text {For}\ n=5, \ 2^5=32&\ngeq 36=(1+1)^2\\
    \end{align*}
    
    Therefore, by induction, every integer greater than or equal to 6 satisfies \(2^n\geqslant(n+1)^2\)
\section*{Problem 3}
    \subsection*{a)}
        Base step:\\
        As said, \(a_1\) and \(a_2\) are odd, and \(a_3=2a_{2}+3a_n{1}\), \(2\times a_{2}\) will be even and \(3 \times a_1\) will be odd as \(a_1\) is odd.
        Therfore, a odd number plus a even number is odd and thus \(a_3\) is odd.\\
        Induction step:\\
        Assume that \(a_k,\ \forall n,k\in \mathbb{N}, \text{ and } n\geqslant 3, k\leqslant n\) is odd. For \(a_{n+1}\),
        there is that \(a_{n+1}=2a_n+3a_{n-1}\), \(2\) is even and thus \(2\times a_{n}\) is even. As
        \(a_{n-1}\) is also odd as assumed, \(3\times a_{n-1}\) is odd. Thus, \(a_{n+1}\) equals to the sum
        of a even and a odd number. Therefore, \(a_{n+1}\) is also odd.\\
        Thus, by induction, if \(a_1,a_2\) are odd,\(\forall b\in \mathbb{N}\), \(a_n\) is also odd.
    \subsection*{b)}
        Base step:\\
        \(a_1=1=\frac{1}{2} (3^0-(-1))\) and \(a_2=\frac{1}{2} (3^1-(-1)^2)\)\\
        Induction step:\\
        Assume that \(a_n=\frac{1}{2}(3^{n-1}-(-1)^n)\), and \(a_{n-1}=\frac{1}{2}(3^{n-2}-(-1)^{n-1})\).\\
        Therefore, for \(a_{n+1}\), there is:\\
        \begin{center}
            \(a_{n+1}=2a_n+3a_{n-1}\) and \(a_{n+1}=\frac{1}{2}(3^{n}-(-1)^{n+1})\)
        \end{center}
        \begin{align*}
            2a_n+3a_{n+1}&=2\times\frac{1}{2}(3^{n-1}-(-1)^n)+3\times \frac{1}{2}(3^{n-2}-(-1)^{n-1})\\
            &=3^{n-1}-(-1)^n+\frac{3}{2}(3^{n-2}-(-1)^{n-1})\\
            &=\frac{2\cdot 3^{n-1}+3\cdot3^{n-2}}{2}-\left((-1)^{n}+\frac{3}{2}(-1)^{n-1}\right)\\
            &=\frac{1}{2}(2+1)3^{n-1}-((-1)^{n-1}(-1+\frac{3}{2}))\\
            &=\frac{1}{2}(3^n-(-1)^{n-1})\\
            \\
            \text{Because }(-1)^{n+1}=(-1)^2(-1)^{n-1}&=(-1)^{n-1}, \\
            \text{original function}&=\frac{1}{2}(3^n-(-1)^{n+1})=a_{n+1}
        \end{align*}
        Thus, by induction, \(a_n=\frac{1}{2}(3^{n-1}-(-1)^n)\) for all \(n\in \mathbb{N}\)
\newpage
\section*{Problem 4} 
    \subsection*{a)}
        Base step:\\
        \(F_1=1\), \(F_2=1, F_3=2\), \(2|F_3\). Also, for \(k=0\), there is \(0\div 2 =0\in \mathbb{Z}\), therefore, \(2|F_{3\times 0}\)\\
        Induction step:\\
        \(k\in \mathbb{N}\) Assume that \(2|F_{3k}\), therefore, for \(F_{3(k+1)}=F_{3k+3}\) there is:
        \begin{align*}
            F_{3k+3}&=F_{3k+2}+F_{3k+1}\\
                    &=F_{3k+1}+F_{3k}+F_{3k}+F_{3k-1}\\
                    &=F_{3k}+F_{3k-1}+F_{3k}+F_{3k}+F_{3k-1}\\
                    &=3\cdot F_{3k}+2\cdot F_{3k-1}\\
        \end{align*}
        As \(2|F_{3k}\), \(\displaystyle{\frac{F_{3k}}{2}\in \mathbb{Z}}\), therefore, \(3F_{3k} \div 2=3\times \frac{F_{3k}}{2}\in \mathbb{Z}\).
        Therefore, \(2|(3 F_{3k})\). Because \(2\cdot F_{3k-1} \div 2 = F_{3k-1}\in \mathbb{Z}\), \(2|(2 F_{3k-1})\).
        Thus, \(2|F_{3(k+1)}\).\\Therefore, by induction, there is \(2|F_{3k}\) for every \(k\geqslant 0\)
    \subsection*{b)}
    Base step:\\
        \(F_1=1\), \(F_2=1,\  F_3=2,\ F_4=3\), \(3|F_4\). Also, for \(k=0\), there is \(0\div 3 =0\in \mathbb{Z}\), therefore, \(3|F_{4\times 0}\)\\
        Induction step:\\
        \(k\in \mathbb{N}\) Assume that \(3|F_{4k}\), therefore, for \(F_{4(k+1)}=F_{4k+4}\) there is:\\
        \begin{align*}
            F_{4k+4}&=F_{4k+3}+F_{4k+2}\\
                    &=F_{4k+2}+F_{4k+1}+F_{4k+1}+F_{4k}\\
                    &=F_{4k+1}+F_{4k}+F_{4k+1}+F_{4k+1}+F_{4k}\\
                    &=2\times F_{4k}+3\times F_{4k+1}\\
        \end{align*}
        As \(3|F_{4k}\), \(\displaystyle{\frac{F_{4k}}{3}\in \mathbb{Z}}\), therefore, \(2F_{4k} \div 3=2\times \frac{F_{4k}}{3}\in \mathbb{Z}\).
        Therefore, \(3|(2 F_{4k})\). Because \(3\cdot F_{4k+1} \div 3 = F_{4k+1}\in \mathbb{Z}\), \(3|(3 F_{4k+1})\).
        Thus, \(3|F_{3(k+1)}\).\\Therefore, by induction, there is \(3|F_{4k}\) for every \(k\geqslant 0\)
\newpage
\section*{Problem 5}
        Base step:\\
        \(P(24)\) is true as \(2\times 5+2\times 7=24\)\\
        \(P(25)\) is true as \(5\times 5=25\)\\
        \(P(26)\) is true as \(5+3\times 7=26\)\\
        \(P(27)\) is true as \(4\times 5 + 7 = 27\)\\
        \(P(28)\) is true as \(4\times 7=28\)\\
        \(P(29)\) is true as \(3\times 5 + 2\times 7 = 29\)\\
        Induction step:\\
        \(\forall k,n\in \mathbb{N}\) and \(24\leqslant k\leqslant n ,\ n\geqslant 29\) 
        Assume that \(P(k)\) are all true, and thus \(P(n)\) is also true, then, \(P(n+1-5)\) is true
        as \(n-4<n \text{ for }n\in \mathbb{N}\). That means that \(\exists a,b\in \mathbb{N}\) that satisfies \(a\cdot 5 + b\cdot 7 = n-4\).
        Therefore, \((a+1)\cdot 5+b\cdot 7=n+1\). Thus, it is possible to make \(n\) cents postage with 5-cent and 7-cent stamps for all \(n\geqslant 24\).
        \\  \\
        If \(P(23)\) is true, it means that \(\exists a,b\in \mathbb{N}\) that makes \(a\cdot 5+b\cdot 7=23\):

        \begin{align*}
            a&=\frac {23-7b}{5}\\
        \end{align*}
        If \(a\) is an integer, then \(\exists b\in \mathbb{N},\ \text{making } (23-7b)\geqslant0,\ \text{and }5|23-7b\)
        \begin{align*}
            \text{for }b=1&, \ 5\nmid 16=23-7\\
            \text{for }b=2&, \ 5\nmid 9=23-14\\
            \text{for }b=3&, \ 5\nmid 2=23-21\\
            \text{for }b=4&, \ 23-28=-5<0\\
        \end{align*}

        Therefore, \(P(23)\) is false.
\newpage
\section*{Problem 6}
        First proof that for every shape that is a quare of \(2^{k-1}\) side length subtracted from \(2^k\) side length square from a corner. The discribed shape is similar to figure \ref{first}.

        \begin{figure}[h]
            \centering
            \includegraphics[width=0.25\textwidth]{figures/demonstration of first idea.png}
            \caption{the blue region is the intended region the red region is the removed square}
            \label{first}
        \end{figure}
        The statement is: \(\forall n\in \mathbb{N},\ n\geqslant 1\), the shape mentioned above with side length (the side length of a complete square, similar when mentioned below)\(2^n\)
        can be tiled use L-shapes.\\
        Base step:\\
        For \(n=1\), there is a following way shown below in figure \ref{n=1}:\\
        \begin{figure}[h]
            \centering
            \includegraphics[width=0.25\textwidth]{figures/for n=1.png}
            \caption{A way to tile when \(n=1\)}
            \label{n=1}
        \end{figure}\\
        
        Induction step:
        Assume that for a shape with side length of \(2^n\) can still be tiled with L-shape. Then, for shape with side length of \(2^{n+1}\), there is the following way to tile it with 
        shapes of length \(2^n\) in figure \ref{n=n}.\\ 
        
        \begin{figure}[h]
            \centering
            \includegraphics[width=0.25\textwidth]{figures/for induction.png}
            \caption{A way to tile while each \(2\times2\) square have a side length of \(2^n\)}
            \label{n=n}
        \end{figure}\newpage
        As the \(2^n\) shape is tileable with L-shapes, the \(2^{n+1}\) is also tileable. Therfore, by induction, \(\forall n\in \mathbb{N},\ n\geqslant 1\),
        the shape mentioned above with side length \(2^n\) can be tiled use L-shapes.
        \\ \\
        Secondly, the statemnet need to proof is: \(\forall m\in\mathbb{N},\ m\geqslant 1\), shape that is a \(1\times 1\) square subtracted from a corner of a square with side length of \(2^m\) can be tiled with L-shapes.\\
        Base step:\\
        For \(m=1\), there is a tiling way similar to figure \ref{m=1}.
        \begin{figure}[h]
            \centering
            \includegraphics[width=0.25\textwidth]{figures/for n=1.png}
            \caption{A way to tile when \(m=1\)}
            \label{m=1}
        \end{figure}\\
        Induction step:\\
        Assume tha for a square with sidelength \(2^n\) can be tiled with then for a square having side length of \(2^{n+1}\), a corner of a \(2^n\times 2^n\) can be tiled,
        lefting a shape identical to the shape discribed in the first part which is profed to be tileable. 
        \begin{figure}[h]
            \centering
            \includegraphics[width=0.25\textwidth]{figures/for second prof.png}
            \caption{Blue part profed to be tileable in the first part and red part is assumed to be tileable}
            \label{p2}
        \end{figure}\\
        Therefore, as it is like in figure \ref{p2}, the if the red region is 
        tileable, the shape is tileable.\\
        Thus, by induction, \(\forall m\in\mathbb{N},\ m\geqslant 1\), shape that is a \(1\times 1\) square subtracted from a corner of a square with side length of \(2^m\) can be tiled with L-shapes.
\end{document}
