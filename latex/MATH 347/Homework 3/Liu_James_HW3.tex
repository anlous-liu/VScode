\documentclass{article}
\usepackage{graphicx}
\usepackage{amsmath}
\usepackage{array}
\usepackage[font=small, labelfont={sf,bf}, margin=1cm]{caption}
\usepackage{tabularx}
\usepackage{amssymb}
\title{\textbf{Homework \#3 }}
\author{James Liu}
\date{\ }


\begin{document}
\maketitle

\section*{Problem 1}
    The negation of statement:"\\
    \(\forall x,y\in \mathbb{R}\) and \(x>0,\ y>0\), there is \(\dfrac{2}{1/x+1/y} \leqslant \sqrt{xy}\)"\\
    is:"\(\forall x,y\in \mathbb{R}\) and \(x>0,\ y>0\), there is \(\dfrac{2}{1/x+1/y} >\sqrt{xy}\)"
    \\
    Let \(x,y\in \mathbb{R}\) and \(x>0,\ y>0\), suppose\(\dfrac {2}{1/x+1/y}> \sqrt{xy}\). Then:
    \begin{align*}
        \dfrac {2}{1/x+1/y}&> \sqrt{xy}\\
        \dfrac{2xy}{x+y}&>\sqrt{xy}\\
        \left( \dfrac{2xy}{x+y}\right)^2 &>xy\\
        \dfrac{4x^2y^2}{x^2+2xy+y^2}&>xy\\
        4x^2y^2&>x^3y+xy^3+2x^2y^2\\
        0&>x^3y+xy^3-2x^2y^2\\
        0&>xy(x^2+y^2-2xy)\\
        0&>xy(x-y)^2
    \end{align*}
    However, \(x>0\), and \(y>0\), \(xy>0\) and \((x-y)^2\geqslant 0\),\\
    therefore, \(xy(x-y)^2\geqslant 0\). Thus, the negation is false and the orginal statement is true.
    \\
    \\
    Let \(x=y\), then\(\dfrac{2}{1/x+1/y} = \dfrac{2}{1/x+1/x}=x\),\\
    and \(\sqrt{xy}=\sqrt{x^2}\) due to \(x>0\), \(\sqrt{x^2}=x=\dfrac{2}{1/x+1/y}\)
    \\
    \\
    Let \(\dfrac{2}{1/x+1/y} = \sqrt{xy}\):
    \begin{align*}
        %
        \dfrac {2}{1/x+1/y}&= \sqrt{xy}\\
        \dfrac{2xy}{x+y}&=\sqrt{xy}\\
        \left( \dfrac{2xy}{x+y}\right)^2 &=xy\\
        \dfrac{4x^2y^2}{x^2+2xy+y^2}&=xy\\
        4x^2y^2&=x^3y+xy^3+2x^2y^2\\
        0&=x^3y+xy^3-2x^2y^2\\
        0&=xy(x^2+y^2-2xy)\\
        0&=xy(x-y)^2
        %
    \end{align*}
    Because \(x>0\text{, and }y>0\), \(xy>0\), and \((x-y)^2\geqslant 0\). Therefore, for \(xy(x-y)^2\) to be \(0\)
    it must be \((x-y)^2=0\). Thus, \(x-y=0\) and \(x=y\)\\
    \\
    Therefore, the harmonic mean of x, y is always less than or equal to the geometric
    mean of x and y, and that equality occurs if and only if x = y.
\newpage
\section*{Problem 2}
    Base step:

    When \(n\in \mathbb{N},\ n=1\), \(\displaystyle{\sum_{i = 1}^{1} 1^3}= \left(\dfrac{1(1+1)}{2}\right)^2=1\)\\
    Induction step:

    Assume that, for \(n\in \mathbb{N},\ \displaystyle{\sum_{i=1}^n i^3}=\left(\dfrac{n(n+1)}{2}\right)^2\) is true.

    \(\displaystyle{\sum^{n+1}_{i=1}i^3}=\displaystyle{\sum_{i=1}^n i^3}+(n+1)^3\) and \(\left(\dfrac{(n+2)(n+1)}{2}\right)^2=\left(\dfrac {n(n+1)}{2}+\dfrac{2(n+1)}{2}\right)^2\)
    
    Suppose \(\displaystyle{\sum_{i=1}^n i^3}+(n+1)^3\neq \left(\dfrac {n(n+1)}{2}+\dfrac{2(n+1)}{2}\right)^2\), then:\\
    \begin{align*}
        %
        \sum_{i=1}^n i^3+(n+1)^3&\neq \left(\dfrac {n(n+1)}{2}+(n+1)\right)^2\\
        \sum_{i=1}^n i^3+(n+1)^3&\neq \left(\dfrac {n(n+1)}{2}\right)^2 +n(n+1)^2+(n+1)^2\\
        \text{As } \displaystyle{\sum_{i=1}^n i^3}=\left(\dfrac{n(n+1)}{2}\right)^2\\
        (n+1)^3&\neq n(n+1)^2+(n+1)^2\\
        (n+1)^3&\neq (n+1)(n+1)^2\\
        (n+1)^3&\neq (n+1)^3
        %
    \end{align*}
    Therefore, \(\displaystyle{\sum_{i=1}^{n+1} i^3} = \left(\dfrac {(n+2)(n+1)}{2}\right)^2\) is true if \(\displaystyle{\sum_{i=1}^n i^3}=\left(\dfrac{n(n+1)}{2}\right)^2\) is true.\\
    Thus, by induction, for \(n\in \mathbb{N},\ \displaystyle{\sum_{i=1}^n i^3}=\left(\dfrac{n(n+1)}{2}\right)^2\) is true.
\newpage
\section*{Problem 3}
    Base step:

    When \(n=2\), \(\displaystyle{\prod ^2_{i=2}\left(1-\dfrac{1}{i^2}\right)=\frac3 4=\dfrac {2+1}{2\times 2}}\)\\
    Induction step:

    Assuming that, for \(n\in \mathbb{N},\ n\geqslant2,\ \displaystyle {\prod ^n_{i=2}\left(1-\dfrac{1}{i^2}\right)=\frac{n+1}{2n}}\).
    
    Then, for \(n+1\) suppose they are not equal, there will be:
    \begin{align*}
        %
        \prod ^{n+1}_{i=2}\left(1-\dfrac{1}{i^2}\right)&\neq \frac{(n+1)+1}{2(n+1)}\\
        \prod ^{n}_{i=2}\left(1-\dfrac{1}{i^2}\right)\times \left(1-\frac{1}{(n+1)^2}\right)&\neq \frac{(n+1)+1}{2(n+1)}\\
        \text{As  \( \prod ^n_{i=2}\left(1-\dfrac{1}{i^2}\right)=\frac{n+1}{2n}\)}\\
        \frac{n+1}{2n} \times \left(1-\frac{1}{(n+1)^2}\right)&\neq \frac{(n+1)+1}{2(n+1)}\\
        \frac{n+1}{2n}\times \frac{(n+1)^2-1}{(n+1)^2}&\neq \frac{(n+1)+1}{2(n+1)}\\
        \frac{n+1}{2n}\times \frac{n(n+2)}{(n+1)^2}&\neq \frac{(n+1)+1}{2(n+1)}\\
        \frac{n+2}{2(n+1)} &\neq \frac{n+2}{2(n+1)}
        %
    \end{align*}    
    Therefore, the negation is false, and, therefore, \(\displaystyle{\prod ^{n+1}_{i=2}\left(1-\dfrac{1}{i^2}\right) = \frac{(n+1)+1}{2(n+1)}}\) is true.
    Therefore, by induction, \(\forall n\in \mathbb{N},\ n\geqslant2,\ \displaystyle {\prod ^n_{i=2}\left(1-\dfrac{1}{i^2}\right)=\frac{n+1}{2n}}\)
\newpage
\section*{Problem 4}
    The solution is \(\displaystyle{a_n =\sum ^{n}_{i=1} 3^{i-1}}\)\\
    Base step:

    When \(n=1\), \(a_{n+1}=a_2=3\times 1+1=4\), There is also, \(\displaystyle{\sum_{i=1}^{n} 3^{i-1}=3^0+3^1=1+3=4}\).
    
    Similarly, when \(n=2\), there is:

    \(a_{2+1}=3\times a_2+1=13 = 1+3+9= 3^0+3^1+3^2=\displaystyle {\sum ^3_{i=1}3^{i-1}}\)
    \\
    Induction step:

    Assume \(\displaystyle{\sum^n_{i=1}3^{i-1}=a_n}\), then for \(a_{n+1}\), if the patter does not work, there will be:


    \begin{align*}
        %
        \sum^{n+1}_{i=1} 3^{i-1}\neq3\cdot a_n +1\\
        \sum^{n+1}_{i=1} 3^{i-1}\neq3\sum^{n}_{i=1} 3^{i-1}+1\\
        \sum^{n+1}_{i=1} 3^{i-1}\neq \sum^{n+1}_{i=2} 3^{i-1}+1\\
        \text{When } i =1\text{,} \   3^{i-1}=3^0=1\\
        \therefore \sum^{n+1}_{i=1} 3^{i-1}\neq \sum^{n+1}_{i=1} 3^{i-1}\\
        %
    \end{align*}
    Therefore, the negation is false and \(\displaystyle{\sum^{n+1}_{i=1} 3^{i-1}=3\cdot a_n +1}\).
    Thus, by induction, it is safe to conclude that \(\displaystyle{a_n =\sum ^{n}_{i=1} 3^{i-1}}\).




\newpage
\section*{Problem 5}
    Base step: 

    when \(k=2\), if \(f_1\) is bounded and \(f_2\) is bounded, then \(\exists M\in \mathbb{R},\ M>0\) and \(\exists N\in \mathbb{N},\ N>0\)
    that \(\forall x\in A\), \(|f_1(x)|<M\) and \(|f_2(x)|<N\). Therefore, \(|f_1|+|f_2|<M+N\). Therefore, \(f_1+f_2\) is bounded.\\
    Induction step:

    Assume that \(f_1+f_2+f_3+\cdots +f_k\) is bounded, then we get that \\\(\exists O\in \mathbb{R}, \text{ and }O>0\), such that \(|f_1+f_2+f_3+\cdots +f_k|<O\)
    Due to that \(f_{k+1}\) is also bounded, \(\exists P\in \mathbb{R}, \ \text{and } P>0\), such that \(f_{k+1}<P\). Therefore, \(|f_1+f_2+f_3+\cdots +f_k|+|f_{k+1}|<O+P\).
    Therefore, if \(f_1+f_2+f_3+\cdots +f_k\) is bounded, then \(f_1+f_2+f_3+\cdots +f_k+f_{k+1}\) is also bounded. 
    \\
    Thus, by induction, for \(k\geqslant 2\), \(f_1+f_2+f_3+\cdots +f_k\) is bounded.



    %trying to prof that \(\displaystyle{\sum^k_{i=1}M_k<\infty}\)
\end{document}
