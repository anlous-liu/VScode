\documentclass{article}
\usepackage{graphicx}
\usepackage{amsmath}
\usepackage{array}
\usepackage[font=small, labelfont={sf,bf}, margin=1cm]{caption}
\usepackage{tabularx}
\usepackage{amssymb}
\usepackage{bm}
\title{\textbf{Homework \#6 }}
\author{James Liu}
\date{\today }

\begin{document}

\maketitle

\section*{Problem 1}
\subsection*{a)}
As \((k-2)|2k\), then\(\exists m\in \mathbb{Z}\) that \(m(k-2)=2k\).
Thus, \(m=\frac{2k}{k-2}=\frac{2k-4+4}{k-2}=2+\frac{4}{k-2}\). Also, 
as \(m\in \mathbb{Z},\ k\in \mathbb{N}\), \(\frac{4}{k-2}\in \mathbb{Z}\).
Thus, \((k-2)|4\), and \(k-2=\pm 1,\ \pm 2,\ \pm 4\). Thus, \(k=-2, 0 ,1, 3,4,6\).
As \(k\in \mathbb{N}\), \(k\in\{1,3,4,6\}\)
\subsection*{b)}
there is:
\begin{align*}
    \frac {x+y}{xy}&=\frac{1}{p}\\
    px+py-xy&=0
\end{align*}
Thus: \(\displaystyle{x=\frac{py}{y-p}}\) and \(\displaystyle{y=\frac{px}{x-p}}\).
Then, \(x-y=\displaystyle{\frac{py}{y-p}-\frac{px}{x-p}=\frac{0}{(y-p)(x-p)}}\).
Thus, \(x=y\). Therefore, there is \(\displaystyle{\frac{2}{x}=\frac{1}{p}}\).
Thus there is \(p=\frac{1}{2}x\). Thus, \(\forall x,y\in \mathbb{N},\ x=y=2p\) or \((2p,2p)\) gives \(\frac{1}{x}+\frac{1}{y}=\frac{1}{p}\).
\newpage
\section*{Problem 2}
\begin{align*}
    F_1&&F_2&&F_3&&F_4&&F_5&&F_6&&F_7&&F_8&&F_9\\
    1&&1&&2&&3&&5&&8&&13&&21&&34
\end{align*}
Base step:\\
    \(F_1\equiv1(\text{mod }3)\)\\
    \(F_2\equiv1(\text{mod }3)\)\\
    \(F_3\equiv2(\text{mod }3)\)\\
    \(F_4\equiv0(\text{mod }3)\)\\
    \(3|F_4\) and \(3\nmid F_{3}\)\\
Induction step:\\
    Assume that for \(p,k\in \mathbb {N},\ 0<p\leqslant k\), \(3|F_{4p}\) and \(3\nmid F_{4p-1}\).
    Therefore, we have:\\
     \(F_{4k}\equiv0(\text{mod }3)\) and \(F_{4k-1}\not\equiv0(\text{mod }3)\).
     Then, there is \(F_{4k-1}\equiv1(\text{mod }3)\) or \(F_{4k-1}\equiv2(\text{mod }3)\)\\
1. \(F_{4k-1}\equiv 1 (\text{mod }3)\)
    Then we have:
    \begin{align*}
        F_{4k-1} &\equiv 1 (\text{mod }3)\\
        F_{4k} &\equiv 0 (\text{mod }3)\\
        F_{4k+1} &\equiv (0+1) (\text{mod }3)\\
        F_{4k+2} &\equiv (0+1) (\text{mod }3)\\
        F_{4k+3} &\equiv (1+1) (\text{mod }3)\\
        F_{4(k+1)}&\equiv 0 (\text{mod }3)\\
    \end{align*}
2. \(F_{4k-1}\equiv 2 (\text{mod }3)\)
    Then we have:
    \begin{align*}
        F_{4k-1} &\equiv 2 (\text{mod }3)\\
        F_{4k} &\equiv 0 (\text{mod }3)\\
        F_{4k+1} &\equiv (0+2) (\text{mod }3)\\
        F_{4k+2} &\equiv (0+2) (\text{mod }3)\\
        F_{4k+3} &\equiv (2+2-3=1) (\text{mod }3)\\
        F_{4(k+1)}&\equiv (1+2-3=0) (\text{mod }3)\\
    \end{align*}
Thus, by induction, \(\forall n \in \mathbb{N},\ 3|F_{4k}\) and \(3\nmid F_{4k-1}\) Therefore, if \(4|n\), \(3|F_n\).\\
Induction step:\\
Assume that for \(p,k\in \mathbb {N},\ 0<p\leqslant k\), \(3|F_{p}\) and \(3\nmid F_{p-1}\)
Then, there is \(F_{k-1}\equiv1(\text{mod }3)\) or \(F_{k-1}\equiv2(\text{mod }3)\)\\
1. \(F_{k-1}\equiv 1 (\text{mod }3)\)
    Then we have:
    \begin{align*}
        F_{k-1} &\equiv 1 (\text{mod }3)\\
        F_{k} &\equiv 0 (\text{mod }3)\\
        F_{k+1} &\equiv (0+1) (\text{mod }3)\\
        F_{k+2} &\equiv (0+1) (\text{mod }3)\\
        F_{k+3} &\equiv (1+1) (\text{mod }3)\\
        F_{k+4}&\equiv 0 (\text{mod }3)\\
    \end{align*}
2. \(F_{k-1}\equiv 2 (\text{mod }3)\)
    Then we have:
    \begin{align*}
        F_{k-1} &\equiv 2 (\text{mod }3)\\
        F_{k} &\equiv 0 (\text{mod }3)\\
        F_{k+1} &\equiv (0+2) (\text{mod }3)\\
        F_{k+2} &\equiv (0+2) (\text{mod }3)\\
        F_{k+3} &\equiv (2+2-3=1) (\text{mod }3)\\
        F_{k+4}&\equiv (1+2-3=0) (\text{mod }3)\\
    \end{align*}
Therefore, by induction, \(\forall k\in\mathbb{N}\), if \(3|F_k\), then \(3|F_{k+4}\).\\
When \(k=4\), \(3|F_k\) and \(4|k\) or \(k\equiv0(\text{mod }4) \). Also, as \(4\equiv4(\text{mod } 4)\),  \\ \(k+4\equiv0(\text{mod }4)\).
Thus, if \(3|F_k\), then \(4|k\). Thus \(3|F_n\) if and only if \(4|n\). 
\newpage
\section*{Problem 3}
\textbf{statement 1: the set \(A\) that contains all $n\in\mathbb{N}$ that has a facterization containing $a$, equals the set \(B\) containing \(m\in\mathbb{N}\) that $a|m$.}\\
proof: Assume that the facterization of $n$ contains $a$, then $\exists i\in \mathbb{N}$ that $n=ia$, \(B\subseteq A\).
Assume that $a|n$, then $\exists i\in \mathbb{N}$ that $n=ia$, thus, \(A\subseteq B\).\\
\textbf{statement 2: if \(m_k\in \mathbb{Z}\ n,p^k\in \mathbb{N},\ m_k=\displaystyle{\left\lfloor \frac{n}{p^k}\right\rfloor}\), then there are \(m_k\) elements in the set \(\{a:a\in \mathbb{N},\ 0<a\leqslant n,\ p^k|a\}\)}\\
proof: According to the quotient-remainder theorem, there is \(l\in\mathbb{Z}, r\in \mathbb{N}\) that \(n=lp^k+r,\ 0\leqslant r<n\). 
Thus, \(n-r=lp^k\geqslant0\), therefore \(lp^k\leqslant n\), \(l\geqslant 0\). \(\underbrace{p^k|1 p^k,\ p^k|2 p^k,\ p^k|3 p^k\ \dots\ p^k|l p^k }_{l \ \text{expressions}}\)
Thus, \(p^k\) divides \(l\)  natural numbers between 0 and \(n\). As \(\displaystyle{\left\lfloor \frac{n}{p^k}\right\rfloor}=l=m_k\), the statement is true.\\
\textbf{statement 3: \(\forall a, p,k\in \mathbb{N}\), if \(p^k|a\), then \(\forall l\in \mathbb{N},\ 0<l\leqslant k\) there is \(p^l|a\)}\\
proof: Base step:\\
\(p|a\) as \(p^k=p\times p^{k-1}\).\\
Induction step:\\
Assume that \(n\in\mathbb{N},\ 0<n< k\), \(p^n|a\). Then, \(p^{n+1}=p\times p^n\). Thus, \(p^{n+1}|a\).
Thus, by induction, the statement is true.\\ \newline
\(\exists a,p,l,n\in \mathbb{N},\ a<n, \  p^l|a\) and \(k\in\mathbb{N},\ l<k,\ p^k\nmid a\) for example \(p^l|p^l\) and \(p^l<p^k\). Then also due to statement 3, for
\(a,b,p,k,l\in\mathbb{N},\ l<k\ \{a:p^k|a\}\subseteq\{b:p^l|b\} \). Therefore, \(\{a:p^k|a\}\subseteq\{a:p^{k-1}|a\}\subseteq\dots\subseteq\{a:p^2|a\}\subseteq\{a:p^1|a\}\).
Therefore, for \(l,h\in \mathbb{N},\ 0<l<h,\ \{a:p^l|a\}-\{a:p^{l+1}|a\}\) give a set \(T\), that \(T\subseteq\{a:p^l|a\}\)and \(T\nsubseteq\{a:p^h|a\}\). 
According to statement 2, \(\displaystyle{\sum^k_{k=1}\left\lfloor \frac{n}{p^k}\right\rfloor=\sum^k_{k=1}m_k}\).
Thus, \(m_l - m_{l+1}\) give the number of elements in set \(T\) mentioned above. Also, \(m_k\), represents how many numbers smaller than \(n\) 
divides \(p^k\). Therefore, due to statement 1, there are \(m_1\) possible numbers that have a prime facterization containing prime \(p\).
Therefore, the power of \(p\) in the facterization \(a_p\) would have following property.
\begin{align*}
    a_p&=1(m_1-m_2)+2(m_2-m_3)+\dots+(k-1)(m_{k-1}-m_k)+k(m_k)\\
        &=m_1+(-1+2)m_2+(-2+3)m_3+\dots+(-(k-1)+k)m_k\\
        &=\sum_{k=1}^i m_k (p^i\leqslant n)
\end{align*}
For \(j>i\), \(\left\lfloor \frac{n}{p^k}\right\rfloor = 0\), therefore, \(\displaystyle{a_p = \sum_{k=1}^k m_k=\sum^k_{k=1} \left\lfloor \frac{n}{p^k}\right\rfloor}\)
\newpage
\section*{Problem 4}
\subsection*{a)}
For any \(n\in\mathbb{N}\), \(n\) have \(k\) digits, there is \(\displaystyle{n=\sum^k_{i=0}10^id_i}\) where \(d_i\) represents the \(i\) digit of number \(n\) starting with 0.
Then, there is:
\begin{align*}
    \sum^k_{i=0}10^id_i&=\sum^k_{i=0} (10^i-1+1)d_i\\
                            &=\sum^k_{i=0} \left(\left(\sum^k_{i=0} 10^{i-1}\times 9\right)+d_i\right)\\
                            &=\sum^k_{i=0}10^{i-1}\times 9 + \sum^k_{i=0}d_i\\
                            &=9\sum^k_{i=0}10^{i-1} + \sum^k_{i=0}d_i
\end{align*}
Thus, \(9|9\sum^k_{i=0}10^{i-1}\), and thus \(n/9\equiv \left(\sum^k_{i=0}d_i\right)/9\) (mod \(9\)).
Thus, if \\ \(n/9\equiv 0 (\text{mod }9)\), then \(\left(\sum^k_{i=0}d_i\right)/9 (\text{mod }9)\), then \(\sum^k_{i=0}d_i (\text{mod }9)\). 
Also, if \(\left(\sum^k_{i=0}d_i\right)/9 \equiv 0 (\text{mod }9)\). Then, \(n/9\equiv 0 (\text{mod }9)\), then \(n\equiv 0 (\text{mod }9)\). 
Thus, positive integer is divisible by 9 if and only if the sum of its digits is divisible by 9.\newpage
\subsection*{b)}
For any \(n\in\mathbb{N}\), \(n\) have \(k\) digits, there is \(\displaystyle{n=\sum^k_{i=0}10^id_i}\) where \(d_i\) represents the \(i\) digit of number \(n\) starting with 0.
Then, there is: 
\begin{align*}
    \sum_{i=0}^{\left\lceil \frac{k}{3}\right\rceil} \left(10^{3i}\times \sum^2_{j=0} 10^jd_{j+3i}\right)
    =&\sum_{i=0}^{\left\lceil \frac{k}{3}\right\rceil} \left((10^{3i}-10^{3i-3}+10^{3i-3})\sum^2_{j=0} 10^jd_{j+3i}\right)\\
    =&999\times \sum_{i=0}^{\left\lceil \frac{k}{3}\right\rceil} 10^{3i-3}+\sum_{i=0}^{\left\lceil \frac{k}{3}\right\rceil}10^{3i-3}\sum^2_{j=0} 10^jd_{j+3i}\\
\end{align*}
\(999|999\times \sum_{i=0}^{\left\lceil \frac{k}{3}\right\rceil} 10^{3i-3}\) and \(37|999\). Thus, \(999\times \sum_{i=0}^{\left\lceil \frac{k}{3}\right\rceil} 10^{3i-3}\equiv 0 (\text{mod } 37)\)\\
Suppose \(\sum_{i=0}^{\left\lceil \frac{k}{3}\right\rceil}10^{3i-3}\sum^2_{j=0} 10^jd_{j+3i}\equiv x (\text{mod }37)\). \\
Then \(10^3 \sum_{i=0}^{\left\lceil \frac{k}{3}\right\rceil}10^{3i-3}\sum^2_{j=0} 10^jd_{j+3i}\equiv x (\text{mod }37)\). \\Then,
\(999\times \sum_{i=0}^{\left\lceil \frac{k}{3}\right\rceil} 10^{3i-3}+\sum_{i=0}^{\left\lceil \frac{k}{3}\right\rceil}10^{3i}\sum^2_{j=0} 10^jd_{j+3i} \equiv 0+x(\text{mod }37)\)\\ \\
If \(37|n\), then there is \(n\equiv 0(\text{mod }37)\), then \(x=0\), then \(37|\sum_{i=0}^{\left\lceil \frac{k}{3}\right\rceil}10^{3i}\sum^2_{j=0} 10^jd_{j+3i}\). Thus, if \(37|n\) then
the 3-block sum of digits is divisible by 37. Also, if  3-block sum of digits is divisible by 37, then \(x=0\), gives that \(37|n\).\\
Thus, a positive integer is divisible by 37 if and only if the 3-block sum of digits is divisible by 37.
\newpage
\section*{Problem 5}
It is impossible.\\
\(5\equiv1 (\text{mod }2)\)\\
\(9\equiv1 (\text{mod }2)\)\\
\(91\equiv1 (\text{mod }2)\)\\
Therefore, only using one pile we cannot proceed.
If combining them, 3 conditions raises up.\\
a. 14 \& 91\ gcd(14,91)=7\\
b. 100 \& 5\ gcd(100,5)=5\\
c. 96 \& 9\ gcd(96,9)=3\\
There for, no matter how to split the 2 piles, there will always left a pile with pultiplication of either (7,5,3) depend on group. And as none of them are 1, therefore, there is no way to do this.
\end{document}