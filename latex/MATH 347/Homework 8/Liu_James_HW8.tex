\documentclass{article}
\usepackage{graphicx}
\usepackage{amsmath}
\usepackage{array}
\usepackage[font=small, labelfont={sf,bf}, margin=1cm]{caption}
\usepackage{tabularx}
\usepackage{amssymb}
\title{\textbf{Homework \#8 }}
\author{James Liu}
\date{\ }

\begin{document}

\maketitle

\section*{Problem 1}
\subsection*{a)}
Since \(n\) is squarefree, there is that \(n=p_1\cdot p_2\cdot ... \cdot p_k\), where 
\(p_1\) to \(p_k\) are different prime factors. Thus,\(\forall b \in \mathbb{Z}\), \(b^{p_i-1}\equiv 1 (\text{mod }p_i)\) where \(p_i\) is one
of the prime factor of \(n\). As \(p_i-1|n-1\), then \(\exists j \in \mathbb{Z}\) that \(j(p_i-1)=n-1\). Thus, there is \({b^{p_i-1}}^j\equiv b^{n-1} \equiv 1 (\text{mod }p_i)\).
Each prime factor of \(n\) are relatively prime as they are themself primes. Thus, there \(b^{n-1}\equiv 1 (\text{mod }p_1\cdot p_2 \cdot ... \cdot p_k)\),
 which is equivalent to \(b^{n-1} \equiv 1 (\text{mod }n)\). Thus, \(n\) is a Carmichael number.
\subsection*{b)}
For all \(b\) that is relatively prime with \(n\), appling Fermat's little theorem:
\begin{align*}
    b^{18k}&\equiv 1 (\text{mod }(18k+1))\\
    b^{12k}&\equiv 1 (\text{mod }(12k+1))\\
    b^{6k}&\equiv 1 (\text{mod }(6k+1))\\
\end{align*}
Thus, \(b^{36k}\equiv 1 (\text{mod }n)\)\\
For \(n=(6k+1)(12k+1)(18k+1)\), we can derive:
\begin{align*}
    n&=(6k+1)(12k+1)(18k+1)
    \\  &=(2\cdot3\cdot k+1)(2^2\cdot3\cdot k +1)(2\cdot3^2\cdot k +1)
    \\  &=(2^3\cdot3^2k^2+2\cdot 3^2 k+1)(2\cdot3^2\cdot k +1)
    \\  &=2^43^4k^3+2^23^4k^2+2\cdot3^2 k+2^3\cdot3^2k^2+2\cdot 3^2 k+1\
    \\  &=2^43^4k^3+2^23^2(3^2+2)k^2+2^23^2k+1
\end{align*}
Thus, \(n-1=2^43^4k^3+2^23^2(3^2+2)k^2+2^23^2k\), and thus:\\
\(\displaystyle \frac{n-1}{36k}=\frac{n-1}{2^23^2 k}=2^23^2k^2+(3^2+2)k+1\) which is a integer, meaning that \(36k|n-1\).
Thus, \(\exists j \in \mathbb{Z}\) that \(\left(b^{36k}\right)^j= b^{n-1}\). Thus, \(b^{n-1}\equiv1(\text{mod }n)\) and \(b\) is a 
Carmichael number.
\section*{Problem 2}
\(|\mathbb{Z}_p-\{0\}|=p-1\), as \(2\) pre-image gives \(1\) image, and there are \(p-1\) pre-images, there are \(\frac{p-1}{2}\) images.
\section*{Problem 3}
\subsection*{a)}
It failes closure as \(3\times 1 + 3\times 3=10\) and \(3\nmid 10\)
\subsection*{b)}
It failes closure as \(\displaystyle \frac{3}{5}\times \frac{7}{5}= \frac{21}{25}\) and \(\displaystyle \frac{21}{25}\notin \frac{1}{5}\mathbb{Z}\)
\subsection*{c)}
It failes multiplicative communitivity.\\
\(\left<1,1,1\right>\times \left<0,0,1  \right>= \)
$\begin{bmatrix}
    \hat{i}&\hat{j}&\hat{k}\\
    1&1&1\\
    0&0&1\\
\end{bmatrix}$
\(=\left<1,-1,0\right>\)\\
\(\left<0,0,1\right>\times \left<1,1,1  \right>= \)
$\begin{bmatrix}
    \hat{i}&\hat{j}&\hat{k}\\
    0&0&1\\
    1&1&1\\
\end{bmatrix}$
\(=\left<-1,1,0\right>\)

%%% it is NOT finished, i should point out every xiom it does not satisfy.












\end{document}