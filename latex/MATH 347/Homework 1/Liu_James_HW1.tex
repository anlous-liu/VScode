\documentclass{article}
\usepackage{graphicx}
\usepackage{amsmath}
\usepackage{array}
\usepackage[font=small, labelfont={sf,bf}, margin=1cm]{caption}
\usepackage{tabularx}
\usepackage{amssymb}

\begin{document}
\title{\textbf{Homework \#1 }}
\author{James Liu}
\date{\ }
\maketitle

\section*{Problem 1}
    
    a) There is a \(x\in A\) such that for all \(b \in B\), there is \(b \le x\)\\
    b) For all \(x \in A\) there is a \(b\in B\) satisfies \(b \le x\)\\
    c) There is a pair of \(x,y \in \mathbb{R}\) such that \(x\neq y\), satisfies \(f(x)=f(y)\)\\
    d) There is a \(\varepsilon \in P\), that all \(\delta \in P\) satisfies the
    statement that exist a pair of \(x,y \in \mathbb{R}\) that \(|x-y| < \delta\) and \(|f(x)-f(y)|\geq \varepsilon\)
\section*{problem 2}
    The negation of the statement is: \\
    There is a group of sets \(A\), \(B\) and \(C\) that satisfies \(A\nsubseteq B\), \(B \subseteq C\) and \(A \subseteq C\).
    Suppose there are three sets \(A=\{1,2\}, B=\{2,3\},C=\{1,2,3\}\). \\ 
    There is \(1 \in A, 1\notin B\). Therefore, \(A \nsubseteq B\).\\
    There is \(2,3 \in C\). Therefore, \(B \subseteq C\).\\
    There is \(1,2 \in C\). Therefore, \(A \subseteq C\).\\
    Thus, this group of sets satisfies \(A\nsubseteq B\), \(B \subseteq C\) and \(A \subseteq C\).\\
    The negation of the statement is true, Therefore the original statement is false.
\section*{Problem 3}
    Suppose there is a new set \(C\), that \(A\cap B = A\cup B = C\).\\
    There is\(A\cup B =C\), so every element in B and A shall be in C. \\
    Therefore\(A\subseteq C\), and \(B \subseteq C\).\\
    There is\(A\cap B =C\), so every element in \(C\) shall be in both \(A\) and \(B\).\\
    Therefore\(C\subseteq A\), and \(C \subseteq B\).\\
    There is \(C\subseteq A\), and\(A \subseteq C\), so every element in \(C\) is in \(A\) and every element in \(A\) in also in \(C\).\\
    Therefore, \(C = A\). Similarly, \(C = B\). Thus, \(A=B\).
\section*{Problem 4}
\subsection*{1. If \(A = B\), then \(A-B=B-A\)}
        There is\(A=B\), so every elements in set \(A\) is also in \(B\) and vice versa.\\
        Therefore, such element that is in set \(A\), but not in set \(B\) does not exist.\\
        Thus, \(A-B=\varnothing\), and, similarly, \(B-A = \varnothing\). 
        Therefore, \(A-B = B-A\).\\
\subsection*{2. If \(A-B=B-A\), then \(A = B\)}
        Suppose \(A\neq B\), so there shall be\(x \in A, x\notin B\). Therefore, \(x\in A-B\)\\
        Therefore, \(A-B \neq \varnothing\). Similarly, \(B-A \neq \varnothing\).\\
        Therefore, suppose \(a \in A-B, b \in B-A\), then \(a \in A, a\notin B, b\in B, b\notin A\).\\
        If the statement:" If \(A-B=B-A\), then \(A\neq B\)." is true, then there should be a pair of \(a=b\).\\
        There is \(a \in A, b \notin A\). Therefore \(a\neq b\). Thus, the statement is false.\\
        The statement is the negation of the statemen:"If \(A-B=B-A\), then \(A = B\)".\\
        Therefore, the statement is true.
\subsection*{Thus, the original statement is true}

\section*{Problem 5}
    There is \(0\) times every other real number equals zero. \\
    Therefore, If \(x ={a_1}\),\((x-a_1)(x-a_2)\cdots(x-a_n)=0\). Therefore, \(x\neq a_1\).\\
    Similarly,\(x\neq a_1, a_2, a_3, \cdots, a_n\).
    \(A = (a_n,a_{n-1})\cap (a_{n-2},a_{n-3})\cap (a_{n-4},a_{n-5})
    \cap \dots \cap odd \ and \ even \ selection\)

\end{document}