\documentclass[a4paper,fleqn]{article}
\usepackage{graphicx}
\usepackage{amsmath}
\usepackage{array}
\usepackage[font=small, labelfont={sf,bf}, margin=1cm]{caption}
\usepackage{tabularx}
\usepackage{amssymb}


\title{\textbf{Homework \#2 }}
\author{James Liu}
\date{\ }


\begin{document}

\maketitle


\section*{Problem 1}
    
    \[
        \left |\frac{x}{x+1} \right |\leqslant 1 
    \]\[
        -1 \leqslant \frac{x}{x+1} \leqslant 1 
    \]
    1. suppose \(x+1 > 0\)
    \[
    -x-1 \leqslant x \leqslant x+1    
    \]\[
    -1 \leqslant 2x \leqslant 2x+1    
    \]\[
    - \frac {1} {2} \leqslant x\leqslant x+\frac1 2
    \]
    Because 
    \(\forall x \in \mathbb{R},\  x \leqslant x + \frac 1 2\)
    Therefore, solution set will be \(x\geqslant -\frac 1 2\) \\
    which is \(\left[-\frac 1 2, \infty\right)\)\\
    2. suppose \(x+1 < 0\)
    \[
    -x-1 \geqslant x \geqslant x+1    
    \]\[
    -1 \geqslant 2x \geqslant 2x+1    
    \]\[
    - \frac {1} {2} \geqslant x\geqslant x+\frac1 2
    \]
    Because
    \(\forall x \in \mathbb{R},\  x \leqslant x + \frac 1 2\). Therefore, there are no solution in this region.
    Therefore, the solution set is \(\left[-\frac 1 2, \infty\right)\).

\section*{Problem 2}
    \subsection*{a)}
        Because that \(f\) is increasing and \(g\) is decreasing, \(\forall x,y, \ x\in A, \text{and } y\in A \),
        if \(x<y, \ \text{then }f(x)\leqslant f(y)\). Similarly, if \( x<y, \ \text{then } g(x)\geqslant g(y)\), then \(-g(x)\leqslant -g(y)\).
        \(\forall x \in A, f-g \text{ at }x \ \text{equals to } f(x)-g(x)\), similarly, \(f-g \text{ at }y \ \text{equals to } f(y)-g(y)\).
        Recall previous inequalities of \(f(x)\leqslant f(y)\ \text{and } -g(x)\leqslant -g(y) \). Adding them will give
        \(f(x)-g(x) \leqslant f(y)-g(y)\) under that \(x<y\). Therefore, \(f-g\) is increasing.
    


\end{document}