\documentclass[a4paper,fleqn]{article}
\usepackage{graphicx}
\usepackage[fleqn]{amsmath}
\usepackage{array}
\usepackage[font=small, labelfont={sf,bf}, margin=1cm]{caption}
\usepackage{tabularx}
\usepackage{amssymb}


\title{\textbf{Homework \#2 }}
\author{James Liu}
\date{\ }


\begin{document}

\maketitle


\section*{Problem 1}
    
    \[
        \left |\frac{x}{x+1} \right |\leqslant 1 
    \]\[
        -1 \leqslant \frac{x}{x+1} \leqslant 1 
    \]
    1. suppose \(x+1 > 0\)
    \[
    -x-1 \leqslant x \leqslant x+1    
    \]\[
    -1 \leqslant 2x \leqslant 2x+1    
    \]\[
    - \frac {1} {2} \leqslant x\leqslant x+\frac1 2
    \]
    Because 
    \(\forall x \in \mathbb{R},\  x \leqslant x + \frac 1 2\)
    Therefore, solution set will be \(x\geqslant -\frac 1 2\) \\
    which is \(\left[-\frac 1 2, \infty\right)\)\\
    2. suppose \(x+1 < 0\)
    \[
    -x-1 \geqslant x \geqslant x+1    
    \]\[
    -1 \geqslant 2x \geqslant 2x+1    
    \]\[
    - \frac {1} {2} \geqslant x\geqslant x+\frac1 2
    \]
    Because
    \(\forall x \in \mathbb{R},\  x \leqslant x + \frac 1 2\). Therefore, there are no solution in this region.
    Therefore, the solution set is \(\left[-\frac 1 2, \infty\right)\).

\section*{Problem 2}
    \subsection*{a)}
        Because that \(f\) is increasing and \(g\) is decreasing, \(\forall x,y, \ x\in A, \text{and } y\in A \),
        if \(x<y, \ \text{then }f(x)\leqslant f(y)\). Similarly, if \( x<y, \ \text{then } g(x)\geqslant g(y)\), then \(-g(x)\leqslant -g(y)\).
        \(\forall x \in A, f-g \text{ at }x \ \text{equals to } f(x)-g(x)\), similarly, \(f-g \text{ at }y \ \text{equals to } f(y)-g(y)\).
        Recall previous inequalities of \(f(x)\leqslant f(y)\ \text{and } -g(x)\leqslant -g(y) \). Adding them will give
        \(f(x)-g(x) \leqslant f(y)-g(y)\) under that \(x<y\). Therefore, \(f-g\) is increasing.
    \subsection*{b)}
        If \(f\) is not increasing, then  \( \forall x,y \in A, \text{ when } x<y,\) there is \(f(x)>f(y)\). \\ 
        Therefore \(-1 \times f(x)< -1 \times f(y)\) which is \(-f(x)<-f(y)\) under \(x<y\). Therefore, \(-f\) is not decreasing.
\section*{Problem 3}
    \subsection*{a)}
    It is true.
        Because \(f,g\) are both odd, there is \(\forall x\in \mathbb{R}, \ f(-x)=-f(x)\) and \(g(-x)=-g(x)\). 
        Adding them will give \(f(-x)+g(-x)=-\left(f(x)+g(x)\right)\). Suppose \(h=f+g\), then the expression 
        turns \(h(-x)=-h(x)\) under \(\forall x \in \mathbb{R}\). Therefore, \(f+g\) is odd.
    \subsection*{b)}
    It is true.
        For \(f\) to be bounded, \(M\in \mathbb{R}, \ \text{ and} \ \exists M > 0\) so that \(\forall x \in \mathbb{R}, \ \left | f(x) \right | < M\)
        Similarly, for \(N\in \mathbb{R}\), and \(\exists N>0\) so that \(\forall x\in \mathbb{R}, \ |g(x)|<N\).
        Because \(M>0,\ N>0,\ |f(x)|\geqslant 0,\ |g(x)| \geqslant 0\), then multiply them together will give \(|f(x)|\cdot |g(x)|<N\cdot M\) Which is 
        equivalent to \(|f(x)\cdot g(x)|<N\cdot M\). Because \(N\) and \(M\) are all finite numbers, \(N\cdot M\) will also be finite number. Thus, \(f\cdot g\) is still bounded.
    \subsection*{c)}
    It is false.
        For example, if \(f(x)=e^x\) and \(g(x)=\frac x {e^x}\). \(f\cdot g = x\). \(\forall a \in \mathbb{R}, \ 
        -x = -x \). Thereofre, \(f\cdot g\) is an odd function. However, for example, take\(take x=1, \ and x=-1\). 
        \(f(1)=e^1\), and \(f(-1)=e^{-1}\). Because \(-e^1 \neq e^{-1}\), \(f(x)\) is not odd. Similarly, \(g(-1)=-e\)
        , and \(g(1)=\frac 1 e\). Because \(\frac 1 e \neq -e\), \(g(x)\) is also not odd.
    \subsection*{d)}
    It is false.
        For example, if \(f(x)=e^x\) and \(g(x) = e^{-x}\), \(f(x)\cdot g(x) = 1\) which is bounded. While  \(e^x\) and \(e^{-x}\) are all unbounded.
        Therefore, the statement is false.
\section*{Problem 4}
    1. If \(A=B\):\\ 
    Because \(A=B\), \(A \subseteq B, \text{ and }B\subseteq A\). Because \(A\subseteq B\), there is \(A\cup B = B\). Similarly, 
    because \(B\subseteq A\), \(A\cup B = A\). Thus,if \( A=B, A\cup B = A = B\). Because \(A\subseteq B\), there is \(A\cap B = A\). Similarly, 
    because \(B\subseteq A\), \(A\cap B = B\). Thus,if \( A=B, A\cap B = A = B\). Thus, if \( A=B, A\cup B = A\cap B\).
    \\2. If \(A\cap B = A\cup B\):\\
    Suppose \(x\in A\), then \(x\in A\cup B\). Because \(A\cap B = A\cup B\), \(x\in A\cap B\). Thus, \(x\in B\) which means 
    \(A \subseteq B\). Similarly , if \(y\in B\), then \(y\in A\cup B\), then \(y\in A\cap B\), and \(y\in B\) meaning \(B\subseteq A\). Thus, \(A=B\)
\section*{Problem 5}
    \subsection*{a)}




        A bettter prof.\\
            Suppose \(x\in (A\triangle B)\triangle C=((A\triangle B)-C )\cup (C-(A\triangle B))\).\\
            Then \(x\in A\triangle B\) and \(x\notin C\), 
            or \(x\in C\) and \(x\notin (A\triangle B)\).\\
            To \(x\in (A\triangle B)\) and \(x\notin C\),\\
            there is \(x\in A\) and \(x\notin B\) and \(x\notin C\), \\
            or \(x\notin A\) and \(x\in B\) and \(x\notin C\). \\
            To \(x\in C\) and \(x\notin (A\triangle B)\), \\
            there is \(x\in C\) and \(x\in A\) and \(x \in B\),\\
            or \(x\in C\) and \(x\notin B\) and \(x\notin A\).\\
            Therefore, for \(x\in A\triangle (B\triangle C)\), \\
            \(x\) belongs in exactly one of the 4 following categories. \\
            1. \(x\in C\) and \(x\in A\) and \(x\in B\)\\
            2. \(x\in C\) and \(x\notin A\) and \(x\notin B\)\\
            3. \(x\notin C\) and \(x \in A\) and \(x \notin B\)\\
            4. \(x\notin C\) and \(x\notin A\) and \(x\in B\)\\
            Similarly, Suppose \(x\in (C\triangle B)\triangle A=((C\triangle B)-A )\cup (A-(C\triangle B))\).\\
            Then \(x\in C\triangle B\) and \(x\notin A\), 
            or \(x\in A\) and \(x\notin (C\triangle B)\).\\
            To \(x\in (C\triangle B)\) and \(x\notin A\),\\
            there is \(x\in C\) and \(x\notin B\) and \(x\notin A\), \\
            or \(x\notin C\) and \(x\in B\) and \(x\notin A\). \\
            To \(x\in A\) and \(x\notin (C\triangle B)\), \\
            there is \(x\in A\) and \(x\in C\) and \(x \in B\),\\
            or \(x\in A\) and \(x\notin B\) and \(x\notin C\).\\
            Therefore, for \(x\in C\triangle (B\triangle A)\), \\
            \(x\) belongs in exactly one of the 4 following categories. \\
            1. \(x\in A\) and \(x\in C\) and \(x\in B\)\\
            2. \(x\in A\) and \(x\notin C\) and \(x\notin B\)\\
            3. \(x\notin A\) and \(x \in C\) and \(x \notin B\)\\
            4. \(x\notin A\) and \(x\notin C\) and \(x\in B\)\\
            \\
            Thus, in both circumstances, any element in the solution set is in exactlly one of the following set.\\
            1. \(x\in A\) and \(x\in B\) and \(x\in C\)\\
            2. \(x\in A\) and \(x\notin B\) and \(x\notin C\)\\
            3. \(x\notin A\) and \(x\in B\) and \(x\notin C\)\\
            4. \(x\notin A\) and \(x\notin B\) and \(x\in C\)\\
            Thus, symmetric difference do satisfy associative law.
            


\newpage
    \subsection*{b)}
        The statement proofing will here be: \(A\triangle B \triangle C\) 
        can be characterize as the set of element belonging exactlly one or all of three of the sets A,B,C, which is the shaded area of figure 1.
        \begin{figure}[h]
            \centering
            \includegraphics*[scale = 0.5]{image/Venn.png}
            \caption{}
        \end{figure}
        
        from the previous answer, the solution set of \(A\triangle B \triangle C\) is in exactlly one of the four sets.\\
            1. \(x\in A\) and \(x\in B\) and \(x\in C\)\\
            2. \(x\in A\) and \(x\notin B\) and \(x\notin C\)\\
            3. \(x\notin A\) and \(x\in B\) and \(x\notin C\)\\
            4. \(x\notin A\) and \(x\notin B\) and \(x\in C\)\\
            The solution sets basically means that it the element is either in all three of sets or in only one of the sets.
            
        Therefore, \(A\triangle B \triangle C\) 
        can be characterize as the set of element belonging exactlly one or all of three of the sets A,B,C.
\end{document}

