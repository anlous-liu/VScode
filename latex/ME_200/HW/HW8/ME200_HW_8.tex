\documentclass{article}
\usepackage{graphicx}
\usepackage{amsmath}
\usepackage{array}
\usepackage[font=small, labelfont={sf,bf}, margin=1cm]{caption}
\usepackage{tabularx}
\usepackage{amssymb}


\date{Due: Oct 25 Edit: \today}
\title{ME 200 Homework 8}
\author{James Liu}

\begin{document}
\maketitle
\begin{itemize}
    \item [1.] 
    \begin{itemize}
        \item [a)]
        \begin{align*}
            s_1 &= 8.2570\\
            u_1 &= 2430.1\\
            T_2 &=\frac{500-440}{8.3251-8.1538}\times (8.2570-8.1538) +440\\
            &=476.147 \ {}^\circ C\\
            u_2 &=\frac{3130-3030.6}{8.3251-8.1538}\times (8.2570-8.1538)+3030.6\\
            &=3090.48\\
            \Delta u &= u_2-u_1\\
            &=660.384 \text{ kJ/kg}
        \end{align*}
        \item [b)]
        \begin{align*}
            s_1 &= 0.8\times 8.9501 + 0.2\times 0.1212 = 7.18432\\
            h_1 &= 0.8\times 2574.3 + 0.2 \times 167.57 = 2092.95\\
            T_2 &= \frac{320-280}{7.1962-7.0465}\times (7.18432-7.0465)+280\\
            &=316.826\ {}^\circ C\\
            h_2 &= \frac{3093.9-3008.2}{7.1962-7.0465}\times (7.18432-7.0465)+3008.2\\
            &=3087.1\\
            \Delta h &= u_2-u_1\\
            &=3087.1-2092.95\\
            &=994.149 \text{ kJ/kg}
        \end{align*}
    \end{itemize}
    \newpage
    \item[2.]
    \begin{align*}
        u_1 &= 6229\\
        s^\circ_1 &= 197.732\\
        u_2 &= 7689\\
        s^\circ_2 &=203.842\\
    \end{align*}
    \begin{itemize}
        \item [a)]
        \begin{align*}
            Q &= W-n(u_2-u_1)\\
            &= -300 + 0.1(7689-6229)\\
            &=-154 \text{ kJ}
        \end{align*}
        \item[b)]
        \begin{align*}
            \Delta s &=n\left((s^\circ_2-s^\circ_1)-R \ln\left(\frac{P_2}{P_1}\right)\right)\\
            &=0.1\left((203.842-197.723)-8.314 \ln \left(\frac{500}{150}\right)\right)\\
            &=-0.389\ \text {kJ/K}
        \end{align*}
        \includegraphics*[scale = 0.25]{figure/fig1_ME200.jpeg}
    \end{itemize}
    \newpage
    \item [3.]
    \begin{align*}
        s_1  &= s_f = 1.9426\\
        u_1  &= u_f = 674.79\\
        Q  &= mT(s_2-s_1)\\
        2700&=2\times (160+273)(s_2-1.9426)\\
        s_2 &= 5.0604\\
        \text{for \(T_2\) = 160:}\\
        P_2 &=\fbox{6.1823 bar}\\
        5.0604 &= 1.9426+ x(4.8066)\\
        x&=0.649\\
        u_2 &= u_f+x u_{fg}\\
        &= 674.79 + 0.649\times 1893.0\\
        &=1903.347
        \Delta U = m(u_2-u_1)\\
        &=2(1903.347-674.79)\\
        &=2457.114\\
        W &= Q-\Delta U\\
        &=2700-2457.114\\
        &=\fbox{242.886 kJ}\\
    \end{align*}
    \item[4.] 
    \begin{align*}
        h_3 &= h_f+x_3(h_g-h_f)\\
        &=583.77+0.9\times (1634.23-583.77)\\
        &=1529.18\\
        s_3 &= 2.2706+0.9(5.5213-2.2706)\\
        &=5.196 = s_2\\
        h_2 &= 1244\\
        h_4 &=h_f = 583.77\\
        s_4 &=s_f = 2.2706\\
        s_1 &= s_4 = 2.2706\\
        s_1&= s_f+x_1(s_g-s_f)\\
        x_1 &=0.237\\
        h_1 &=h_f+x_1(h_g-h_f)\\
        &=206.76+0.237(1566.47-206.76)\\
        &=529.17\\
        W_n&=nm((h_3-h_2)-(h_4-h_1))\\
        &=50\times 0.1 \times ((1529.18-1244)-(583.77-529.17))\\
        &=1152.9 \text{ kJ}
    \end{align*}
    \item[5.]
    \begin{align*}
        p_1 &= 1 \text{bar}\\
        u_1 &= 2537.3\\
        s_1 &=7.4668\\
        p_2 &= 100 \text{bar}\\
        s_2=s_1 &=7.4668\\
        u_2 &=(7.4668-7.2670)\frac{3515.1-3434.7}{7.2670-7.1687}+3512.1\\
        &= 3675.52\\
        T_2 &= (7.4668-7.2670)\frac{740-700}{7.2670-7.1687}+740\\
        &=821.302\ ^\circ C\\
        W/m &=(u_1-u_2)\\
        &=-1142.7 \text{kJ/kg}
    \end{align*}
\end{itemize}     
\end{document}