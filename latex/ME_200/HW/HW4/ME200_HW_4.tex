\documentclass{article}
\usepackage{graphicx}
\usepackage{amsmath}
\usepackage{array}
\usepackage[font=small, labelfont={sf,bf}, margin=1cm]{caption}
\usepackage{tabularx}
\usepackage{amssymb}



\date{Due: Sep 27  Edit: \today}
\title{ME 200 Homework 4}
\author{James Liu}

\begin{document}
\maketitle
\begin{itemize}
    \item [1.] \begin{align*}
        \nu_1 = \nu_2 &= 0.4625 \text{ kg/m}^3\\
        \frac{2960.9-2963.2}{0.4397-0.6173}&=12.9505\\
        0.4397\times 12.9505+2960.9&=2961.2 \text{ kJ/kg}\\
        Q = (2961.2 - 2553.6 )\times 1 &= \fbox{\(407.595 \text{ kJ/kg}\)}
    \end{align*}

    \item [2.]\ 
    \begin{itemize}
        \item [\(1\rightarrow 2\):] \[Q = mT(s_2-s_1) = 1\times (503.23+459.67)\times(1.4305-0.6927) = \fbox{\(710.428 \text{ Btu}\)}\]
                                    \[W = 710.428-(1117-488.9)=\fbox{\(82.32 \text{Btu}\)}\]
        \item [\(2\rightarrow 3\):] \begin{align*}
            6.20x+(1-x)0.01748 &=0.656\\
            x &=0.1032\\
            u_3 = xu_g+(1-x)u_f &= 358.114 \text{ Btu/lb}\\
            Q = m(u_3-u_2) &= \fbox{-758.886}\\
            W = \fbox{0}
        \end{align*}
        \item [\(3\rightarrow 4\):]
        \begin{align*}
            s_3 = x_3s_g+(1-x_3)s_f &= 0.5642\\
            6.20x_4+(1-x_4)0.01748 &=0.02501\\
            x_4&=0.001218\\
            s_4 = x_4s_g+(1-x_4)s_f &=0.44265\\
            u_4 = x_4u_g +(1-x_4)u_f&=273.608\\
            Q = mT(s_4-s_3) &= 1\times (302.96+459.67)(0.44265-0.5642)\\
            &=\fbox{-92.6973 Btu}\\
            W = Q-m(u_4-u_3)&=-92.6973 -273.608+358.114 = \fbox{-8.1918 Btu}
        \end{align*}
        \item [\(4\rightarrow 1\):] \begin{align*}
            Q = m(u_1-u_4) & = \fbox{215.292 Btu}\\
            w = \fbox{0 Btu}
        \end{align*}
        \(\displaystyle \eta = q-\frac{Q_{out}}{Q_{in}} = 1-\frac{758.886+92.6973}{710.428+215.292}=0.080085\)

    \end{itemize}
    \item [3.] \
    \begin{itemize}
        \item [\(1\rightarrow 2\):] Since isothermal, \(W = pv \text{ln}(\frac{v_2}{v_1}) = 5\times 10^5 \times \text{ln}(\frac{5}{1})= 804719\text { J/kg}\)\\
        \item [\(2\rightarrow 3\):] \begin{align*}
            T_2 = \frac{Pv}{nR}=\frac{10^5\times 5}{\frac{1}{28.97\times 10^-3}\times 8.314} = 1742.24 \text{ K}\\
            T_3 = \frac{Pv}{nR}=\frac{10^5}{286.987}=348.448 \text{ K}\\
            u_2 = \frac{1439.8-1392.7}{1750-1700}\times 42.24 + 1392.7 = 1432.49 \text{ kJ/kg}\\
            u_3 = \frac{250.02-242.82}{350-340}\times 8.448 +242.82 = 248.903\text{ kJ/kg} \\
            W = p\Delta V = 10^5 \times -4 = -4\times 10^5 \text{ J/kg}\\
            Q = Q-W+W = (u_3-u_2)+W = -1.58359\times 10^6 \text{J/kg}\\
        \end{align*}
        \item [\(3\rightarrow 1\):]\begin{align*}
            T_1 = \frac{Pv}{nR} = \frac{5\times 10^5}{286.987}=1742.24 \text{ K}\\
            u_1 = \frac{1439.8-1392.7}{1750-1700}\times 42.24 + 1392.7 = 1432.49 \text{ kJ/kg}\\
            Q = m(u_1-u_3) = 1432.49-248.903 = 1183.59 \text{kJ/kg}
        \end{align*}
    \end{itemize}
    \[\eta = 1-\frac{Q_{out}}{Q_{in}} = \frac{1.58359\times 10^5}{1183.59\times 10^3+804719} = \fbox{0.20355}\]
    \item [4.]\
    \begin{itemize}
        \item [in:] \(\frac{20\times 10\times10^{-3}}{0.1996}=1.002\) kg/s
        \item [exit:] \(\frac{1\times 6\times 10^{-3}}{1.905\times 10^-3}=5.50206\) kg/s
        \item [total:] 1.002-5.50206=\fbox{-4.5006 kg/s}
    \end{itemize}
    \item [5.]
    \begin{itemize}
        \item [a)] \(v = RT/p = 286.987 \times 450 / 350\times 10^3 = 0.3689 \text{ m\(^3\)/kg}\)\\
                    \(A = \frac{mv}{V}=2.3\times0.3689/3 = 0.2829 \text{ m}^2\)
        \item [b)] \(\Delta Q = \Delta U +\Delta E_v -\Delta W= m\Delta Tc_p+0.5m(\Delta v)^2-0\)\\
                    \(=2.3\times (1.011\times 10^3 (300-450)+0.5\times (460^2-3^2)) = -105465= \fbox{-105.465 kW}\)

    \end{itemize}
\end{itemize}
\end{document}