\documentclass{article}
\usepackage{graphicx}
\usepackage{amsmath}
\usepackage{array}
\usepackage[font=small, labelfont={sf,bf}, margin=1cm]{caption}
\usepackage{tabularx}
\usepackage{amssymb}



\date{Due: Sep 18 Edit: \today}
\title{MATH 416H HW 3}
\author{James Liu}

\begin{document}
\maketitle

\begin{itemize}
    \item [1.] \(\forall W \subseteq V\) that \(W\) is a subspace containing \(S\),  \(S\subseteq \bigcap_i W_i\) as \(W\)s are vector spaces, thus they are closed on the 2 operations, thus, linear combinations of \(S\) are also in any \(W\) containing S.
    Thus, \((\text{span }S)\subseteq\bigcap_i W_i\). \\ \\
    Suppose that  \(\bigcap_iW_i\not\subseteq\text{span }S\), then it exisist such \(v\in V\), that \(v\notin \text{span }S\), while \(v\in \bigcap_iW_i\). Consider a special subspace (span \(S)\subseteq V\), \(S\subseteq \text{span} S\), thus \(span S \in \{W_i\}\) while span \(S\subseteq\)span \(S\). Therefore, any \(v\notin span\ S, v\notin \bigcap_iW_i\). Thus, by definition of intersections, such vector does not exist, thus, \(\bigcap_iW_i\subseteq\text{span }S\)
    \\ \\
    Thus, the span of \(S\) equals the intersection of all
    the subspaces of \( V\) containing \(S\).
    \item [2.] For the 2 vectors to be linear independent, the only solution to \(\left\{\begin{matrix}
    \lambda_1+\lambda_2 = 0\\
    x\lambda_1+y\lambda_2 = 0
    \end{matrix}\right.\) 
    should be \(\left\{\begin{matrix}
        \lambda_1=0\\\lambda_2 = 0
    \end{matrix}\right.\).
    for \(\lambda_1+\lambda_2 = 0\), there could be \(\lambda_1=\lambda_2=0\) or \(\lambda_1=-\lambda_2\) As the vectors should  be linear independent, then, \(-\lambda_1x+\lambda_2y\neq0\), thus \(x\neq y\).\\

    For vectors with \(x\neq y\), there is \(\left\{\begin{matrix}
    \lambda_1+\lambda_2 = 0\\
    x\lambda_1+y\lambda_2 = 0
    \end{matrix}\right.\) 
    with only one set of solution of \(\lambda_1=\lambda_2 = 0\), Thus they are linearly independent with condition \(x\neq y\).
    \item [3.]
    \begin{itemize}
        \item [i:]\(\displaystyle \begin{pmatrix}
            1\\2\\1
        \end{pmatrix}\) is one of the vectors that can form a basis. suppose they are linear dependent, then exisist \(\lambda_1,\lambda_2,\lambda_3\in\mathbb{R}\), \\that \(\lambda_1\begin{pmatrix}
            0\\0\\1
        \end{pmatrix} +\lambda_2 \begin{pmatrix}
        1\\1\\0
        \end{pmatrix} + \lambda_3 \begin{pmatrix}
            1\\2\\1
            \end{pmatrix} = \begin{pmatrix}
                0\\0\\0
            \end{pmatrix}\)\\
        while \(\exists\lambda_i\neq0\) however \(\left\{\begin{matrix}
            \lambda_2+2\times\lambda_3=0\\
            \lambda_2+\lambda_3 = 0
        \end{matrix}\right.\) Showing that \(\lambda_2=\lambda_3 = 0\) and thus, as \(\lambda_1+0+0=0\)
        \(\lambda_1 = 0\), Thus they are linear independent, as there are 3 elements while the dimension of \(\mathbb{R}^3\) is also 3, it is a basis.
        \item[ii:]\(\displaystyle \begin{pmatrix}
            1\\0\\1
        \end{pmatrix}\) is one of the vectors that can form a basis. suppose they are linear dependent, then exisist \(\lambda_1,\lambda_2,\lambda_3\in\mathbb{R}\), \\that \(\lambda_1\begin{pmatrix}
            0\\0\\1
        \end{pmatrix} +\lambda_2 \begin{pmatrix}
        1\\1\\0
        \end{pmatrix} + \lambda_3 \begin{pmatrix}
            1\\0\\1
            \end{pmatrix} = \begin{pmatrix}
                0\\0\\0
            \end{pmatrix}\)\\
        while \(\exists\lambda_i\neq0\) however \(\left\{\begin{matrix}
            \lambda_2+\lambda_3=0\\
            \lambda_2= 0
        \end{matrix}\right.\) Showing that \(\lambda_2=\lambda_3 = 0\) and thus, as \(\lambda_1+0+0=0\)
        \(\lambda_1 = 0\), Thus they are linear independent, as there are 3 elements while the dimension of \(\mathbb{R}^3\) is also 3, it is a basis.
        
    \end{itemize}
    \item[4.] As \(x+y+z = \overrightarrow{0}\), thus, \(z = -x-y\). In \(\forall v\in\)span \(\{y,z\}\)
    \(\exists \lambda_1,\lambda_2\in F\) that \(v=\lambda_1y+\lambda_2z=\lambda_1y+\lambda_2(-x-y)=(\lambda_1-\lambda_2)y-\lambda_2x\), \((\lambda_1-\lambda_2 \in F)\), Thus \(v\) is also in 
    span \(\{x,y\}\). Thus, span \(\{x,y\}=\)span \(\{y,z\}\)
    \item[5.] 
    \(N(T) = \{x|T(x)=\overrightarrow{0}\} \), for \(\mathcal{P}_n\), \(N(T) = \{x|\frac{d}{dx} x=0\}=k+0x+0x^2+\cdots+0x^n,\) for some \(k\in\mathbb{R}\)\\
    \(R(T) = \{x|T(v)=x,v\in V\} \), for \(\mathcal{P}_n\), \(R(T) = \{x|\frac{d}{dx} v=x,\ v\in V\}=k_0+k_1x+k_2x^2+\cdots+k_{n-1}x^{n-1}+0x^n\) for some \(k_i\in\mathbb{R}\)\\
    For example, take \(v_1 = 1\) and \(v_2 = x\), \(v_1\in R(T)\) and \(T(v_2)=v_1\in N(T)\). Thus, the intersection is not empty. 











    \item[6.] \(\forall w\in R(T)\), \(\exists\lambda_i\in F\) that \(w = T(\sum_i^n\lambda_i v_i)\). As the map is linear,
    there is \(T(\sum_i^n\lambda_i v_i)=\sum_i^n \lambda_i(T(v_i))\), Thus, \(w\in\)span \(T(v_i)\), thus \(T(v_i)\) spans \(R(T)\), Also, as profed in class, \(T(\overrightarrow{0}) = \overrightarrow{0}\) for linear transformations, and as \(T(v)\) is ,
    then \(T(\sum_i^n\lambda_i v_i)=\overrightarrow{0}\) only when \(\sum_i^n\lambda_i v_i = \overrightarrow{0}\) which is that all \(\lambda_i = 0\).
    Therefore, \(\sum_i^n \lambda_i(T(v_i))=\overrightarrow{0}\) only when all \(\lambda_i = 0\). Thus, \(T(v_i)\) is linear independent. Thus it is one of the basis of \(R(T)\). \\ \\ If the mapping not injective,
    meaning that \(T(v_i)\) is not nessesarily linear independent as there might be some other vectors in the null space of \(T\), leading to a set of \(\lambda_i\) not all equal to 0 that gives \(\sum_i^n \lambda_i(T(v_i))=\overrightarrow{0}\).
    \item[7.]
    \begin{itemize}
        \item [a)] these two operations are closed under the set of all maps between \(X\) and \(W\). 
        \begin{itemize}
            \item [i.] \((f+g)(x)=f(x)+g(x)=g(x)+f(x)=(g+f)(x)\) addition is communitivitive.
            \item [ii.] \((f+(g+h))(x)=f(x)+(g(x)+h(x))=(f(x)+g(x))+h(x)=((f+g)+h)(x)\) addition is associative.
            \item [iii.] Consider \(f(x)=0\), \((f+g)(x) = f(x)+g(x)=0+g(x)=g(x)\) zero vector do exisist.
            \item [iv.] \(\forall f \in \)MAP\((X,W)\), \(-f(x)+f(x) = \overrightarrow{0}\) for all vector do exist a negative vecotr.
            \item [v.] Consider \(\lambda = 1\) \(1f(x) = f(x)\), multiplicative identity exists.
            \item [vi.] \(\lambda,\mu\in\mathbb{R}\), \(\lambda(\mu f(x))=(\mu\lambda)f(x)\)
            \item [vii.] \(\lambda\in\mathbb{R}\), \(\lambda(f+g)(x) = \lambda(f(x)+g(x))=\lambda f(x)+\lambda g(x) = (\lambda f+\lambda g)(x)\)
            \item [viii.] \(\lambda,\mu\in\mathbb{R}\), \((\lambda+\mu)f(x) = \lambda f(x)+\mu f(x)= (\lambda f +\mu f)(x)\)
        \end{itemize}
        Thus it is a vector space.
        \item [b)] \(\forall\) linear maps \(f(x),g(x)\), \((f+g)(\lambda x) = f(\lambda x)+g(\lambda x)= \lambda f(x)+\lambda g(x) = \lambda (f+g)(x)\), thus, addition is closed.
        Consider \(\lambda,\mu\in\mathbb{R}\), \((\lambda f)(\mu x) = \lambda f(\mu x)= \mu\lambda f(x)\). Thus scaler multiplication is also closed.
        The other eight properties were checked in part \(a)\), thus, it is subspace.
    \end{itemize}
    \item [8.] \(\forall w \in \mathbb{R}^3\), \(w = \lambda_1e_1+\lambda_2e_2+\lambda_3e_3\) where \(e_i\) are basis in \(\mathbb{R}^3\). Thus \(\forall v\in\)Hom\((\mathbb{R}^3,\mathbb{R})\), 
    \(v =f(w) =  f(\lambda_1e_1+\lambda_2e_2+\lambda_3e_3)\) Since \(f(x)\) is linear, then \(f(\lambda_1e_1+\lambda_2e_2+\lambda_3e_3) = \lambda_1f(e_1)+\lambda_2f(e_2)+\lambda_3f(e_3)\).
    Consider 3 linear maps \(l_1,l_2,l_3\) in the prompt. For standard basis, \(e_1,e_2,e_3\), \(w =  \lambda_1e_1+\lambda_2e_2+\lambda_3e_3\)
    . \(l_1(w)=\lambda_1,\ l_2(w)=\lambda_2,\ l_3(w)=\lambda_3\), Thus, \(f(w) = l_1(w)f(e_1)+l_2(w)f(e_2)+l_3(w)f(e_3)\). 
    Thus, as \(f(e_i)\in\mathbb{R}\), \(l_1,l_2,l_3\) do span Hom\((\mathbb{R}^3,\mathbb{R})\).
    \\ Next, prof that the set is linearly independent.
    Suppose they are not, then exists \(\lambda_1,\lambda_2,\lambda_3\in F\) not all zero, while \(\lambda_1l_1+\lambda_2l_2+\lambda_3l_3=\overrightarrow{0}\) as \(\overrightarrow{0}= (f(w)=0),\ \forall w\in\mathbb{R}^3\).
    As \(w\) is a random vector is \(\mathbb{R}^3\), \(\lambda_1l_1+\lambda_2l_2+\lambda_3l_3 = \lambda_1x_1+\lambda_2x_2+\lambda_3x_3=0\). However, does not exist such combinations of set of \(\lambda_i\)
    Therefore, they are linear independent.\\
    Thus, the set of maps is a set of basis for Hom\((\mathbb{R}^3,\mathbb{R})\)











\end{itemize}
\end{document}