\documentclass{article}
\usepackage{graphicx}
\usepackage{amsmath}
\usepackage{array}
\usepackage[font=small, labelfont={sf,bf}, margin=1cm]{caption}
\usepackage{tabularx}
\usepackage{amssymb}



\date{Due: Dec 4 Edit: \today}
\title{MATH 416H HW 11}
\author{James Liu}

\begin{document}
\maketitle
Note that \(A^i_j,A_{ij}\) all means that the \(i\)th row and \(j\)th collumn of the matrix \(A\)
\begin{itemize}
    \item [1.] Consider the base: \(\left\{\frac{1}{0!}x^0,\frac{1}{1!}x^1,\frac{1}{2!}x^2\cdots,\frac{1}{n!}x^n\right\}\),\\
    Then, 
    \(A=[T]_{\mathcal{B} \mathcal{B} }=\begin{bmatrix}
        0&1&0&\cdots&0&0\\
        0&0&1&\cdots&0&0\\
        0&0&0&\cdots&0&0\\
        \vdots&\vdots&\vdots&\ddots&\vdots&\vdots\\
        0&0&0&\cdots&0&1\\
        0&0&0&\cdots&0&0\\
    \end{bmatrix}\)\\
    Which is in jordan normal form
    \item [2.] Notice that \(\begin{pmatrix}
        1&3\\3&1
    \end{pmatrix}\) is diagonalizable. Therefore, \(USDS^{-1}U^{-1}\) is a diagonal matrix. Then, diagonalize \(\begin{pmatrix}
        1&3\\3&1
    \end{pmatrix}\). There is \(\lambda_1=4,\lambda_2=-2\) and \(v_1=\left(s,-s\right)^T,v_2=\left(s,s\right)^T\) Then the equation becomes:
    \begin{align*}
        U\begin{bmatrix}
            s&s\\s&-s
        \end{bmatrix}
        &\begin{bmatrix}
            1&3\\3&1
        \end{bmatrix}
        \begin{bmatrix}
            \frac{1}{2s}&\frac{1}{2s}\\
            \frac{1}{2s}&-\frac{1}{2s}
        \end{bmatrix}
        U^*
    \end{align*}
    Consider the specific case where \(US=I,U^*S^{-1}=I\) while setting the imaginary part to zero, we have \(UU=I,US=I,US^{-1}=I\), then \(S=S^{-1}\), we have \(        \begin{bmatrix}
        \frac{1}{2s}&\frac{1}{2s}\\
        \frac{1}{2s}&-\frac{1}{2s}
    \end{bmatrix}=\begin{bmatrix}
        s&s\\s&-s
    \end{bmatrix}\), then \(s=\frac{\sqrt{2}}{2}\), then \( U=S^{-1}=\begin{bmatrix}
        \frac{\sqrt{2}}{2}&\frac{\sqrt{2}}{2}\\
        \frac{\sqrt{2}}{2}&-\frac{\sqrt{2}}{2}
    \end{bmatrix}=U^* \).
    \item [3.]
    \begin{itemize}
        \item [a)] Since the standard basis of \(\mathbb{C}^n\) is orthornomal, \(\langle e_i,e_j\rangle=\delta^i_j\) and \(\varphi e_i=b_i\), then
        We have \(\langle \varphi (e_i),\varphi (e_j)\rangle=\langle b_i,b_j\rangle = \delta^i_j = \langle e_i,e_j\rangle\). Thus the map is unitary.
        \item [b)]
        The transformation that sents \(\{e_1,e_2,\cdots,e_n\}\) into \(\mathcal{B}'\) is Unitary with similar prof done in a). And for a Unitary map we can write out a matrix, note \(\varphi=[\varphi]=\mathcal{T}\) that \([\varphi]=\mathcal{T}\) is the matrix representation.
        Then its inverse would be \(\mathcal{T}^*\) which is also unitary as:\\
         \(\mathcal{T}^*\left(\mathcal{T}^*\right)^*=\mathcal{T}^*\mathcal{T}=I\). Then consider \(\varphi'\circ \varphi\) (\(\varphi' = \mathcal{T}'\)):
        \begin{align*}
            \varphi'\circ \varphi &=\mathcal{T}'\mathcal{T}=T\\
            TT^*&=\mathcal{T}'\mathcal{T}(\mathcal{T}'\mathcal{T})^*\\
            &=\mathcal{T}'\mathcal{T}\mathcal{T}^*\mathcal{T}'^*\\
            &=I
        \end{align*}
        Therefore, \(\varphi'\circ\varphi\) is unitary. And such transformation transforms from \(\mathcal{B}\) to \(\mathcal{B}'\). Which is \([\text{id}_V]_{\mathcal{B,B'}}\)
    \end{itemize}
    \item [4.] \
    \begin{itemize}
        \item [Forward:] For any Hermitian matrix A that is positive definite, forall \(v\in V\) \(\langle v,Av \rangle>0\) then for all eigenvectors \(v_i\) There is:
        \begin{align*}
            Av_i&=\lambda_iv_i\\
            \langle v_i,Av_i \rangle&=\langle v_i,\lambda_iv_i \rangle\\
            &=\overline{\lambda_i}\langle v_i,v_i \rangle\\
            &=\overline{\lambda_i}||v_i||^2
        \end{align*}
        For any non-zero vector \(v_i\), \(||v_i||^2>0\), then for \(\langle v_i,Av_i \rangle>0\), \(\overline{\lambda_i}||v_i||^2>0\)  Thus \(\lambda_i>0\) also as it is a hermitian matrix, all of its eigenvalues are real.
        \item [Backward:] Assume tha \(\forall \lambda_i >0\). According to the spectual therome, \(\forall A,\ \exists U\) that is unitary that \(UAU^*=D\) which is the diagonal matrix with all of \(A\)'s eigenvalues on its main diagonal. Then we have:
        \begin{align*}
            \langle v,Av \rangle&=\langle Uv,UAv \rangle\\
            &=\langle UU^*v,UAU^*v \rangle\\
            &=\langle Iv,Dv \rangle
        \end{align*} 
        Consider a set of orthornomal basis of \(V\), \(\{b_1,\cdots,b_n\}\). \(v=\sum \mu_ib_i\), \(\mu_i\in \mathbb{C}\), then \(\displaystyle Dv=\sum_{i=1}^{n}\lambda_i\mu_ib_i\), Thus:
        \begin{align*}
            \langle v,Dv \rangle&= \sum_{i=1}^{n} \mu_i\overline{\lambda_i\mu_i}||b_i||^2\\
            &=\sum_{i=1}^{n}\lambda_i||\mu_i||^2||b_i||^2
        \end{align*}
        since \(||\mu_i||^2>0\), \(||b_i||^2>0\) and \(\forall \lambda_i>0\)(and real as it is from a hermitian matrix), \(\sum_{i=1}^{n}\lambda_i||\mu_i||^2||b_i||^2>0\) Therefore the matrix is positive definite.
    \end{itemize}
    \item [5.] Since \(A\) is hermitian, then \(\exists U,D\) that \(A=UDU^*\) where \(U\) is unitary and \(D\) is diagonal with entries equaling to eigenvalues. Therefore, as \(A\) is positve definite, all entries of \(D\) would be positve real numbers.\\
    then write the matrix as \(D_{ij}=\delta_{ij}\lambda_i\), then \(\exists D'\)  that \(D'_{ij}=\delta_{ij}\sqrt{\lambda_i}\), then \(D'D'=\Gamma\), and \(\Gamma_{ij}=\sum_{k=1}^{n} \delta_{ik}\sqrt{\lambda_{k}}\delta_{kj}\sqrt{\lambda_{k}}=\delta_{ij}\lambda_i=D\), Therefore:
    \(A=UD'D'U^*\) As \(D'\) only have real positve entries in its main diagonal, \(D'^*=D'\), then take \(S=D'U^*\), then \(S^*=(D'U^*)^*=UD'^*=UD'\), Thus, \(S^*S=UD'D'U^*=A\)
    \item [6.]
    \begin{itemize}
        \item [a)]\
        \begin{itemize}
            \item [Forward:]
            \begin{align*}
                A^*&=-A\\
                -A^*&=A\\
                -\sqrt{-1}A^*&=\sqrt{-1}A\\
                -iA^*&=iA\\
                (iA)^*&=-iA^*
            \end{align*}
            Therefore, if \(A\) is skew-Hermitian then \(A^*=-A\)
            \item [Backward:] If \(\sqrt{-1}A\) is Hermitian, then:
            \begin{align*}
                (iA)^*&=iA\\
                iA &=-iA^*\\
                A&=-A^*
            \end{align*}
        \end{itemize}
        \item [b)]For any Hermitian matrix \(H\), forall eigenvalues \(\lambda_i\) of \(H\), \(\lambda_i\in \mathbb{R}\). As \(iA\) is Hermitian, 
        \(\exists U,\ iA=U\Lambda U^*\) where \(\Lambda\) only have real diagonal entries. Then \(A=U(\frac{1}{i}\Lambda )U^*\). \(\forall r\in \mathbb{R}, \frac{r}{i}=-ri\) which is purely imaginary (if include r=0)
        therefore \(\left(\frac{1}{i}\Lambda\right)\) have only pure imaginary (including zero) diagonal entries.
        Also, as \(U\) is unitary and \(UU^*=I\), \(A=U\left(\frac{1}{i}\Lambda\right)U^{-1}\) which means that \(\left(\frac{1}{i}\Lambda\right)\) is the matrix containing all of its eigenvalues which means that they are all purely imaginary.
        \item [c)]Using lemma 31.4 profed in class, as \(iA\) is herimitan, then \(iA=U\Lambda U^*\) and \(A=U\left(\frac{1}{i}\Lambda\right)U^*\) which is unitary equavalent to the diagonal matrix \(-i\Lambda\) meaning that it includes all eigenvectors of A.
    \end{itemize}
    \item [7.]
    \begin{itemize}
        \item [a)]
        \begin{align*}
            [B,A]&=AB-BA\\
            &=-(BA-AB)\\
            &=-[A,B]
        \end{align*}
        \item [b)]
        \begin{align*}
            -[A,B]&=[B,A]\\
            &=BA-AB\\
            &=(-B)(-A)-(-A)(-B)\\
            &=B^*A^*-A^*B^*\\
            [A,B]^*&=\left(AB-BA\right)^*\\
            &=(AB)^*-(BA)^*\\
            &=B^*A^*-A^*B^*\\
            &=-[A,B]
        \end{align*}
        Therefore it is also skew-hermitian.
        \item [c)]
        \begin{align*}
            [A,B]&=AB-BA\\
            &=A^*B^*-B^*A^*\\
            &=A^*B^*-(AB)^*\\
            [A,B]^*&=B^*A^*-A^*B^*\\
            &=(AB)^*-A^*B^*\\
            &=-[A,B]
        \end{align*}
        Therefore, in general \([A,B]\) is not hermitian. And is when \((AB)^*=A^*B^*\)
    \end{itemize}
    \item [8.]
    \begin{itemize}
        \item [a)] Since A is Hermitian, note \([T]_{\mathcal{EE}}=A\), then \(A=U\Lambda U^*\) where \(U\in U(n)\) and thus \(U^*=U^{-1}\), since \(Av_i = \lambda_i\), As \(\Lambda\) is the diagonal matrix with eigenvalues entries. Then on eigenvector basis \(\mathcal{B}\), \(\Lambda v_i=\lambda_iv_i\). Thus, \([T]_{\mathcal{BB}}=\Lambda\).
        \(A=[\text{id}]_{\mathcal{EB}}[T]_{\mathcal{BB}}[\text{id}]_{\mathcal{BE}}=U\Lambda U^{-1}\) Therefore, \(U=[v_1,v_2,\cdots,v_n]\) (collumns would be the eigenvectors of \(A\)). Claim that 
        \(\sum v_jv_j^*=UU^*=I\). Note the \(k\)'s entry of \(v_i\) as \(v_i(k)\)
        \begin{align*}
            (UU^*)_{ij}&=\sum_{k=1}^{n}U^{i}_{k}U^{*k}_j\\
            &=\sum_{k=1}^{n}v_k(i)v^*_k(j)\\
            &=\sum_{j=1}^{n}v_jv_j^*
        \end{align*}
        \item [b)] As stated above, we have \(A=U\Lambda U^*\)
        \begin{align*}
            (\Lambda U^*)_{ij}&=\sum_{k=1}^{n}\Lambda^i_k U^{*k}_j\\
            &=\sum_{k=1}^{n}\delta^i_k\lambda_iU^{*k}_j\\
            \left(U(\Lambda U^*)\right)_{ij}&=\sum_{\ell=1}^{n} U^i_\ell(\Lambda U^*)^\ell_j\\
            &=\sum_{\ell=1}^{n}\left(U^i_\ell\sum_{k=1}^{n}\delta^\ell_k\lambda_\ell U^{*k}_j\right)\\
            &=\sum_{\ell=1}^{n}\left(U^i_\ell\lambda_\ell U^{*\ell}_j\right)\\
            &=A
        \end{align*}
        Therefore,  \(A=\sum_{j=1}^{n}\lambda_jv_jv_j^*\)
    \end{itemize}
\end{itemize}
\end{document}