\documentclass{article}
\usepackage{graphicx}
\usepackage{amsmath}
\usepackage{array}
\usepackage[font=small, labelfont={sf,bf}, margin=1cm]{caption}
\usepackage{tabularx}
\usepackage{amssymb}



\date{Due: Sep 11 Edit: \today}
\title{MATH 416H HW 2}
\author{James Liu}

\begin{document}
\maketitle

\begin{itemize}
    \item [1.] No it is no longer a vector space, as:
    \begin{align*}
        \exists \lambda,\mu \in \mathbb{R},\ (\lambda+\mu)\overrightarrow{v}&= ((\lambda+\mu)^2\times v_1,\cdots,(\lambda+\mu)^2\times v_n)^T
        \\ &=((\lambda^2+\mu^2+2\lambda\mu)\times v_1,\cdots,(\lambda^2+\mu^2+2\lambda\mu)\times v_n)^T
        \\ &\neq ((\lambda^2+\mu^2)\times v_1,\cdots,(\lambda^2+\mu^2)\times v_n)^T
        \\ &= \lambda \overrightarrow{v}+\mu\overrightarrow{v}
    \end{align*}
    Thus, as it is not a vector space.
    \item [2.] No it is not a vector space, as:
    \begin{align*}
        \exists \lambda,\mu \in \mathbb{R},\ (\lambda+\mu)\overrightarrow{v}&= a^{\lambda+\mu}\\
        & = a^\lambda\times a^\mu \neq a^\lambda+a^\mu\\
        & = \lambda\overrightarrow{v}+\mu\overrightarrow{v}
    \end{align*}
    For example: \(\lambda=2,\mu=1,a=2,\) as \(2^{(1+2)}=8\neq6=2^1+2^2\)
    Thus, it is not a vector space.
    \item [3.] No it is not in the span. Suppose it is then \(\exists a_1,a_2\in\mathbb{R}\) such that \(a_1(3, 4, 2)^T + a_2(1, 3, 3)^T =(-1,2,3)^T\)
    which is:
    \[\left\{ \begin{matrix}
        3a_1+a_2&=&-1\\
        4a_1+3a_2&=&2\\
        2a_1+3a_2&=&3
    \end{matrix} \right.\]
    Deriving it will result in:
    \begin{align}
        \left\{ \begin{matrix}
            3a_1+a_2&=&-1\\
            -5a_1+0a_2&=&5\\
            -7a_1+0a_2&=&6
        \end{matrix} \right.\rightarrow
        \left\{ \begin{matrix}
            a_1 = -1\\
            a_1 = -\frac{6}{7}
        \end{matrix} \right.
    \end{align}
    Which raises a contradiction and thus, such \(a_1,a_2\) does not exist thus, it is not in the span.
    \item[4.] as \(v\) is in span of the subspace, then:\(\exists {a_1,\cdot,a_n}, a_i\in\mathbb{R}\) that \\
    \(v = a_1\overrightarrow{v}_1+\cdots +a_n\overrightarrow{v}_n \)
    Thus, \(-1v+a_1\overrightarrow{v}_1+\cdots +a_n\overrightarrow{v}_n=0\) thus, the set is linear dependent.
    \newpage
    \item [5.] As \(U,W\) are subspace of \(V\), then \(\overrightarrow{0}\in U,W\) thus \(\overrightarrow{0}\in U\cap W\)
    Also, consider that \(x,y\in U\cap W\), then  \(x,y\in W\) and \(x,y\in U\), Thus, \(x+y\in U\) as \(U\) is a subspace, 
    similarly, \(x+y\in W\) therefore. \(x+y\in U\cap W\) similarly, \(\forall \lambda \in \mathbb{R}\), \(\lambda x \in U\), and \(\lambda x \in V\), therefore, the addition
    and multiplication are closed in \(U\cap W\) following \(V\) while the intersection contains the zero vector making it not empty, thus, the intersection is still a subspace.
    \item [6.] 
    \begin{itemize}
    \item [a)] As profed in class, any set of vector \(S\) that spans \(V\) will have a element number \(n\) that is larger or equal to any set that is linear independent.
    and for any vector space of dimension \(m\), it has a basis consisting of \(m\) elements. And any additional vector in the vector space will be lienar dependent with this base. Thus, basis is 
    the largest linear independent subsets in a vector space. Therefore, a subset of a vector space that spans the vector space must have elements that is larger or equal to the
    number of elements in a base. Thus
    \item [b)] As \(V\) have a dimension of \(n\), then exists a base \(\{v_1,\cdots,v_n\}\subseteq V\) and spans \(V\) with \(n\) linearly independent vectors \(v_1,\cdots,v_n\).
    Thus, for any other vector, \(\forall v\in V,\exists \lambda_1,\cdots,\lambda_n\in F\) that \(v=\lambda_1v_1+\cdots+\lambda_nv_n\)
    and \(\exists \mu_1,\cdots,\mu_n\in F\) that \(-v=\mu_1v_1+\cdots+\mu_nv_n\).\\ Thus, \(v+\mu_1v_1+\cdots+\mu_nv_n=0\) and it is linear dependent with the set of basis. Thus, the maximum numebr of elements is \(n\).

\end{itemize}
    \item [7.] 
    \item [8.] As \(v\in \text{span}  \ S\), \(\text{span } S = \text{span }S\cup \{v\}\)
    \item [9.] \(\forall \lambda_i,\mu_i,a,b\in F\):
    \begin{align*}
        F(a\lambda_1+b\mu_1,\cdots,a\lambda_n+b\mu_n)&=a\lambda_1v_1+b\mu_1v_1+\cdots+a\lambda_nv_n+b\mu_nv_n\\
        &=a\lambda_1v_1+\cdots+a\lambda_nv_n + b\mu_1v_1+b\mu_nv_n\\
        &=a(\lambda_1v_1+\cdots+\lambda_nv_n) + b(\mu_1v_1+\mu_nv_n)\\
        &=aF(\lambda_1,\cdots,\lambda_n) + bF(\mu_1,\cdots,\mu_n)
    \end{align*}
    Therefore it is a linear map
\end{itemize}
\end{document}