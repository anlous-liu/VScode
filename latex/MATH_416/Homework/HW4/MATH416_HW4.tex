\documentclass{article}
\usepackage{graphicx}
\usepackage{amsmath}
\usepackage{array}
\usepackage[font=small, labelfont={sf,bf}, margin=1cm]{caption}
\usepackage{tabularx}
\usepackage{amssymb}



\date{Due: Sep 25 Edit: \today}
\title{MATH 416H HW 4}
\author{James Liu}

\begin{document}
\maketitle

\begin{itemize}
\item [1.]It is \(\sum_{i=1}^{m} a_ix_i \)
\item [2.]It is $
\begin{pmatrix}
    yb_1\\yb_2\\\vdots\\yb_n
\end{pmatrix}$
\item [3.] \ 
\begin{itemize}
    \item [\(E_1A=\)]\(
         \begin{bmatrix}
        0 & 1 & 0 \\
        1 & 0 & 0 \\
        0 & 0 & 3
        \end{bmatrix}
        \begin{bmatrix}
        a_{11} & a_{12} & a_{13} & a_{14} \\
        a_{21} & a_{22} & a_{23} & a_{24} \\
        a_{31} & a_{32} & a_{33} & a_{34}
        \end{bmatrix}=
        \begin{bmatrix}
            a_{21} & a_{22} & a_{23} & a_{24} \\
            a_{11} & a_{12} & a_{13} & a_{14} \\
            3\times a_{31} & 3\times a_{32} & 3\times a_{33} & 3\times a_{34}
        \end{bmatrix}
        \)
    \item [\(E_2A=\)]\(
         \begin{bmatrix}
        0 & 0 & 1 \\
        0 & 1 & 0 \\
        1 & 0 & 0
        \end{bmatrix}
        \begin{bmatrix}
        a_{11} & a_{12} & a_{13} & a_{14} \\
        a_{21} & a_{22} & a_{23} & a_{24} \\
        a_{31} & a_{32} & a_{33} & a_{34}
        \end{bmatrix}=
        \begin{bmatrix}
            a_{31} & a_{32} & a_{33} & a_{34}\\
            a_{21} & a_{22} & a_{23} & a_{24} \\
            a_{11} & a_{12} & a_{13} & a_{14} \\
        \end{bmatrix}
        \)
    \item [\(E_3A=\)]\(
            \begin{bmatrix}
           1 & 0 & 0 \\
           0 & 1 & -5 \\
           0 & 0 & 1
           \end{bmatrix}
           \begin{bmatrix}
           a_{11} & a_{12} & a_{13} & a_{14} \\
           a_{21} & a_{22} & a_{23} & a_{24} \\
           a_{31} & a_{32} & a_{33} & a_{34}
           \end{bmatrix}=
           \begin{bmatrix}
            a_{11} & a_{12} & a_{13} & a_{14} \\
            a_{21}-5a_{31}& a_{22}-5a_{32}& a_{23}-5a_{33}& a_{24}-5a_{34}\\
            a_{31} & a_{32} & a_{33} & a_{34}
           \end{bmatrix}
           \)
\end{itemize}
\item [4.]
\begin{itemize}
    \item [a)]\(\iota(w)= w,\ \iota(v)=v,\ \iota(w)+\iota(v)=w+v=\iota(w+v)\)\\
                \(\lambda\iota(v)=\lambda\cdot v = \iota(\lambda v)\). Thus, the inclusion map is linear.
    \item [b)]
    \begin{itemize}
        \item [i:] \(\forall u\in U\), \(T|_U(u) = w = T(u)\), \(\iota_U(u) = u\). Thus \(T(\iota_U(u)) = T(u) = w = T|_U(u)\). Thus \(T|_U = T\circ \iota_U\)
        \\ As the composition of 2 linear maps is still linear, \(T|_U\) is linear.
    \end{itemize}
\end{itemize}
\newpage
\item [5.]
\begin{itemize}
    \item [a)] \
    \begin{itemize}
        \item [forward:] Suppose \(\exists w_1,w'_1\in W_1,\ w_2,w'_2,\in W_2\), that \(v = w_1+w_2=w'_1+w'_2\). Thus:
        \begin{align*}
            w_1-w'_1&=w'_2-w_2\\
            w_1-w'_1\in W_1\ &\  w'_2-w_2\in W_2\\
            \text{as }W_1\cap W_2 = \{\overrightarrow{0}\}\\
            w_1-w'_1=w'_2-w_2&=\overrightarrow{0}\\
            w_1=w'_1\ &\ w_2=w'_2
        \end{align*}
        Thus, if \(V=W_1\oplus W_2\), \(\forall v\in V\), exists a unique \(w_1,w_2\) that \(v = w_1+w_2\)
        \item [backward:] If \(\forall v \in V\), \(\exists w_1\in W_1,\ w_2\in W_2\) that \(v=w_1+w_2\), \(V = W_1+W_2\) by definition. Suppose that \(W_1\cap W_2 \neq \{\overrightarrow{0}\}\), 
        then \(\exists w\in W_1,W_2\). Thus, forsome \(v\in W_1\), or \(v = w_1+\overrightarrow{0}\),  Thus define \(k = w_1-w\), Therefore \(\exists v = w_1-w+w\), where \(w\neq \overrightarrow{0}\). However, there are only one set of \(w_1,w_2\) that \(v=w_1+w_2\), Therefroe, \(W_1\cap W_2 = \overrightarrow{0}\)
    \end{itemize}
    \item [b)] \textbf{Existence}:\\
                \(\forall v\in V\), exists a unique set \(w_1\in W_1,\ w_2\in W_2\), that \(v = w_1+w_2\) as profed in 5.a).
                Define:
                \begin{align*}
                    \forall v &= w_1+w_2\\
                    \iota_1(v) = w_1&,\ \iota_2(v) = w_2\\
                    \iota'_1:W_1\rightarrow V \ \iota_1' (w_1) = w_1+\overrightarrow{0}&, \ \iota'_2:W_2\rightarrow V \ \iota_2' (w_2) = w_2+\overrightarrow{0}
                \end{align*}
                \(\forall T_1,T_2\) define: \(T(v) = T_1\circ \iota_1(v)+T_2\circ \iota_2(v)\).\\
                \(\forall w_1 \in W_1\): \(T|_{W_1} = T_1(w_1)+T_2(\overrightarrow{0})\), as \(T_1,\ T_2\) are linear, there is \(T_2(\overrightarrow{0})=\overrightarrow{0}\). Thus, \(T|_{W_1} = T_1\) similarly, \(T|_{W_2}=T_2\). Thus, such map exists.\\
                \textbf{Linearity}:\\
                \begin{align*}
                    T(v_1+v_2) &= T(w_{11}+w_{12}+w_{21}+w_{22})\\
                                &= T_1(w_{11}+w_{21})+T_2(w_{12}+w_{22})\\
                                &=T_1(w_{11})+T_1(w_{21})+T_2(w_{12})+T_1(w_{22})\\
                                &=T_1(w_{11})+T_2(w_{12})+T_1(w_{21})+T_1(w_{22})\\
                                &=T(v_1)+T(v_2)\\
                    T(\lambda v_1)& = T(\lambda(w_1+w_2))\\
                                &= T(\lambda w_1+\lambda w_2)\\
                                &=T_1(\lambda w_1)+T_2(\lambda w_2)\\
                                &=\lambda (T_1( w_1)+T_2(w_2))\\
                                &= \lambda T(v_1) \text{ \ \ \ \ Thus, \(T\) is linear.}
                \end{align*}
                
                \textbf{Uniqueness}:\\
                Suppose it exists such \(T'\) that \(T'|_{W_1}=T_1,\ T'|_{W_2} = T_2,\ T'\neq T\). Then \(T'(v)\neq T(v)\). 
                However:
                \begin{align*}
                    T'(v) &= T'(w_1+w_2)\\
                            &= T'(w_1)+T'(w_2)\\
                            &= T'|_{W_1}(w_1)+T'|_{W_2}(w_2)\\
                            &= T_1(w_1)+T_2(w_1)\\
                            &= T(v)
                \end{align*}
                Thus, it is unique.
\end{itemize}
\item [6.]\(\forall v\in V, T(S(v)) = v\), Thus, \(T\) is Surjective, Thus, \(R(T)=dim(V)\), Thus \(N(T)=\{\overrightarrow{0}\}\)
Suppose \(T\) is not injective, then \(\exists v,w \in V\) that \(v\neq w\) that \(T(v)=T(w)\), then \(T(v)-T(w)=\overrightarrow{0}=T(v-w)\), which raises a contradiction. Thus, \(T\) is a bijection. Thus \(T\) is invertable.
As \(T\) is invertable, \(\exists T^{-1}\) that \(T\circ T^{-1} = \text{id}_V\). Therefore, \(T^{-1} = S\). Thus, \(S\circ T = T^{-1}\circ T = \text{id}_V\)
\item [7.] Take a random set of \(a_1,\cdots,a_k,\cdots\in \mathbb{N}, S(a_1,a_2,\cdots,a_k,\cdots)=(0,a_1,a_2,\cdots,a_k,\cdots)\),
            \(T(0,a_1,a_2,\cdots,a_k,\cdots)=(a_1,a_2,\cdots,a_k,\cdots)\). Thus, \(T\circ S = \text{id}_V\)
\item [8.] \(\forall x_i\in\mathbb{R},\ v=(x_1,\cdots,x_n) P(v) = (x_1,0,\cdots,0) \), and \(P(P(v)) = P(x_1,0,\cdots,0) = (x_1,0,\cdots,0) = P(v)\)
Thus, \(P\circ P = P\)\\
\(N(P) = (0,x_2,\cdots,x_n),\ R(P) = (x_1,0,\cdots,0)\), \(\mathbb{R}^n = N(P)+R(P) = (x_1,x_2,\cdots,x_n)\). Also, as \(R(P)\cap N(P) = (0,0,\cdots,0)=\overrightarrow{0}\). Thus, \(\mathbb R ^n = N(P)\oplus R(P)\)
\end{itemize}
\end{document}