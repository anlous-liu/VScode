\documentclass{article}
\usepackage{graphicx}
\usepackage{amsmath}
\usepackage{array}
\usepackage[font=small, labelfont={sf,bf}, margin=1cm]{caption}
\usepackage{tabularx}
\usepackage{amssymb}



\date{Due: Sep 4 Edit: \today}
\title{MATH 416H HW 1}
\author{James Liu}

\begin{document}
\maketitle

\begin{itemize}
    \item [\textbf{1.}] 
    \begin{itemize}
        \item [\textbf{i.}] True.\\
        Functions folowing  natural addition and multiplication is closed under the two operations while is also naturally follows the following 8 properties.
        \begin{itemize}
            \item [a.] + is communitive, \(\overrightarrow{x}+\overrightarrow{y}=\overrightarrow{y}+\overrightarrow{x}\)
            \item [b.] + is associative, \(\overrightarrow{x}+(\overrightarrow{y}+\overrightarrow{z})=(\overrightarrow{x}+\overrightarrow{y})+\overrightarrow{z}\)
            \item [c.] \(\exists \overrightarrow{0}\in V\) such that \(\forall \overrightarrow{v}\in V,\ \overrightarrow{0}+\overrightarrow{v}=\overrightarrow{v}\)
            \item [d.] \(\forall \overrightarrow{v}\in V, \exists -\overrightarrow{v}\)
            \item [e.] \(\forall v\in V, \ 1\times \overrightarrow{v}=\overrightarrow{v}\)
            \item [f.] \(\forall \lambda,\mu\in\mathbb{R},\ \lambda(\mu\cdot\overrightarrow{v})=(\lambda\mu)\cdot \overrightarrow{v}\)
            \item [g.] \(\forall \lambda\in\mathbb{R},\ \lambda(\overrightarrow{v}+\overrightarrow{w})=\lambda \overrightarrow{v}+\lambda\overrightarrow{w}\)
            \item [h.] \(\forall \lambda,\mu\in\mathbb{R},\ (\lambda+\mu)\overrightarrow{v}=\lambda\overrightarrow{v}+\mu\overrightarrow{v}\)
        \end{itemize}
        \item [\textbf{ii.}] False.\\
        Consider \(F(x)=1\), \(-1\cdot F(x)= -1\), it is not close under scaler multiplication.
        \item [\textbf{iii.}] False.\\
        Consider \(F_1(x) = x^2+x^3\) and \(F_2(x) = x^2-x^3\) as both of them are at degree 3. However, \(F_1+F_2=2x^2\) is nolonger degree 3 which means it is not close under addition and thus not a vector space.
    \end{itemize}
    \item [\textbf{2.}]
    Apply the Gaussian illimination:\\
\(
\begin{bmatrix}
    1&3&2&\bigm|&2\\
    1&6&1&\bigm|&3\\
    2&3&5&\bigm|&5
\end{bmatrix}
\rightarrow
\begin{bmatrix}
    1&3&2&\bigm|&2\\
    0&3&-1&\bigm|&1\\
    0&-3&1&\bigm|&1
\end{bmatrix}
\rightarrow
\begin{bmatrix}
    1&3&2&\bigm|&2\\
    0&3&-1&\bigm|&1\\
    0&0&0&\bigm|&2
\end{bmatrix}
\)
\\ \\
Therefore, there is no solution\\
    \newpage
    \item [\textbf{3.}]
    Yes, it is linear.\\
    Assume two different maps \(f(x)\) and \(g(x)\),
    \begin{align*}
        T(\lambda f+\mu g)  &=\int^2_1 \lambda f(x)+\mu g(x)\ dx\\
                            &=\int^2_1 \lambda f(x)\ dx+\int^2_1 \mu g(x)\ dx\\
                            &=\lambda \int^2_1 f(x)\ dx+ \mu\int^2_1  g(x)\ dx\\
                            &=\lambda T(f)+\mu T(g)
    \end{align*}
    Thus,  it is linear.\\
    \item [\textbf{4.}]
    \item []
    \begin{itemize}
        \item [\textbf{a.}]
        \(\begin{bmatrix}
            1&2&3\\4&5&6
        \end{bmatrix}
        \begin{bmatrix}
            1\\3\\2
        \end{bmatrix}
        =
        \begin{bmatrix}
            13\\31
        \end{bmatrix}
        \)
        \item [\textbf{b.}]
        \(
        \begin{bmatrix}
            1&2\\0&1\\2&0
        \end{bmatrix}
        \begin{bmatrix}
            1\\3
        \end{bmatrix}
        =
        \begin{bmatrix}
            7\\3\\2
        \end{bmatrix}
        \)
        \item [\textbf{c.}]
        \(
        \begin{bmatrix}
            1&2&0&0\\
            0&1&2&0\\
            0&0&1&2\\
            0&0&0&1
        \end{bmatrix}
        \begin{bmatrix}
            1\\2\\3\\4
        \end{bmatrix}
        =
        \begin{bmatrix}
            5\\8\\11\\4
        \end{bmatrix}
        \)
    \end{itemize}
    \newpage
    \item [\textbf{5.}]
    \item []
    \begin{itemize}
        \item [\textbf{a.}]
        Apply Gaussian illimination\\
        \(\begin{bmatrix}
            1&2&-1&2&\bigm|&3\\
            3&7&0&5&\bigm|&8\\
            -1&0&7&-2&\bigm|&-1
        \end{bmatrix}
        \rightarrow
        \begin{bmatrix}
            1&2&-1&2&\bigm|&3\\
            0&1&3&-1&\bigm|&-1\\
            0&2&6&0&\bigm|&2
        \end{bmatrix}
        \rightarrow\\
        \begin{bmatrix}
            1&0&-7&4&\bigm|&5\\
            0&1&3&-1&\bigm|&-1\\
            0&0&0&2&\bigm|&4
        \end{bmatrix}
        \rightarrow
        \begin{bmatrix}
            1&0&-7&0&\bigm|&-3\\
            0&1&3&0&\bigm|&1\\
            0&0&0&1&\bigm|&2
        \end{bmatrix}
        \)\\
        Therefore:\\
        \(\left\{\begin{matrix}
            x_1&&&-&7x_3&&=&-3\\
            &&x_2&+&3x_3&&=&1\\
            &&&&&x_4&=&2
        \end{matrix}\right.\)
        \\
        Thus:
        \(\left\{\begin{matrix}
            x_1 &=& -3+7s\\
            x_2 &=& 1-3s\\
            x_3 &=& s\\
            x_4 &=& 2
        \end{matrix}\right.\)
        \item [\textbf{b.}]
        Apply Gaussian illimination\\
        \(\begin{bmatrix}
            1&2&-1&2&\bigm|&a\\
            3&7&0&5&\bigm|&b\\
            -1&0&7&-2&\bigm|&c
        \end{bmatrix}\rightarrow
        \begin{bmatrix}
            1&2&-1&2&\bigm|&a\\
            0&1&3&-1&\bigm|&b-3a\\
            0&2&6&0&\bigm|&c+a
        \end{bmatrix}
        \rightarrow\\
        \begin{bmatrix}
            1&0&-7&4&\bigm|&7a-2b\\
            0&1&3&-1&\bigm|&b-3a\\
            0&0&0&2&\bigm|&c-2b+7a
        \end{bmatrix}
        \rightarrow
        \begin{bmatrix}
            1&0&-7&0&\bigm|&-7a+2b-2c\\
            0&1&3&0&\bigm|&\frac{1}{2}c+\frac{1}{2}a\\
            0&0&0&1&\bigm|&\frac{1}{2}c-1b+\frac{7}{2}a
        \end{bmatrix}
        \)
        Thus:\\
        \(\left\{\begin{matrix}
            x_1 &=& -7a+2b-2c+7s\\
            x_2 &=& \frac{1}{2}c+\frac{1}{2}a-3s\\
            x_3 &=& s\\
            x_4 &=& \frac{1}{2}c-1b+\frac{7}{2}a
        \end{matrix}\right.\)\\
        Therefore, changing the right hand side will still give a solution.
        
    \end{itemize}
    \item [\textbf{6.}] quiz question\\
    Let \(V\) be a vector space containing all \(C^0\)(contiuous) functions. Try prof that \(F(\overrightarrow{x})=3\overrightarrow{x}+1\) is not linear for \(x\in V\)
\end{itemize}










\end{document}