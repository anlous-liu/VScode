\documentclass{article}
\usepackage{graphicx}
\usepackage{amsmath}
\usepackage{array}
\usepackage[font=small, labelfont={sf,bf}, margin=1cm]{caption}
\usepackage{tabularx}
\usepackage{amssymb}



\date{Due: Sep 4 Edit: \today}
\title{MATH 416H HW 1}
\author{James Liu}

\begin{document}
\maketitle

\begin{itemize}
    \item [1.]
    \item [2.]
    Apply the Gaussian illimination:\\
\(
\begin{bmatrix}
    1&3&2&\bigm|&2\\
    1&6&1&\bigm|&3\\
    2&3&5&\bigm|&5
\end{bmatrix}
\rightarrow
\begin{bmatrix}
    1&3&2&\bigm|&2\\
    0&3&-1&\bigm|&1\\
    0&-3&1&\bigm|&1
\end{bmatrix}
\rightarrow
\begin{bmatrix}
    1&3&2&\bigm|&2\\
    0&3&-1&\bigm|&1\\
    0&0&0&\bigm|&2
\end{bmatrix}
\)
\\ \\
Therefore, there is no solution\\
    \item [3.]
    Yes, it is linear.\\
    Assume two different maps \(f(x)\) and \(g(x)\),
    \begin{align*}
        T(\lambda f+\mu g)  &=\int^2_1 \lambda f(x)+\mu g(x)\ dx\\
                            &=\int^2_1 \lambda f(x)\ dx+\int^2_1 \mu g(x)\ dx\\
                            &=\lambda \int^2_1 f(x)\ dx+ \mu\int^2_1  g(x)\ dx\\
                            &=\lambda T(f)+\mu T(g)
    \end{align*}
    Thus,  it is linear.\\
    \item [4.]
    \item []
    \begin{itemize}
        \item [a.]
        \(\begin{bmatrix}
            1&2&3\\4&5&6
        \end{bmatrix}
        \begin{bmatrix}
            1\\3\\2
        \end{bmatrix}
        =
        \begin{bmatrix}
            13\\31
        \end{bmatrix}
        \)
        \item [b.]
        \(
        \begin{bmatrix}
            1&2\\0&1\\2&0
        \end{bmatrix}
        \begin{bmatrix}
            1\\3
        \end{bmatrix}
        =
        \begin{bmatrix}
            7\\3\\2
        \end{bmatrix}
        \)
        \item [c.]
        \(
        \begin{bmatrix}
            1&2&0&0\\
            0&1&2&0\\
            0&0&1&2\\
            0&0&0&1
        \end{bmatrix}
        \begin{bmatrix}
            1\\2\\3\\4
        \end{bmatrix}
        =
        \begin{bmatrix}
            5\\8\\11\\4
        \end{bmatrix}
        \)
    \end{itemize}
    \newpage
    \item [5.]
    \item []
    \begin{itemize}
        \item [a.]
        Apply Gaussian illimination\\
        \(\begin{bmatrix}
            1&2&-1&2&\bigm|&3\\
            3&7&0&5&\bigm|&8\\
            -1&0&7&-2&\bigm|&-1
        \end{bmatrix}
        \rightarrow
        \begin{bmatrix}
            1&2&-1&2&\bigm|&3\\
            0&6&3&-1&\bigm|&-1\\
            0&2&6&0&\bigm|&2
        \end{bmatrix}
        \rightarrow\\
        \begin{bmatrix}
            1&2&-1&2&\bigm|&3\\
            0&2&6&0&\bigm|&2\\
            0&6&3&-1&\bigm|&-1
        \end{bmatrix}
        \rightarrow
        \begin{bmatrix}
            1&2&-1&2&\bigm|&3\\
            0&2&6&0&\bigm|&2\\
            0&0&-15&-1&\bigm|&-7
        \end{bmatrix}
        \)\\
        Therefore:\\
        \(\left\{\begin{matrix}
            x_1&+&2x_2&-&x_3&+&2x_4&=3\\
            &&2x_2&+&6x_3&&&=2\\
            &&&-&15x_3&-&x_4&=-7
        \end{matrix}\right.\)
    \end{itemize}
\end{itemize}










\end{document}