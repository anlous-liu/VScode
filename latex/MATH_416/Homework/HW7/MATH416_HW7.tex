\documentclass{article}
\usepackage{graphicx}
\usepackage{amsmath}
\usepackage{array}
\usepackage[font=small, labelfont={sf,bf}, margin=1cm]{caption}
\usepackage{tabularx}
\usepackage{amssymb}



\date{Due: Oct 23 Edit: \today}
\title{MATH 416H HW 7}
\author{James Liu}

\begin{document}
\maketitle
\begin{itemize}
    \item [1.] First, as U is a vector space, a set of basis can be write out. \(\{u_1,\cdots,u_n\}\), 
    also, since \(U\) is also a subset of \(V\). We can extend the basis into a basis of V: 
    \(\{u_1,\cdots,u_n,z_1,\cdots,z_m\}\). Now consider the dual space \(V^*\) by theromes profed in class, 
    a basis of \(V^*\) is \(\{u_1^*,\cdots,u_n^*,z_1^*,\cdots,z_m^*\}\) and, 
    by definition, \(\{z_1^*,\cdots,z_m^*\}\subseteq U^0\) as \(\forall u \in U\), \(u=\sum_{i=1}^n\lambda_iu_i+\sum_{j=1}^{m}0\times z_i\). 
    Thus, \(\forall u\in U\), \(z_j^*(u)=0\). Since \(U^0\) us a subset of \(V\), then a basis of \(U^0\) must also be a subset of \(V\)'s.
    \\
    \\
    Suppose \(\{z_1,\cdots,z_m\}\) is not a basis for \(U^0\), then \(\exists k\) that for some \(u^{*0}\in U^0\), \(u^*0=\lambda u_k^*+\mu_1z_1^*+\cdots+\mu_mz_m^*\) However, consider \(u_k\in U\) (the one in basis). 
    \(u^{*0}(u_k)=\lambda u_k^*(u_k)+\mu_1z_1^*(u_k)+\cdots+\mu_mz_m^*(u_k)=\lambda +0+\cdots+0\neq 0\). Thus, there is a contradiction, therefore, \(\{z_1^*,\cdots ,z_m^*\}\) is a basis of \(U^0\).
    \begin{align*}
        \dim(V)=m+n&&\dim(U)=n&&\dim(U^0)=m
    \end{align*}
    Thus, \(\dim(U^0)+\dim(U)=\dim(V)\).
    \item [2.]
    \begin{itemize}
        \item [a)]\(\forall w^*\in W,\ T^*(w^*) = w^*\circ T(v)\). \(\forall w^*\in N(T^*)\), \(T^*(w^*)=\overrightarrow{0}=v^{*0},\\ v^{*0}\in V^0\), Thus, \(N(T^*)=\{w^*|\forall v\in V, w^*(T(v))=0\}\)
        \\ or \(\{w^*|w^*\in W^*,\ \forall w \in R(T), w^*(w)=0\}\).\\ \\ 
        \((R(T))^0=\{w^*|w^*(w)=0,\ w^*\in W^*, \ \forall w\in R(T)\}\). Therefore, \(N(T^*)=(R(T))^0\)
        \item [b)]
        \begin{align*}
            N(T^*)&=(R(T))^0&(\text{by part a })\\
            \dim(N(T^*))&=\dim(W)-\dim(R(T))&(\text{by problem 1})\\
            \dim(W^*)-\dim(R(T^*))&=\dim(W)-\dim(R(T))&(\text{by rank nullity})\\
            \because \dim(W)&=\dim(W^*)\\
            \therefore \dim(R(T^*))&=\dim(R(T))
        \end{align*}
        \item [c)]
        Any \(m\times n\) matrix \(A\) would be equavelent to \(T:V\rightarrow W\) where \(\dim(V)=n,\ \dim(W)=m\), and \(A^T\) would be equavelent with \(T^*:W^*\rightarrow V^*\). As profed by b), \(R(T^*)=R(T)\). Thus, they do have same rank.
    \end{itemize}
    \item [3.] By 2.a) if \(T^*\) is injective then \(N(T^*)={\overrightarrow{0}}=(R(T))^0\). Then by 1., we have \(\dim(R(T))=\dim(W^*)-\dim((R(T))^0)=\dim(W^*)=\dim(W)\). as \(R(T)\subseteq W\), therefore, \(R(T)=W\) Thus, the transformation is surjective.
    \item [4.] \begin{align*}
        (\lambda b)(v_1+v_2,\gamma w)&= \lambda b(v_1+v_2,\gamma w)\\
        &=\lambda\gamma b(v_1,w)+\lambda\gamma b(v_2,w)\\
        &=(\lambda\gamma b)(v_1,w)+(\lambda \gamma b)(v_2,w)&(\text{Close under scaler multi.})\\
    (b_1+b_2)(v_1+v_2,\gamma w)&=b_1(v_1+v_2,\gamma w)+b_2(v_1+v_2,\gamma w)\\
    &=b_1(v_1,w)+b_1(v_2,w)+b_2(v_1,w)+b_2(v_2,w)\\
    &=b_1(v_1,w)+b_2(v_1,w)+b_1(v_2,w)+b_2(v_2,w)\\
    &=(b_1+b_2)(v_1,w)+(b_1+b_2)(v_2,w)&(\text{Close under vector add})
    \end{align*}
    \begin{itemize}
        \item [a)] 
        \begin{align*}
            (b_1+b_2)(v,w)&=b_1(v,w)+b_2(v+w)\\
            &=b_2(v,w)+b_1(v+w)\\
            &=(b_2+b_1)(v,w)&(\text{vector ddition is communitive})
        \end{align*}
        \item [b)]
        \begin{align*}
            ((b_1+b_2)+b_3)(v,w)&=(b_1+b_2)(v,w)+b_3(v,w)\\
            &=b_1(v,w)+b_2(v+w)+b_3(v,w)\\
            &=b_1(v,w)+(b_2+b_3)(v,w)\\
            &=(b_1+(b_2+b_3))(v,w) &(\text{vector addition is associative})
        \end{align*}
        \item [c)] Consider the bilinear map \(b_0(v,w)=0\)
        \begin{align*}
            (b+b_0)(v,w) &=b(v,w)+b_0(v,w)\\
            &=b(v,w)+0\\
            &=b(v,w)&(\exists \overrightarrow{0} \text{ such that }b+\overrightarrow{0}=b)
        \end{align*}
        \item [d)]
        \begin{align*}
            (b+(-1\cdot b))(v,w)&=b(v,w)+(-1)\times b(v,w)\\
            &=0\\
            &=b_0& (\forall b,\ \exists-b,\ b+(-b)=0)
        \end{align*}
        \item [e)]
        \begin{align*}
            (1\cdot b)(v,w)&=1\times b(v,w)\\
            &=b(v,w)
        \end{align*}
        \newpage
        \item [f)]
        \begin{align*}
            \lambda((\mu \cdot b)(v,w))&=\lambda(\mu b(v,w))\\
            &=\lambda\mu b(v,w)\\
            &=\mu\lambda b(v,w)\\
            &=\mu(( \lambda\cdot b)(v,w))
        \end{align*}
        \item [g)]
        \begin{align*}
            \lambda(b_1+b_2)(v,w)&=\lambda(b_1(v,w)+b_2(v,w))\\
            &=\lambda b_1(v,w)+\lambda b_2(v,w)\\
            &=(\lambda b_1)(v,w)+(\lambda b_2)(v,w)
        \end{align*}
        \item [h)]
        \begin{align*}
            ((\lambda+\mu)b)(v,w)&=(\lambda + \mu)b(v,w)\\
            &=\lambda b(v,w)+\mu b(v,w)\\
            &=(\lambda b)(v,w)+(\mu b)(v,w)
        \end{align*}
    \end{itemize}
    \item [5.]
    \begin{itemize}
        \item [a)]
        \begin{align*}
            w_0^\#(v_1+v_2) &= b(v_1+v_2,w_0)\\
            &= b(v_1,w_0)+b(v_2,w_0)\\
            &= w_0^\#(v_1)+w_0^\#(v_2)\\
            w_0^\#(\lambda v)&= b(\lambda v,w_0)\\
            &= \lambda b(v,w_0)\\
            &=\lambda w_0^\#(v)
        \end{align*}
        Thus it is linear.
        \item [b)]
        Mark the transformation by \(T\) instead of \(\#\).\\
        \begin{align*}
            T(\lambda w_0)&=b(v,\lambda w_0)\\
            &=\lambda b(v,w_0)\\
            &=\lambda T(w_0)\\
            T(w_1+w_2)&=b(v,w_1+w_2)\\
            &=b(v,w_1)+b(v,w_2)\\
            &=T(w_1)+T(w_2)
        \end{align*}
        Thus, it is linear.
        \item [c)]
        \(\forall w\in N(\#)\), \(\#(w)=0\) which is \(\forall v\in V\), \(w^\#(v)=0\) which is \(\forall v \in V\), \(b(v,w)=0\). Therefore, it is correct.
        \item [d)]
        As proved in c) \(N(\#)=\{w_0\in W|b(w^*,w_0)=0 \text{ for all } w^*\in W^*\}\). For any \(w_0\) we can write a bsis of \(W\) containing \(w_0\). Thus, the dual vector space would also have a basis containing \(w_0^*\).
        Suppose it exists some \(w_0\neq 0\) that \(\forall w^*\in W^*\), \(b(w^*,w_0)=0\), consider \(w_0^*\in W^*\), \(b(w_0^*,w_0)=w_0^*(w_0)=1\neq 0\). Thus there is a contradiction. Therefore, \(N(\#)=\{0\}\). Thus, the map is injective.
    \end{itemize}
    \item [6.]
    \begin{itemize}
        \item [a)] \[\sigma = (1,2)(3,1)(4,5)\]
        \item [b)] Since there are 3 swaps, the sign is \(-1\)
    \end{itemize}
    \item [7.] It shall equal to \(1+1+1=3\)
    \item [8.] 
\end{itemize}
\end{document}