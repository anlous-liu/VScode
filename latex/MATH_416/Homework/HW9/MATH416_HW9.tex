\documentclass{article}
\usepackage{graphicx}
\usepackage{amsmath}
\usepackage{array}
\usepackage[font=small, labelfont={sf,bf}, margin=1cm]{caption}
\usepackage{tabularx}
\usepackage{amssymb}



\date{Due: Nov 13 Edit: \today}
\title{MATH 416H HW 9}
\author{James Liu}

\begin{document}
\maketitle
\begin{itemize}
    \item [1.] Note the basis as \(\{b_1,b_2\}\), as \(b_1,b_2\in V\), \(b_{11}+b_{12}+b_{13}=0\) similar to \(b_2\). consider such vector \(m=\begin{pmatrix}
        1+0i\\
        0+0i\\
        -1+0i
    \end{pmatrix}\), then, the orthogonal basis would have a property like: \(\langle m\cdot n\rangle =0\) which is:
    \begin{align*}
        1\times \bar{n}_1+0\times \bar{n}_2-1\bar{n}_3&=0\\
        \bar{n}_1-\bar{n_3}&=0
    \end{align*}
    Therefore, \(m=\begin{pmatrix}
        1+0i\\
        0+0i\\
        -1+0i
    \end{pmatrix}, n=\begin{pmatrix}
        1+0i\\
        -2+0i\\
        1+0i
    \end{pmatrix}\) is a set of orthogonal vectors.
    Claim that it is a basis.\\
    Notice that \(\forall v\in V, \exists v = (v_1+\frac{1}{2}v_2)m-\frac{1}{2}v_2n\), therefore, it is a basis.
    \item [2.] \(T(x_1,x_2,\cdots,x_n) = \sum_{i,j=1}^{n}\delta _i^jx_{ij}\), \(\). 
    \begin{align*}
        T(a+b,x_2,\cdots,x_n)&=(a_{11}+b_{11})+\sum_{i,j=2}^{n}\delta _i^jx_{ij}\\
        &\neq a_{11}+\sum_{i,j=2}^{n}\delta _i^jx_{ij}+b_{11}+\sum_{i,j=2}^{n}\delta _i^jx_{ij}\\
        &=(a_{11}+b_{11})+2\sum_{i,j=2}^{n}\delta _i^jx_{ij}\\
        &=T(a,x_2,\cdots,x_n)+T(b,x_2,\cdots,x_n)\\
        &\neq T(a+b,x_2,\cdots,x_n)\\
    \end{align*}
    Therefore, it is not n-linear.\\
    \item [3.] Consider \(v,w\in W\)
    \begin{align*}
        T(v+w)&=T(v)+T(w)\\
        &=0+0\\
        &\text{ Therefore it is close under addition }\\
        T(\lambda v) &= \lambda T(v)\\
        &=0\\
        &\text{ Therefore it is close under scaler multiplication }\\
        v+\overrightarrow{0}&=v\\
        &\text{ Thus ther exists a zero vector }\\
    \end{align*}
    \item [4.]
    \begin{itemize}
        \item [a)]
        \begin{align*}
            \langle x,y\rangle  &=\langle \sum\alpha_iv_i,\sum\beta_jv_j\rangle \\
            &=\sum \alpha_i \langle  v_i,\sum \beta_j v_j\rangle \\
            &=\sum_{i=1}^{n} \alpha_i \overline{\langle \sum_{j=1}^n\beta_jv_j,v_i\rangle }\\
            &=\sum_{i=1}^n \left(\alpha_i \sum_{j=1}^n \overline{\beta_j}\overline{\langle v_j,v_i\rangle }\right)\\
            &=\sum_{i=1}^n \left(\alpha_i \sum_{j=1}^n \overline{\beta_j}\langle v_i,v_j\rangle \right)\\
            &=\sum_{i=1}^{n} \left(\alpha_i\sum_{j=1}^{n} \overline{\beta_j}\delta_i^j\right)\\
            &=\sum \alpha_k\beta_k
        \end{align*}
        \item [b)]
        \begin{align*}
            \langle x,v_k\rangle &=\sum_{i=1}^{n} \delta_{i}^k\alpha_k \times 1\\
            &=\alpha_k\\
            \overline{\langle y,v_k\rangle }&=\sum_{i=1}^{n} \delta_{i}^k\overline{\beta}_k \times 1\\
            &=\overline{\beta}_k\\
            \sum\langle x,v_k\rangle \overline{\langle y,v_k\rangle }&=\sum \alpha_k\overline{\beta_k}\\
            &=\langle x,y\rangle 
        \end{align*}
    \end{itemize}
    \item[5.] It is not a real inner product. If it is a inner product, then \(\langle A,A\rangle = \overrightarrow{0}\) then \(A=\overrightarrow{0}\). However, consider \(
    A = \begin{pmatrix}
        0&1\\
        0&0
    \end{pmatrix}
    \), 
    \(AA = \overrightarrow{0}\), \(\text{Tr}(A,A)= 0+0\) which is 0 while \(A\) is not \(\overrightarrow{0}\). Thus, it is not a real inner product 
        \begin{align*}
        \end{align*}
    \item [6.]\
    \begin{itemize}
        \item [\(||u+v||^2\)]
        \begin{align*}
            ||u+v||^2&= \langle u+v,u+v\rangle\\
            &=\langle u,u+v\rangle+\langle v,u+v\rangle\\
            &=\overline{\langle u+v,u \rangle}+\overline{\langle u+v,v \rangle}\\
            &=\overline{\langle u,u \rangle}+\overline{\langle v,u \rangle}+\overline{\langle u,v \rangle}+\overline{\langle v,v \rangle}\\
            &=||u||^2+\langle u,v \rangle +\overline{\langle u,v \rangle}+ ||v||^2\\
            &=2^2 + (2+i)+(2-i)+3^2\\
            &=17
        \end{align*}
        \item [\(||u-v||^2\)]
        \begin{align*}
            ||u-v||^2&= \langle u-v,u-v\rangle\\
            &=\langle u,u-v\rangle-\langle v,u-v\rangle\\
            &=\overline{\langle u-v,u\rangle}-\overline{\langle u-v,v\rangle}\\
            &=\overline{\langle u,u\rangle}-\overline{\langle v,u\rangle}-\left(\overline{\langle u,v\rangle}-\overline{\langle v,v\rangle}\right)\\
            &=||u||^2-\langle u,v \rangle - \overline{\langle u,v\rangle}+ ||v||^2\\
            &=4-(2+i)-(2-i)+3^2\\
            &=9
        \end{align*}
        \item [\(\langle u+iv,v+iu \rangle\)]
        \begin{align*}
            \langle u+iv,v+iu\rangle &=\langle u,v\rangle +\langle u,iu\rangle +\langle iv,v\rangle +\langle iv,iu\rangle \\
            &=\langle u,v\rangle -i\langle u,u\rangle +i\langle v,v\rangle +\langle v,u\rangle \\
            &=(2+i)-i\cdot 4+i\cdot 9+(2-i)\\
            &=4+5i
        \end{align*}
    \end{itemize}
    \item [7.]
    \begin{align*}
        \overline{\left\langle\sum \beta_jb_j, \sum \alpha_ib_i\right\rangle}&=\overline{\sum \beta_k\overline{\alpha_k}}\\
        &=\sum \alpha_k\overline{\beta_k}\\
        &=\left\langle\sum \alpha_ib_i,\sum \beta_jb_j\right\rangle\\
        \left\langle \sum \lambda\alpha_ib_i+\sum\mu\gamma_lb_l,\sum \beta_jb_j\right\rangle &=\left\langle \sum\lambda \alpha_ib_i+\mu\gamma_ib_i,\sum \beta_jb_j\right\rangle\\
        &=\sum (\lambda\alpha_i+\mu\gamma_i)\overline{\beta_i}\\
        &=\lambda\sum \alpha_i \overline{\beta_i}+\mu\sum \gamma_i\overline{\beta_i}\\
        &=\lambda \left\langle \sum \alpha_ib_i,\sum \beta_jb_j\right\rangle+\mu \left\langle \sum\gamma_ib_i,\sum \beta_jb_j\right\rangle\\
        \left\langle \sum \alpha_ib_i,\sum \alpha_jb_j\right\rangle &=\sum \alpha_k\overline{\alpha}_k\\
        &=\sum ||\alpha_k||^2 \geq0\\
        \text{if }\left\langle \sum \alpha_ib_i,\sum \alpha_jb_j\right\rangle &= 0\\
        \text{then }\sum ||\alpha_k||^2 &= 0\\
        \text{since} ||a_k||^2\geq 0, ||\alpha_k||^2&=0\\
        \text{if } v&=0\\
        \text{then } \alpha_i &=0\\
        \text{Thus } \langle v,v\rangle&=0
    \end{align*}
    Therefore it is a Hermitian inner product
    \item[8.]
    \begin{align*}
        &\frac{1}{4}\left(||x+y||^2-||x-y||^2-i||x-iy||^2+i||x+iy||^2\right) \\
        &=\frac{1}{4}\left(\langle x+y,x+y\rangle-\langle x-y,x-y\rangle-i\langle x-iy,x-iy\rangle+i\langle x+iy,x+iy\rangle\right)\\
    \end{align*}
    Calculate each part seperatly:
    \begin{align*}
        \langle x+y,x+y\rangle &= \langle x,x\rangle+\langle x,y\rangle+\langle y,x \rangle+\langle y,y \rangle\\
        \langle x-y,x-y\rangle &= \langle x,x\rangle+\langle x,-y\rangle+\langle -y,x\rangle+\langle -y,-y\rangle\\
        -\langle x-y,x-y\rangle &= -\langle x,x\rangle+\langle x,y\rangle+\langle y,x\rangle-\langle y,y\rangle\\
        \langle x+y,x+y\rangle -\langle x-y,x-y\rangle &=2\langle x,y\rangle+2\langle y,x\rangle\\
        \langle x-iy,x-iy\rangle &= \langle x,x\rangle+\langle x,-iy\rangle + \langle -iy,x\rangle+\langle -iy,-iy\rangle\\
        &=\langle x,x\rangle+i\langle x,y\rangle -i \langle y,x\rangle+(-i\times i)\langle y,y\rangle\\
        &=\langle x,x\rangle+i\langle x,y\rangle -i \langle y,x\rangle+\langle y,y\rangle\\
        -i\langle x-iy,x-iy\rangle&=-i\langle x,x\rangle+\langle x,y\rangle - \langle y,x\rangle-i\langle y,y\rangle\\
        \langle x+iy,x+iy\rangle&= \langle x,x\rangle+\langle x,iy\rangle + \langle iy,x\rangle+\langle iy,iy\rangle\\
        &=\langle x,x\rangle-i\langle x,y\rangle+i\langle y,x\rangle+(i\times -i)\langle y,y\rangle\\
        &=\langle x,x\rangle-i\langle x,y\rangle+i\langle y,x\rangle+\langle y,y\rangle\\
        +i\langle x+iy,x+iy\rangle&=i\langle x,x\rangle+\langle x,y\rangle-\langle y,x\rangle+i\langle y,y\rangle\\
        -i\langle x-iy,x-iy\rangle+i\langle x+iy,x+iy\rangle&=2\langle x,y\rangle-2\langle y,x\rangle\\
        \frac{1}{4}\left(||x+y||^2-||x-y||^2-i||x-iy||^2+i||x+iy||^2\right)&=\frac{1}{4}(2\langle x,y\rangle+2\langle y,x\rangle+2\langle x,y\rangle-2\langle y,x\rangle)\\
        &= \langle x,y \rangle
    \end{align*}
    
\end{itemize}
\end{document}