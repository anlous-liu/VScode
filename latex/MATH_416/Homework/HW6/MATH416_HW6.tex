\documentclass{article}
\usepackage{graphicx}
\usepackage{amsmath}
\usepackage{array}
\usepackage[font=small, labelfont={sf,bf}, margin=1cm]{caption}
\usepackage{tabularx}
\usepackage{amssymb}



\date{Due: Oct 10 Edit: \today}
\title{MATH 416H HW 6}
\author{James Liu}

\begin{document}
\maketitle
\begin{itemize}
    \item [1.]
    \begin{itemize}
        \item [a)]
        Rank would be 3 and Nullity would be 2 as the matrix is already in reduced row-echlon form and the number of pivots is the rank and the number of none pivot column is Nullity.
        \item [b)]
        \begin{align*}
            \begin{bmatrix}
                1&2&0&1&0&\bigm|&0\\
                0&0&1&1&0&\bigm|&0\\
                0&0&0&0&1&\bigm|&0
            \end{bmatrix}
            \rightarrow
            \left\{\begin{matrix}
                x_1 = -2x_2-x_4\\
                x_2 = x_2\\
                x_3 = -x_4\\
                x_4 = x_4\\
                x_5 = 0
            \end{matrix}\right.
        \end{align*}
        Take \(x_2=1,x_4=1\), seperatly, we have a basis consisting 2 element: \(\displaystyle \left\{\begin{pmatrix}
        -2\\1\\0\\0\\0
        \end{pmatrix},\begin{pmatrix}
            -1\\0\\-1\\1\\0
        \end{pmatrix}\right\}\)
    \end{itemize}
    \item [2.]
    \begin{itemize}
        \item [a)]Scaler Multiplication: multiplying a scaler does not change the symetric of the matrix.\\
        \begin{align*}
            A&=A^T\\
            a_{ij}&=A_{ji} \quad 0\leq ij\leq n\\
            ka_{ij}&=ka_{ji} \quad k\in F\\
        \end{align*}
        Vector Addition: Adding two such matrix also does not change such symetry:
        \begin{align*}
            A&=A^T &B&=B^T\\
            a_{ij}&=a_{ji}&b_{ij}&=b_{ji}\\
            a_{ij}+b_{ij}&=a_{ji}+{b_{ji}}
        \end{align*}
        Consider \(a_{ij}=0,\ \forall i,j\), such matrix will be the additive identity.
        And these operations do fullfill the 8 properties as in the question, \(M_{n,n}(F)\) is already a vector space. And \(S_n\) is close under the 2 operations, thus it is a subspace.
        \item [b)] Notice that one possible set of basis would be:
        \begin{align*}
            \left\{
            \begin{pmatrix}
                0&1&0\\
                1&0&0\\
                0&0&0
            \end{pmatrix}
            ,
            \begin{pmatrix}
                0&0&1\\
                0&0&0\\
                1&0&0
            \end{pmatrix}
            ,
            \begin{pmatrix}
                0&0&0\\
                0&0&1\\
                0&1&0
            \end{pmatrix},
            \begin{pmatrix}
                1&0&0\\
                0&0&0\\
                0&0&0
            \end{pmatrix},
            \begin{pmatrix}
                0&0&0\\
                0&1&0\\
                0&0&0
            \end{pmatrix},
            \begin{pmatrix}
                0&0&0\\
                0&0&0\\
                0&0&1
            \end{pmatrix}
            \right\}
        \end{align*}
    \end{itemize}
    \item [3.] \(T\) is injective, then\(\forall w\in W,\ \exists v\in V\) such that \(w=T(v)\). By definition, 
    \(T^*(\psi) =\psi\circ T (v) \), \(\psi\in W^*\). Consider \(T^*(\psi)(v) = 0\)
    \begin{align*}
        T^*(\psi)(v) &= 0\\
        \psi(T(v))&=0 \ \forall v\in V\\
    \end{align*}
    As \(T(v)\) is surjective on \(W\), which means that \(\psi(w)=0,\ \forall w\in W\). Thus, \(\psi\) is a zero map, thus, \(N(T^*) = \{\overrightarrow{0}\}\). Thus, by rank/nullity, the \(T^*\) is injective.
    \item [4.]\(\forall \ell\in U^0\), \(\ell_1(x)+\ell_2(x) = 0+0\). Thus, \(\exists \ell_3\) such that \(\ell_1+\ell_2=\ell_3\). Thus \(U^0\) is closed under addition.
              \(\forall \lambda\in F\), \(\lambda \ell(x)=\lambda\times 0 = 0\). Thus, it is also closed under scaler multiplication. Also, \(\ell+\ell=\ell\) as \(0+0=0\), thus, there also exists a zero element. Thus, \(U^0\) is a subspace.
    \item [5.]Consider a map: \(\pi:V\rightarrow V/U\), \(\pi(v)=v+U\). Thus, \(\forall u\in U\), \(\exists v\) that \(u=v+U\) by definition. Thus, \(\pi\) is surjective. Thus, according to 3., the dual map \(\pi^*:(V/U)^*\rightarrow V^*\) is injective. 
    Thus, \(N(\pi^*)={\overrightarrow{0}}\). In this case, profed by 4., \(\overrightarrow{0}=U^0=\{\ell\}\). Thus, accroding to the 1st isomorphism law,
    \({(V/U)^*}/{N(\pi^*)}\rightarrow R(\pi^*)\) is isomorphic.
\end{itemize}
\end{document}