\documentclass{article}
\usepackage{graphicx}
\usepackage{amsmath}
\usepackage{array}
\usepackage[font=small, labelfont={sf,bf}, margin=1cm]{caption}
\usepackage{tabularx}
\usepackage{amssymb}



\date{Edit: \today, Class: Aug 28}
\title{MATH 416H Lecture 2 Note}
\author{James Liu}

\begin{document}
\maketitle
\section*{Vector Space}
\subsection*{Def}
\begin{itemize}
    \item [1.] Close under these 2 calcualtions
    \begin{itemize}
        \item [a)]
        \begin{align}
            \mathbb{R}\times V&\longrightarrow V \text{  Scaler Multiplication}\\
            (\lambda,\overrightarrow{v})&\longmapsto \lambda \overrightarrow{v}
        \end{align}
        \item [b)]
        \begin{align}
            V\times V &\longrightarrow V \text{ Vector Addition}\\
            (\overrightarrow{v_1},\overrightarrow{v_2})&\longmapsto \overrightarrow{v_1}+\overrightarrow{v_2}
        \end{align}
    \end{itemize}
    \item [2.] Holds following 8 property
    \begin{itemize}
        \item [a.] + is communitive, \(\overrightarrow{x}+\overrightarrow{y}=\overrightarrow{y}+\overrightarrow{x}\)
        \item [b.] + is associative, \(\overrightarrow{x}+(\overrightarrow{y}+\overrightarrow{z})=(\overrightarrow{x}+\overrightarrow{y})+\overrightarrow{z}\)
        \item [c.] \(\exists \overrightarrow{0}\in V\) such that \(\forall \overrightarrow{v}\in V,\ \overrightarrow{0}+\overrightarrow{v}=\overrightarrow{v}\)
        \item [d.] \(\forall \overrightarrow{v}\in V, \exists -\overrightarrow{v}\)
        \item [e.] \(\forall v\in V, \ 1\times \overrightarrow{v}=\overrightarrow{v}\)
        \item [f.] \(\forall \lambda,\mu\in\mathbb{R},\ \lambda(\mu\cdot\overrightarrow{v})=(\lambda\mu)\cdot \overrightarrow{v}\)
        \item [g.] \(\forall \lambda\in\mathbb{R},\ \lambda(\overrightarrow{v}+\overrightarrow{w})=\lambda \overrightarrow{v}+\lambda\overrightarrow{w}\)
        \item [h.] \(\forall \lambda,\mu\in\mathbb{R},\ (\lambda+\mu)\overrightarrow{v}=\lambda\overrightarrow{v}+\mu\overrightarrow{v}\)
    \end{itemize}
\end{itemize}
Any set that have all properties above is called a vector space and the elements are called vectors.
\subsection*{Examples}
\begin{itemize}
    \item [1.] \(\forall n \geq 0,\ \mathbb{R}^n\) is a vector space, note that \(\mathbb{R}^0=\{0\}\)
    \item [2.] \(\mathbb{R}[x]\) is a set of all polynomials having one variable \(x\in\mathbb{R}\), such as: \(\mathbb{R}[x]=\{a_0+a_1x_1+\cdots+a_nx_n|^{n\geq0}_{a_i\in\mathbb{R}}\}\). Note it is a infinitly large verctor space.
    \item [3.] \(C[0,1]\equiv C^0[0,1]=\{f:[0,1]\rightarrow\mathbb{R}|f\text{ continuous}\}\) where \(C^0\) means the function is continuous.
    \item [4.] WRONG EXAMPLE: \(v\in \left\{ \begin{pmatrix}x_1\\x_2\end{pmatrix}\in\mathbb{R}^2\left|x_1,x_2\geq0\right.\right\}\) which is the 1 quadrant with axises, while it is not close under scaler Multiplication.
\end{itemize}
\section*{Properties}
\subsection*{lemma 2.1}
For any vector space \(V\) there is a unique \(\overrightarrow{0}\)
\subsubsection*{Prof}
if \(\exists \overrightarrow{0}'\neq \overrightarrow{0}\) then:
\begin{align*}
    \overrightarrow{0}' &=\overrightarrow{0}+\overrightarrow{0}' =\overrightarrow{0}
\end{align*}
\subsection*{lemma 2.2}
If \(V\) is a vector space, then \(\forall \overrightarrow{v} \in V, \ 0\cdot v = \overrightarrow{0}\)
\subsubsection*{Prof}
\begin{align*}
    0\cdot \overrightarrow{v}&=(0+0)\cdot \overrightarrow{v}=0\cdot  \overrightarrow{v}+0\cdot \overrightarrow{v}\\
    \forall \overrightarrow{w}&\in V,\ \exists -\overrightarrow{w} \text{ that } \overrightarrow{w}-\overrightarrow{w}=0\\
    0\cdot  \overrightarrow{v}&-0\cdot  \overrightarrow{v}=0\cdot  \overrightarrow{v}+0\cdot \overrightarrow{v}-0\cdot  \overrightarrow{v}\\
    &\overrightarrow{0}=0\cdot\overrightarrow{v}
\end{align*}
\newpage
\subsection*{lemma 2.3}
For any vector space \(V\)
\begin{itemize}
    \item [1.] Additive inverse is unique
    \item [2.] \(-1\cdot\overrightarrow{v}=-\overrightarrow{v}\)
\end{itemize}
\subsubsection*{Prof}
\begin{itemize}
    \item [1.] Suppose \(\overrightarrow{u}+\overrightarrow{v}=0\)
    \begin{align*}
        -\overrightarrow{v}&=\overrightarrow{0}+(-\overrightarrow{v})=(\overrightarrow{u}+\overrightarrow{v})-\overrightarrow{v}\\
        &=\overrightarrow{u}+\overrightarrow{0}
    \end{align*}
    \item [2.] 
    \begin{align*}
        -1\cdot \overrightarrow{v} +\overrightarrow{v}&=-1\cdot\overrightarrow{v}+1\cdot\overrightarrow{v}\\
        &=(1-1)\cdot \overrightarrow{v}\\
        &=\overrightarrow{0}
    \end{align*}
\end{itemize}
\section*{Subspace}
\subsection*{Definition}
Let \(V\) be a real(\(\mathbb{R}\)) vector space, a nonempty subset \(W\) of \(V\) is a subsapce of V if \(W\) is also a vector space with definition of the 2 operations same with \(V\).
\subsection*{Properties}
\begin{itemize}
    \item [1.] \(\overrightarrow{0}\in W\)
    \item [2.] \(\forall \overrightarrow{w} \in W, -\overrightarrow{w}\in W\)
\end{itemize}
(Both of the properties are esay to prof)
\section*{Linear Maps between Vector Space}
\subsubsection*{Definition}
Define a map \(T:V\longrightarrow W\), if \(\forall \overrightarrow{v_1},\overrightarrow{v_2}\in V,\ \lambda_1,\lambda_2\in\mathbb{R}\), if 
\\\(T(\lambda_1\overrightarrow{v}_1+\lambda_2\overrightarrow{v}_2)=\lambda_1T(\overrightarrow{v_1})+\lambda_2T(\overrightarrow{v_2})\)\\
Then \(T\) is a linear map between vector spaces.


\end{document}