\documentclass{article}
\usepackage{graphicx}
\usepackage{amsmath}
\usepackage{array}
\usepackage[font=small, labelfont={sf,bf}, margin=1cm]{caption}
\usepackage{tabularx}
\usepackage{amssymb}



\date{Edit: \today, Class: Aug 30}
\title{MATH 416H Lecture 3 Note}
\author{James Liu}

\begin{document}
\maketitle
\(F=\mathbb{C}\) or \(F=\mathbb{R}\) distinguish by context.
\section{Linear maps}
\(A=\begin{bmatrix}
    a_{11}&a_{12}&a_{13}&\cdots&a_{1n}\\
    a_{21}&a_{22}&a_{23}&\cdots&a_{2n}\\
    a_{31}&a_{32}&a_{33}&\cdots&a_{3n}\\
    \vdots&\vdots&\vdots&\ddots&\vdots\\
    a_{m1}&a_{m2}&a_{m3}&\cdots&a_{mn}\\
\end{bmatrix}\)
\\
Claim that for a linear map \(L_A:F^n\rightarrow F^m\)  \(\exists A \) that \(L_A(\overrightarrow{x})=A\overrightarrow{x}\)
prof is similar with the ones in L1 note.
\section{Vector Space}
\subsection{Linear Combinations}
Let \(V\) be a Vector space,  \(\{\overrightarrow{v_1},\overrightarrow{v_2},\cdots,\overrightarrow{v_k}\}\subseteq V\)
A linear combination of \(v_1,\cdots,v_k\) is a vector of the form \(\overrightarrow{u}=\lambda_1\overrightarrow{v_1}+\lambda_2\overrightarrow{v_2}+\lambda_3\overrightarrow{v_3}+\cdots+\lambda_k\overrightarrow{v_k}\)
for some \(\lambda_i \in F\)
\subsubsection{Example}
In \(\mathbb{R}^2\), \(\overrightarrow{u}=\binom{1}{3}\) is a linear combination from \(\overrightarrow{v_1}=\binom{1}{1}\) and \(\overrightarrow{v_2}=\binom{0}{1}\),
as \(\overrightarrow{u}=1\times \overrightarrow{v_1}+2\times \overrightarrow{v_2}\)
\subsection{Span}
A set of vectors \(\{\overrightarrow{v_1},\overrightarrow{v_2},\cdots,\overrightarrow{v_k}\}\subseteq V\)
in a vector space \(V\) span \(V\) if \(\forall \overrightarrow{u}\in V\), \(\exists\lambda_1\cdots\lambda_k\in F\) that 
\(\overrightarrow{u}=\lambda_1\overrightarrow{v_1}+\lambda_2\overrightarrow{v_2}+\lambda_3\overrightarrow{v_3}+\cdots+\lambda_k\overrightarrow{v_k}\)
\subsubsection{Example}
in \(\mathbb{R}^2\), \(\binom{1}{0},\binom{1}{1}\) spans the vector space of \(\mathbb{R}^2\) while \(\binom{1}{0},\binom{2}{0}\) does not.
\subsection{Linear dependency}
A set of vectors \(\{\overrightarrow{v_1},\overrightarrow{v_2},\cdots,\overrightarrow{v_k}\}\subseteq V\) in vector space \(V\) is \textbf{linearly dependent} if
\(\exists \lambda_1,\lambda_2,\lambda_3,\cdots,\lambda_k\in F\) not all being zeros while \(\lambda_1\overrightarrow{v_1}+\lambda_2\overrightarrow{v_2}+\cdots+\lambda_k\overrightarrow{v_k}=\overrightarrow{0}\)
\\
And if a set of vectors are not linearly dependent then it is linear independent.
\subsection{Basis}
Let \(V\) be a vector space over \(F\), A set of vectors \(\{\overrightarrow{v_1},\overrightarrow{v_2},\cdots,\overrightarrow{v_k}\}\subseteq V\)
is a basis if it is linearly independent and spans \(V\).

\subsubsection{Definition}
Suppose \(V\) is a vector space, \(\{\overrightarrow{v_1},\overrightarrow{v_2},\cdots,\overrightarrow{v_k}\}\subseteq V\).
\(\forall v\in V\), \(\exists\{\lambda_1,\lambda_2,\lambda_2,\cdots,\lambda_k\}, \\ \lambda_i\in F\)
that \(v=\lambda_1\overrightarrow{v_1}+\lambda_2\overrightarrow{v_2}+\lambda_3\overrightarrow{v_3}+\cdots+\lambda_k\overrightarrow{v_k}\)
then the set of vectors is a set of basis of \(V\)
\subsection{Examples}
\begin{itemize}
\item [1.] NO-EXAMPLE\\
\(\displaystyle\left\{\binom{1}{0},\binom{0}{1},\binom{1}{1}\right\}\) is not a set of basis for \(\mathbb{R}^2\)


\item [2.]


\end{itemize}

\end{document}