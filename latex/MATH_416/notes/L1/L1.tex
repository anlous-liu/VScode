\documentclass{article}
\usepackage{graphicx}
\usepackage{amsmath}
\usepackage{array}
\usepackage[font=small, labelfont={sf,bf}, margin=1cm]{caption}
\usepackage{tabularx}
\usepackage{amssymb}



\date{Edit: \today, Class: Aug 26}
\title{MATH 416H Lecture 1 Note}
\author{James Liu}

\begin{document}
\maketitle

\subsection*{Matrix}
\subsubsection*{Definition}
A \(m\times n\) Matrix \(A\) is an array of numbers (integers, rationals, real, complex)
\subsection*{Linearity}
We say a function (map) is linear when: 
\[F(\lambda \overrightarrow{x} + \mu \overrightarrow{y} ) = \lambda F(\overrightarrow{x})+\mu F(\overrightarrow{y})\]
Note that linear in calculus is not nessesarily linear in linear algebra!
\\e.g. 
\begin{align*}
    f(x)&=2x+3\\
    f(x+y)&=2(x+y)+3\\
    f(x)+f(y)&=2x+3+2y+3 \neq 2(x+y)+3
\end{align*}
\subsection*{examples}
Try prof that if \(\exists f(\overrightarrow{x}):\mathbb{R}^n \rightarrow \mathbb{R}^m\) is linear, then \(\exists A\) that \(A\) is a matrix and \(f(\overrightarrow{x})=A\times\overrightarrow{x}\).
\\ \\
Say \(\overrightarrow{x}\) is a \(n\) dimensional vector, then \(\overrightarrow{x}=\sum_{1}^{n} \overrightarrow{e_i}\times x_i\) where \(e_i\) are base vectors like:

\(\begin{bmatrix}
    1\\0\\0\\0\\ \vdots \\ 0
\end{bmatrix} 
\),
\(\begin{bmatrix}
    0\\1\\0\\0\\ \vdots \\ 0
\end{bmatrix} 
\)\dots\(\begin{bmatrix}
    0\\0\\0\\0\\ \vdots \\ 1
\end{bmatrix} 
\)
and \(x_i\) are respect value at the direction of base vectors
therefore, as \(f(\overrightarrow{x})\) is linear, there is:
\begin{align*}
    f(\overrightarrow{x})&=f\left(\sum_1^n \overrightarrow{e_i}x_i\right)\\
    &=\sum_1^n x_i f(\overrightarrow{e_i})
\end{align*}
thus, 
\begin{align*}
    A = 
    \begin{bmatrix}
        f(\overrightarrow{e_1})&f(\overrightarrow{e_2})&\cdots&f(\overrightarrow{e_n})
    \end{bmatrix}
\end{align*}
therefore such matrix exist.
\end{document}