\documentclass{article}
\usepackage{graphicx}
\usepackage{amsmath}
\usepackage{array}
\usepackage[font=small, labelfont={sf,bf}, margin=1cm]{caption}
\usepackage{tabularx}
\usepackage{amssymb}
\usepackage{multirow}



\date{Due:Nov 14 Edit: \today}
\title{PHYS 225 HW 10}
\author{James Liu}

\begin{document}
\maketitle
\begin{itemize}
    \item [1.]
    \begin{itemize}
        \item [a)]
        \begin{align*}
            V&=\left(\gamma,\gamma v_x,\gamma v_y, \gamma v_z\right)
        \end{align*}
        For \(V\cdot V\) in rest frame, considering \(c=1\), we have \(v_x=v_y=v_z=0\), then:
        \begin{align*}
            V\cdot V &=\gamma^2-(0+0+0)\\
            &=\frac{1}{1-0}=1
        \end{align*} 
        For \(V\cdot V\) in the frame where the particle only moves in the x direction, \(v_x=|v|,v_y=v_z=0\), Then:
        \begin{align*}
            V\cdot V&=\gamma^2-(\gamma^2v_x^2+0+0)\\
            &=\frac{1}{1-v^2}-\frac{v^2}{1-v^2}\\
            &=\frac{1-v^2}{1-v^2}=1
        \end{align*}
        Therefore it is indeed in variant.
        \item [b)]
        \begin{align*}
            A&=\left(\gamma^4va,\gamma^4vav_x+\gamma^2a_x,\gamma^4vav_y+\gamma^2a_y,\gamma^4vav_z+\gamma^2a_z\right)
        \end{align*}
        First consider a rest frame with \(v=0\) and \(a=a_x\), then:
        \begin{align*}
            A\cdot A &=\gamma^8v^2a^2-(\gamma^4va\mathbf{v}+\gamma^2\mathbf{a})\cdot (\gamma^4va\mathbf{v}+\gamma^2\mathbf{a})\\
            &=0-(\gamma^4a_x^2)\\
            &=-a^2_x
        \end{align*}
        Now, consider a frame where \(v'=v'_x\), now calculate the respect acceleration, \(a'_x =\frac{a_x}{\gamma^3}\), now:
        \begin{align*}
            A\cdot A &=\gamma^8v^2a^2-(\gamma^4va\mathbf{v}+\gamma^2\mathbf{a})\cdot (\gamma^4va\mathbf{v}+\gamma^2\mathbf{a})\\
            &=\gamma^8v'^2\left(\frac{a}{\gamma^3}\right)^2-\left(\frac{a_x}{\gamma}\left(v^2+1\right)\right)^2\\
            &=\gamma^2v_x^2a_x^2-\left(\frac{a^2(\gamma^2v^2+1)^2}{\gamma^2}\right)\\
            &=\gamma^2v^2a^2-\left(\frac{a^2\left(\frac{1+v^2}{1-v^2}\right)^2}{\gamma^2}\right)\\
            &=\frac{\gamma^4v^2a^2-a^2\left(\frac{1+v^2}{1-v^2}\right)^2}{\gamma^2}\\
            &=a^2\frac{\gamma^4v^2-\left(\frac{1+v^2}{1-v^2}\right)^2}{\gamma^2}\\
            &=a^2\frac{\left(\gamma^2v-\left(\frac{1+v^2}{1-v^2}\right)\right)\left(\gamma^2v+\left(\frac{1+v^2}{1-v^2}\right)\right)}{\gamma^2}\\
            &=-a^2(1+v^2+v^4)
        \end{align*}
        which is a bit different from the one in rest fram.
        \item [c)]
        \begin{align*}
            F&=m\left(\gamma^4va_F,\gamma^4a_x,\gamma^4a_y,\gamma^4a_z\right)\\
            V&=\left(\gamma,\gamma v_x,\gamma v_y, \gamma v_z\right)
        \end{align*}
        in the rest frame, with \(v_z=\frac{1}{2}\), \(F=(0,f_x,0,0)\), \(V=(0,0,0,1)\), \(V\cdot F=f_x\), Now, transform everything into a moving frame by multipling \(\Lambda \), We have,
        \(V\cdot F=\gamma^2f_x\) which is also different.
        \item [d)] in rest frame, whith \(\mathbf{F}=f_x\), then \(F\cdot dx=f_xdx\), and in a transforming frame, \(F'dx'=\gamma^2f_xdx\)
    \end{itemize}
    \item [2.]
    \begin{itemize}
        \item [a)]
        \begin{align*}
            E_{k,ball}&=3m\\
            E_{ball}&=3mc^2+mc^2=4mc^2\\
            E_{ball}^2&=p^2c^2+m^2c^4\\
            16m^2c^4 &=p^2c^2+m^2c^4\\
            p^2c^2+m^2c^4-16m^2c^4&=0\\
            p^2&=\frac{16m^2c^4-m^2c^4}{c^2}\\
            p_{ball}&=\sqrt{15}mc\\
            p'_{x,ball}&=p\cdot\cos(\theta)=\left(\sqrt{15}mc\right)\cos(\theta)\\
            p'_{y,ball}&=p\cdot\sin(\theta)=\left(\sqrt{15}mc\right)\sin(\theta)
        \end{align*}
        Transform momentum into \(S\) frame.
        \begin{align*}
            \Lambda &=\begin{bmatrix}
                \gamma&\gamma\beta&0&0\\
                \gamma\beta&\gamma&0&0\\
                0&0&1&0\\
                0&0&0&1\\
            \end{bmatrix}\\
            \mathbf{P}c&=\Lambda P'\\
            &=\begin{bmatrix}
                \gamma&\gamma\beta&0&0\\
                \gamma\beta&\gamma&0&0\\
                0&0&1&0\\
                0&0&0&1\\
            \end{bmatrix}
            \begin{bmatrix}
                16m^2c^4\\
                \left(\sqrt{15}mc\right)\cos(\theta)c\\
                \left(\sqrt{15}mc\right)\sin(\theta)c\\
                0
            \end{bmatrix}\\
            &=\begin{bmatrix}
                \gamma\left(4mc^2+\beta\left(\sqrt{15}mc^2\right)\cos(\theta)\right)\\
                \gamma\left(\beta4mc^2+\left(\sqrt{15}mc^2\right)\cos(\theta)\right)\\
                \left(\sqrt{15}mc^2\right)\sin(\theta)\\
                0
            \end{bmatrix}\\
            p_{x,ball}&=\gamma\left(\beta4mc+\left(\sqrt{15}mc\right)\cos(\theta)\right)\\
            p_{y,ball}&=\left(\sqrt{15}mc\right)\sin(\theta)\\
        \end{align*}
        Now get the velocity of the ball in the rest frame or \(S\) frame.\\
        \begin{align*}
            p_{x,ball}&=\gamma mv\\
            \frac{p}{m}&=\frac{v}{\sqrt{1-\frac{v^2}{c^2}}}\\
            \frac{p^2}{m^2}&=\frac{v^2}{1-\frac{v^2}{c^2}}\\
            v^2&=\frac{\left(\frac{p}{m}\right)^2}{1+\left(\frac{p}{mc}\right)^2}\\
            v_{x,ball}&=\frac{\gamma\left(4\beta c+\left(\sqrt{15}c\right)\cos(\theta)\right)}{\sqrt{1+\gamma\left(4\beta+\left(\sqrt{15}\right)\cos(\theta)\right)}}\\
            v_{y,ball}&=\frac{\left(\sqrt{15}c\right)\sin(\theta)}{\sqrt{1+\left(\sqrt{15}\right)\sin(\theta)}}
        \end{align*}
        for some \(t\), the snow ball hits \((0,0)\), thus:
        \begin{align*}
            v_xt&=4\\
            v_yt&=1\\
            \frac{v_x}{v_y}&=4\\
            \frac{\frac{\gamma\left(4\beta c+\left(\sqrt{15}c\right)\cos(\theta)\right)}{\sqrt{1+\gamma\left(4\beta+\left(\sqrt{15}\right)\cos(\theta)\right)}}}{\frac{\left(\sqrt{15}c\right)\sin(\theta)}{\sqrt{1+\left(\sqrt{15}\right)\sin(\theta)}}}&=4\\
            \frac{\gamma\left(4\beta c+\left(\sqrt{15}c\right)\cos(\theta)\right)}{\sqrt{1+\gamma\left(4\beta+\left(\sqrt{15}\right)\cos(\theta)\right)}}\times \frac{\sqrt{1+\left(\sqrt{15}\right)\sin(\theta)}}{\left(\sqrt{15}c\right)\sin(\theta)}&=4\\
            \gamma=\frac{1}{\sqrt{1-\beta^2}}&=\frac{1}{\sqrt{0.51}}\\
            \beta&=0.7
        \end{align*}
        plug back into the equation, we get: 
        \(\cos(\theta)=0.97998 \text{ or } -0.047908\)
        \item [b)]
        \begin{align*}
            p&=\sqrt{p_x^2+p_y^2}\\
            p^2&=\left(\gamma\beta4mc+\left(\sqrt{15}mc\right)\cos(\theta)\right)^2+\left(15m^2c^2\right)\sin^2(\theta)\\
            p&=\sqrt{m^2c^2(30+8\beta\gamma(\sqrt{15}\cos(\theta)+2))}\\
            v&=\frac{pc^2}{E}\\
            &=\frac{pc^2}{4mc^2}\\
            &=\frac{\sqrt{30+8\beta\gamma(\sqrt{15}\cos(\theta)+2)}}{4c}\\
            t&=L/v\\
            &=\frac{\sqrt{16+1}c}{\frac{\sqrt{30+8\beta\gamma(\sqrt{15}\cos(\theta)+2)}}{4c}}\\
            &=\frac{4\sqrt{17}c^2}{\sqrt{30+8\beta\gamma(\sqrt{15}\cos(\theta)+2)}}\\
            &=1.7\times 10^{17} \text{ or } 2.2319 \times 10^{17} \textsl{ s}
        \end{align*}
        \item [c)]
        \begin{align*}
            \tau&=\frac{t}{\gamma_{ball}}=\frac{tp}{mv}\\
            &=6.12\times 10^{-34} s
        \end{align*}
        \item [d)]
        \begin{align*}
            p' &= \sqrt{15}mc\\
            p &=\sqrt{m^2c^2(30+8\beta\gamma(\sqrt{15}\cos(\theta)+2))}\\
        \end{align*}
    \end{itemize}
    \item [3.]
    \begin{itemize}
        \item [a)]
        \begin{align*}
            \frac{f_{\text{obs}}}{f_{\text{emit}}}&=\sqrt{\frac{1+\beta}{1-\beta}}\\
            \frac{550}{700}&=\sqrt{\frac{1+\beta}{1-\beta}}\\
            \beta &=-\frac{12}{37}\\
            v&=\frac{12}{37}c=9.72\times 10^7\text{ m/s}
        \end{align*}
        
        \item [b)]
        \begin{align*}
            v&=4\times 10^4 \text{km/hr}\\
            &=4\times 10^4 \times 10^3 \div60^2 \text{ m/s}\\
            &=1.11\times 10^4 \text{m/s}\\
            &=3.707\times 10^{-5} c\\
            \beta &=3.707\times 10^{-5}\\
            \frac{f_{obs}}{f_{emit}}&=\sqrt{\frac{1+\beta}{1-\beta}}\\
            f_{obs}&=700.026 \text{ nm}
        \end{align*}
        A taylor expansion would be appropriate due to the significantly small value of \(\beta\).
    \end{itemize}
\end{itemize}
\end{document}