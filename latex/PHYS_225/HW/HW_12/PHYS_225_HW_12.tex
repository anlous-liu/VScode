\documentclass{article}
\usepackage{graphicx}
\usepackage{amsmath}
\usepackage{array}
\usepackage[font=small, labelfont={sf,bf}, margin=1cm]{caption}
\usepackage{tabularx}
\usepackage{amssymb}
\usepackage{multirow}



\date{Due:Dec 5 Edit: \today}
\title{PHYS 225 HW 12}
\author{James Liu}

\begin{document}
\maketitle
\begin{itemize}
    \item [1.] 
    \begin{itemize}
        \item [a)]
        \begin{align*}
            \nabla\cdot \left(\nabla\times A\right)&= \partial_i\cdot(\epsilon^{ijk}\partial_j A_k)\\
            &= \epsilon^{ijk}\partial_i\cdot(\partial_j A_k)\\
            &=\partial_i\partial_jA_k-\partial_i\partial_kA_j-\partial_j\partial_iA_k+\partial_j\partial_kA_i+\partial_k\partial_jA_i-\partial_k\partial_iA_j\\
        \end{align*}
        \item [b)] If \(\epsilon_{abc}\epsilon^{ijk}=-1\) then the total number of transpositions is also odd as if the total number is even, then it is either 2 odd which gives \(-1\times (-1)=1\)  or 2 even thich is \(1\times 1=1\) and none of then equals \(-1\).
        similar for \(\epsilon_{abc}\epsilon^{ijk}=1\), the total transpositions will be even. Therefore, we can add the same number of transpositions to \(ijk\) and \(abc\) which will not change the oddity of the sum and permute \(ijk\) back to \(ijk\) then the set of transpositions 
        affacting the sign would be \(abc\) therefore, we just need to permute it and set it to right sign and look up whether it is with \(ijk\). which is what the right side of the
        euqality did.
        \item [c)] 
        \begin{align*}
            &\delta^{i}_{a}\delta^{j}_{b}\delta^{k}_{c}+\delta^{i}_{b}\delta^{j}_{c}\delta^{k}_{a}+\delta^{i}_{c}\delta^{j}_{a}\delta^{k}_{b}-\delta^{i}_{b}\delta^{j}_{a}\delta^{k}_{c}-\delta^{i}_{a}\delta^{j}_{c}\delta^{k}_{b}-\delta^{i}_{c}\delta^{j}_{b}\delta^{k}_{a}\\
            =&\delta^{i}_{i}\delta^{j}_{b}\delta^{k}_{c}+\delta^{i}_{b}\delta^{j}_{c}\delta^{k}_{i}+\delta^{i}_{c}\delta^{j}_{i}\delta^{k}_{b}-\delta^{i}_{b}\delta^{j}_{i}\delta^{k}_{c}-\delta^{i}_{i}\delta^{j}_{c}\delta^{k}_{b}-\delta^{i}_{c}\delta^{j}_{b}\delta^{k}_{i}\\
            =& \\
        \end{align*}
        as \(\delta^i_i=1\) and anything contains \(i\) and is not \(\delta^i_i\) euqlas zero and takes the term with it.
        \item [d)]
        \begin{align*}
            &\nabla\times\left(\nabla\times F\right)_i\\
            =&\epsilon^{ijk}\partial_j(\epsilon_{abc} \partial^b A_c)^k\\
            =&\epsilon^{ijk}\partial_j(\epsilon_{kbc} \partial^b A_c)^k\\
            =&\epsilon^{kij}\partial_j(\epsilon_{kbc} \partial^b A_c)^k\\
            =&(\delta^{i}_{b}\delta^{j}_{c}- \delta^{i}_{c}\delta^{j}_{b})(\partial_j\partial^bA_c)\\
        \end{align*}
        Thus, after considering all the transpositions, there will be:
        \begin{align*}
            \nabla\times\left(\nabla\times F\right)_i &= \nabla(\nabla\cdot F)-\nabla^2 F
        \end{align*}
    \end{itemize}
    \item [2.]
    \begin{itemize}
        \item [a)]
        \begin{align*}
            &\int_{1}^{3}\langle t,3t+1,2t\rangle\cdot\langle1,3,2\rangle dt\\
            =&\int_{1}^{3}14t+3 dt\\
            =&62
        \end{align*}
        \item [b)] Parametrize \(x,y,z\), we get \(\left\{\begin{matrix}
            x=t\\
            y=2t\\
            z=5t^2
        \end{matrix}\right.\)
        \begin{align*}
            &\int_{0}^{2} \langle 4t^2,25t^4,3\rangle\cdot\langle1,2,10t\rangle dt\\
            =&\int_{0}^{2} 50t^4+4t^2+30tdt\\
            =&16010.7
        \end{align*}
        \item [c)]Parametrize \(x,y\), we get \(\left\{\begin{matrix}
            x=t^2\\
            y=\sqrt{4-t}
        \end{matrix}\right.\)
        \begin{align*}
            &\int_{\sqrt{2}}^{-\sqrt{2}}\langle-\sqrt{4-t},t^2,0\rangle\cdot\langle2t,-(4-t)^{-\frac{1}{2}},0 \rangle dt\\
            =&\int_{\sqrt{2}}^{-\sqrt{2}}-2t\sqrt{4-t}-\frac{t^2}{\sqrt{4-t}} dt\\
            &=-1.92322
        \end{align*}
    \end{itemize}
\end{itemize}
\end{document}