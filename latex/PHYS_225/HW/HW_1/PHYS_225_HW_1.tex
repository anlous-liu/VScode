\documentclass{article}
\usepackage{graphicx}
\usepackage{amsmath}
\usepackage{array}
\usepackage[font=small, labelfont={sf,bf}, margin=1cm]{caption}
\usepackage{tabularx}
\usepackage{amssymb}



\date{Due:Sep 5 Edit: \today}
\title{PHYS 225 HW 1}
\author{James Liu}

\begin{document}
\maketitle
\begin{itemize}
    \item [\textbf{1.}] Measure and find  a mid point between the 2 lights, \
    put a machine that will turn two clocks on when it detected that the light emitted from the bulb arrives. Turn on the bulb.
    \item [\textbf{2.}] A travels at speed 4c/5 toward B, who is at rest. C is between A and B. How fast should
    C travel so that she sees both A and B approaching her at the same speed? 
    \\ \\ 
    Set the speed of C as \(v_c\), the direction to B is positive and B's position is origin. Therefore, we have:
    and the velocity of B and A approaching C in C's reference frame is \(u\).
    Therefore, in C's reference frame, there is B moving toward C at \(u\) and in B's reference frame, A is moving toward B at \(0.8c\), thus in C's reference frame, the speed of A to C need to apply relativistic addition.
    \begin{align*}
        u = \frac{-u+\frac{4}{5}c}{1+\frac{-\frac{5}{4}uc}{c^2}}\\
        u = \frac{1}{2}c
    \end{align*}
    \(u=v_c\) Therefore, \(\fbox{\(v_c =\frac{1}{2}c\)}\)
    \newpage
    \item [\textbf{3.}] \textbf{Relative speeds in different frames.}
    \begin{itemize}
        \item [a)]\(v_t\) as the torpedo's velocity in my reference frame and \(v'_t\) as the velocity of torpedo in star destroyer's reference frame.
        \begin{align*}
            v_t &= \frac{v_d+v_t'}{1+\frac{v_dv_t'}{c^2}}\\
                &= \frac{\frac{5}{6}c}{1+\frac{c^2}{6c^2}}\\
                &= \frac{\frac{5}{6}c}{1+\frac{1}{6}}\\
                &= \frac{5}{7}c
        \end{align*}
        \[v_c-v_t = \frac{3}{4}c-\frac{5}{7}c=\fbox{\(\frac{1}{28}c\)}>0\]
        Thus the Corvette does escape.
        \item [b)] See the reference frame moving at \(-v_d\) toward the Star destroyer. Thus:
        \begin{align*}
            v_c' &= \frac{-v_d+v_c}{1+\frac{-v_dv_c}{c^2}}\\
                &= \frac{\frac{1}{4}c}{1-\frac{3}{8}}\\
                &= \frac{2}{5}c
        \end{align*}
        \[v_c'-v_t'=\frac{2}{5}c-\frac{1}{3}c=\fbox{\(\frac{1}{15}c\)}>0\]
        Therefore, the Corvette does escape.
    \end{itemize}
\end{itemize}


\end{document}