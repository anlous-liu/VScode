\documentclass{article}
\usepackage{graphicx}
\usepackage{amsmath}
\usepackage{array}
\usepackage[font=small, labelfont={sf,bf}, margin=1cm]{caption}
\usepackage{tabularx}
\usepackage{amssymb}



\date{Due:Sep 19 Edit: \today}
\title{PHYS 225 HW 3}
\author{James Liu}

\begin{document}
\maketitle
\begin{itemize}
    \item [1.]
    \begin{itemize}
        \item [a)]
        \begin{itemize}
            \item [i:]
            \begin{align*}
                \begin{bmatrix}
                    ct\\
                    x
                \end{bmatrix}
                =\gamma\cdot\begin{bmatrix}
                     1&\beta\\\beta & 1
                \end{bmatrix}\begin{bmatrix}
                    ct'\\ x'
                \end{bmatrix}
            \end{align*}
            putting the spacetime cordinate of the west clapping people and the car sycronized at origin.
            Then the space time event of the east people clapping in ground frame would be \(\displaystyle \begin{pmatrix}
               0\\L 
            \end{pmatrix}\). According to the car's frame, the east people is moving at speed \(-0.8 c\) on x diection.
            Thus, \(\gamma = \frac{1}{\sqrt{1-\frac{v^2}{c^2}}} = \frac{5}{3}\), \(\beta = \frac{v}{c}=-0.8\).
            Then the event in the car's reference frame is given by:
            \begin{align*}
                \begin{bmatrix}
                    \frac{5}{3}&\frac{4}{3}\\
                    \frac{4}{3}&\frac{5}{3}
                \end{bmatrix}
                \begin{bmatrix}
                    0\\L
                \end{bmatrix}=
                \begin{bmatrix}
                    \frac{4}{3}L \\ \frac{5}{3}L
                \end{bmatrix}
            \end{align*}
            As \(ct = \frac{4}{3}L, t = \frac{4L}{3c}\).
            In the car's reference frame, after \(\frac{4L}{3c}\) time, its position will be at \(\frac{4L}{3c}\times \frac{4}{5}c = \frac{16}{15}L\).
            Thus, the space time event of car passing the tree is: \(\begin{pmatrix}
                \frac{4L}{3c}\\ 0
            \end{pmatrix}\)
            Then in the ground reference frame the event happens at:
            \begin{align*}
                \begin{bmatrix}
                    \frac{5}{3}&\frac{4}{3}\\\frac{4}{3}&\frac{5}{3}
                \end{bmatrix}
                \begin{bmatrix}
                    \frac{4L}{3}\\ 0
                \end{bmatrix}=
                \begin{pmatrix}
                    \frac{20}{9}L\\ \frac{16}{9}L
                \end{pmatrix}
            \end{align*}
            Thus, the tree should be \(\frac{16}{9}L\) east from the west people.
            \item [ii:] Setting the speed of car as \(kc\), then \(\gamma = \frac{1}{\sqrt{1-k^2}}\) and \(\beta = k\).
                        Thus, the space time event that the east people claps is still \(\begin{bmatrix}
                        0\\L
                        \end{bmatrix}\). Then the space time even in car's frame is:
                        \begin{align*}
                            \begin{bmatrix}
                                \frac{1}{\sqrt{1-k^2}}&\frac{k}{\sqrt{1-k^2}}\\
                                \frac{k}{\sqrt{1-k^2}}&\frac{1}{\sqrt{1-k^2}}
                            \end{bmatrix}
                            \begin{bmatrix}
                                0\\L
                            \end{bmatrix}=
                            \begin{pmatrix}
                                \frac{k}{\sqrt{1-k^2}}L\\ 
                                \frac{1}{\sqrt{1-k^2}}L
                            \end{pmatrix}
                        \end{align*}
                        The tree event will be at:
                        \begin{align*}
                            \begin{bmatrix}
                                \frac{1}{\sqrt{1-k^2}}&\frac{k}{\sqrt{1-k^2}}\\
                                \frac{k}{\sqrt{1-k^2}}&\frac{1}{\sqrt{1-k^2}}
                            \end{bmatrix}
                            \begin{bmatrix}
                                \frac{k}{\sqrt{1-k^2}}L\\ 0
                            \end{bmatrix}=
                            \begin{bmatrix}
                                \frac{k}{1-k^2}L \\ \frac{k^2}{1-k^2}L
                            \end{bmatrix}
                        \end{align*}
                        set \(\frac{k^2}{1-k^2}L = L\), \(k = \frac{1}{\sqrt 2}\approx 70.7\%\).\\ Thus the velocity should be \(\frac{c}{\sqrt2}\)
        \end{itemize}
        \item [b)] 
        \end{itemize}
    \item [2.]\(\gamma_1 = \frac{1}{\sqrt{1-v_1^2/c^2}},\ \gamma_2 = \frac{1}{\sqrt{1-v_2^2/c^2}}\), \(\beta_1  = \frac{v_1}{c}, \ \beta_1  = \frac{v_1}{c}\).

                \[\begin{bmatrix} 
                        \frac{1}{\sqrt{1-v_1^2/c^2}}& \frac{v_1}{\sqrt{c^2-v_1^2}}\\
                        \frac{v_1}{\sqrt{c^2-v_1^2}}& \frac{1}{\sqrt{1-v_1^2/c^2}}
                \end{bmatrix}
                \begin{bmatrix}
                    \frac{1}{\sqrt{1-v_2^2/c^2}}& \frac{v_2}{\sqrt{c^2-v_2^2}}\\
                    \frac{v_2}{\sqrt{c^2-v_2^2}}& \frac{1}{\sqrt{1-v_2^2/c^2}}
                \end{bmatrix}\]\[=
                \begin{bmatrix}
                    \frac{1}{\sqrt{(1-v_1^2/c^2)(1-v_2^2/c^2)}}+\frac{v_1v_2}{\sqrt{(c^2-v_1^2)(c^2-v_2^2)}}&
                    \frac{v_2}{\sqrt{(1-v_1^2/c^2)(c^2-v_2^2)}}+\frac{v_1}{\sqrt{(1-v_2^2/c^2)(c^2-v_1^2)}}
                    \\ \cdots&\cdots
                \end{bmatrix}\]
                Thus, \[\beta = \gamma \beta / \gamma\]
                \[=\frac{v_2\sqrt{(1-v_2^2/c^2)(c^2-v_1^2)}+v_1\sqrt{(1-v_1^2/c^2)(c^2-v_2^2)}}{\sqrt{(c^2-v_1^2)(c^2-v_2^2)}+v_1v_2\sqrt{(1-v_1^2/c^2)(1-v_2^2/c^2)}}\]
                Thus, \(v = \beta \times c\)
                \begin{align*}
                    &=\frac{v_2\sqrt{(1-v_2^2/c^2)(c^2-v_1^2)}+v_1\sqrt{(1-v_1^2/c^2)(c^2-v_2^2)}}{\sqrt{(c^2-v_1^2)(c^2-v_2^2)}+v_1v_2\sqrt{(1-v_1^2/c^2)(1-v_2^2/c^2)}}
                    \times c\\&=\frac{v_2\sqrt{(c^2-v_2^2)(c^2-v_1^2)}+v_1\sqrt{(c^2-v_1^2)(c^2-v_2^2)}}{\sqrt{(c^2-v_1^2)(c^2-v_2^2)}+v_1v_2\sqrt{(1-v_1^2/c^2)(1-v_2^2/c^2)}}
                    \\&=\frac{\sqrt{(c^2-v_2^2)(c^2-v_1^2)}(v_1+v_2)}{\sqrt{(c^2-v_1^2)(c^2-v_2^2)}+v_1v_2\sqrt{(1-v_1^2/c^2)(1-v_2^2/c^2)}}
                    \\&=\frac{v_1+v_2}{1+v_1v_2\sqrt{\frac{(1-v_1^2/c^2)(1-v_2^2/c^2)}{(c^2-v_2^2)(c^2-v_1^2)}}}
                    \\&=\frac{v_1+v_2}{1+v_1v_2\sqrt{\frac{(1-v_1^2/c^2)(1-v_2^2/c^2)}{(1-v_2^2/c^2)c^2(1-v_1^2/c^2)c^2}}}
                    \\&=\frac{v_1+v_2}{1+v_1v_2/c^2}
                \end{align*}
\end{itemize}


\end{document}