\documentclass{article}
\usepackage{graphicx}
\usepackage{amsmath}
\usepackage{array}
\usepackage[font=small, labelfont={sf,bf}, margin=1cm]{caption}
\usepackage{tabularx}
\usepackage{amssymb}



\date{Due:Sep 19 Edit: \today}
\title{PHYS 225 HW 3}
\author{James Liu}

\begin{document}
\maketitle
\begin{itemize}
    \item [1.]
    \begin{itemize}
        \item [a)] 
        \begin{itemize}
            \item [i:]
                \(f(x) = \sum f^{n}(0)\times \frac{x^n}{n!}\). \(f(0)' = \ln(x)\sin(x)+\frac{\cos(x)}{x}\) does not exist and limit does not exist.
                as \(\ln(-0)\) also does not exist.

            \item [ii:] \(\sin(x) = 0+1x+0x^2-\frac{1}{3!}x^3+\cdots\)\\
                        \(x = 0+1x+0+\cdots\)\\
                        \(\frac{\sin(x)}{x} = 0+1+0x-\frac{1}{3!}x^2+\cdots\)
        \end{itemize}
        \item [b)]
        \(\)
        \begin{align*}
            f(x)&=\frac{1}{1+x^2}\\
            f(0)&=1 \\
            f'(0)&=-2x(1+x)^{-2}=0\\
            f''(0)&=-2(1+x^2)^{-2}+4x^2(1+x^2)^{-3}=-2\\
            f'''(0)&=4x(1-x^2)^{-3}+8x(1+x^2)^{-3}+8x^3(1+x^2)^{-4}=0
        \end{align*}
        Thus, \(\frac{1}{1+x^2} = 1-\frac{2}{2!}x^2=1-x^2+\cdots\)\\
        \(\arctan(x) = \int1-x^2+\cdots = x-\frac{x^3}{3}+\cdots\)
        \item [c)]
        \begin{align*}
            e^x &= 1+t+\frac{t^2}{2!}+\frac{t^3}{3!}+\frac{t^4}{4!}+\frac{t^5}{5!}+\cdots\\
            \sin(x) &= t-\frac{t^3}{3!}\\
            e^{\sin(x)} &= 1 + \left(x - \frac{x^3}{6}\right) + \frac{\left(x - \frac{x^3}{6}\right)^2}{2} + \frac{x^3}{6} + \cdots\\
            &=1 + x + \frac{x^2}{2} - \frac{x^4}{8} +\cdots
        \end{align*}
    
    
    \end{itemize}
    \item [2.] expand \(\frac{1}{\sqrt{1-k^2\sin^2(\theta)}}\) around \(k=0\)
    \begin{align*}
        f(0) &= 1\\
        f'(0)&= -\frac{1}{2}(1-k^2\sin^2(\theta))^{-\frac{3}{2}}\times 2k\sin^2(\theta) = 0\\
        f''(0)&=-2\sin^2(\theta)\frac{1}{2}(1-k^2\sin^2(\theta))^{-\frac{3}{2}}+\cdots = -\sin^2(\theta)\\
        f(k) &= 1-\frac{1}{2}k^2\sin^2(\theta)+\cdots\\
        \int_{0}^{\pi/2}f(k)\text{d}{\theta} &= \frac{\pi}{2}-k^2\frac{\pi}{8}
    \end{align*}
    Thus, it is \(-\frac{\pi}{8}\)
    \item [3.]
    \begin{itemize}
        \item [a)]
        \begin{align*}
            \overrightarrow{w}&=R(\theta)\overrightarrow{v}+\overrightarrow{a}\\
            \overrightarrow{u}&=R(\phi)\overrightarrow{w}+\overrightarrow{b}\\
            &=R(\phi)(R(\theta)\overrightarrow{v}+\overrightarrow{a})+\overrightarrow{b}\\
            &=R(\phi)R(\theta)\overrightarrow{v}+R(\phi)\overrightarrow{a}+\overrightarrow{b}
        \end{align*}
        \item [b)]
        \begin{align*}
            \overrightarrow{w}&=R(\phi)\overrightarrow{v}+\overrightarrow{b}\\
            \overrightarrow{u}&=R(\theta)\overrightarrow{w}+\overrightarrow{a}\\
            &=R(\theta)(R(\phi)\overrightarrow{v}+\overrightarrow{b})+\overrightarrow{a}\\
            &=R(\theta)R(\phi)\overrightarrow{v}+R(\theta)\overrightarrow{b}+\overrightarrow{a}
        \end{align*}
        A different vector is rotated, it was \(\overrightarrow{a}\) rotating the second time while it is now \(\overrightarrow{b}\), the \(\overrightarrow{v}\) is same when rotating 2 times.
        \item [c)]
        \begin{align*}
            M_B &= \begin{bmatrix}
                \cos(\phi)&-\sin(\phi)&x_1\\
                \sin(\phi)&\cos(\phi)&y_1\\
                0&0&1
            \end{bmatrix}\\
            M_BM_A &=\begin{bmatrix}
                \cos(\phi)&-\sin(\phi)&x_1\\
                \sin(\phi)&\cos(\phi)&y_1\\
                0&0&1
            \end{bmatrix}\begin{bmatrix}
                \cos(\theta)&-\sin(\theta)&x_0\\
                \sin(\theta)&\cos(\theta)&y_0\\
                0&0&1
            \end{bmatrix}\\
            &=\begin{bmatrix}
                \cos\theta \cos\phi - \sin\theta \sin\phi & -\cos\theta \sin\phi - \sin\theta \cos\phi & \cos\theta x_0 - \sin\theta y_0 + x_1 \\
                \sin\theta \cos\phi + \cos\theta \sin\phi & -\sin\theta \sin\phi + \cos\theta \cos\phi & \sin\theta x_0 + \cos\theta y_0 + y_1 \\
                0 & 0 & 1
                \end{bmatrix}\\
            M_AM_B &=\begin{bmatrix}
                \cos(\theta)&-\sin(\theta)&x_0\\
                \sin(\theta)&\cos(\theta)&y_0\\
                0&0&1
            \end{bmatrix}\begin{bmatrix}
                \cos(\phi)&-\sin(\phi)&x_1\\
                \sin(\phi)&\cos(\phi)&y_1\\
                0&0&1
            \end{bmatrix}\\
            &=\begin{bmatrix}
                \cos\theta \cos\phi - \sin\theta \sin\phi & -\cos\theta \sin\phi - \sin\theta \cos\phi & \cos\theta x_1 - \sin\theta y_1 + x_0 \\
                \sin\theta \cos\phi + \cos\theta \sin\phi & -\sin\theta \sin\phi + \cos\theta \cos\phi & \sin\theta x_1 + \cos\theta y_1 + y_0 \\
                0 & 0 & 1
                \end{bmatrix}
        \end{align*}
        As \(M_AM_B \neq M_BM_A\) Thus the result is similar with the results above. also,it is clear that there is a different vector that is rotated.
    \end{itemize}
\end{itemize}
\end{document}