\documentclass{article}
\usepackage{graphicx}
\usepackage{amsmath}
\usepackage{array}
\usepackage[font=small, labelfont={sf,bf}, margin=1cm]{caption}
\usepackage{tabularx}
\usepackage{amssymb}
\usepackage{multirow}



\date{Due:Oct 14 Edit: \today}
\title{PHYS 225 HW 7}
\author{James Liu}

\begin{document}
\maketitle
\begin{itemize}
    \item [1.] First get the space time cordinates in ground frame and then transform them into different frames. Frist, the system is synchronized at \(E_1\). Meaning that \(E_1\) happens at \((0,0)\) in all 3 frames. Meaning that we can have build a table like this: 
    \\
    \begin{tabular}{c | c c | c c | c c|}
        &\multicolumn{2}{c|}{Ground Frame}&\multicolumn{2}{c|}{Train's back Frame}&\multicolumn{2}{c|}{C's Frame}\\
        \hline Event& $t$&$x$&$t'$&$x'$&$t_c$&$x_c$\\
        $E_1$& 0&0&0&0&0&0\\ 
        $E_2$& ?&?&?&?&?&0
    \end{tabular}\\
    \begin{align*}
        \Delta v &= \frac{4}{5}c-\frac{3}{5}c=\frac{1}{5}c&
        t &= \frac{L}{v} = \frac{L}{\frac{1}{5}c} = \frac{5L}{c}\\
    \end{align*}
    \[x = v\times t  = \frac{4}{5}c\times \frac{5L}{c} = 4L = L+\frac{3}{5}c\times\frac{5L}{c}\]
    Thus, the space time coordinate of \(E_1\) is \( \left(\frac{5L}{c},4L\right)\). Apply lorentz transformation on this, first to solve for the event in train's frame:
    \begin{align*}
        \begin{bmatrix}
            \gamma&\gamma\beta\\
            \gamma\beta&\gamma
        \end{bmatrix}&=
        \begin{bmatrix}
            \frac{1}{\sqrt{1-\frac{v^2}{c^2}}}&\frac{1}{\sqrt{1-\frac{v^2}{c^2}}}\frac{v}{c}\\
            \frac{1}{\sqrt{1-\frac{v^2}{c^2}}}\frac{v}{c}&\frac{1}{\sqrt{1-\frac{v^2}{c^2}}}
        \end{bmatrix}\\
        &=\begin{bmatrix}
            \frac{5}{4}&-\frac{3}{4}\\
            -\frac{3}{4}&\frac{5}{4}
        \end{bmatrix}\\
        \begin{bmatrix}
            \gamma&\gamma\beta\\
            \gamma\beta&\gamma
        \end{bmatrix}\begin{bmatrix}
            ct\\x
        \end{bmatrix}&=\begin{bmatrix}
            \frac{5}{4}&-\frac{3}{4}\\
            -\frac{3}{4}&\frac{5}{4}
        \end{bmatrix}\begin{bmatrix}
            5L\\4L
        \end{bmatrix}\\
        &=\begin{bmatrix}
            \frac{13}{4}L\\\frac{5}{4}L
        \end{bmatrix}
    \end{align*}
    Thus, the space time coordinate of \(E_2\) at train's back frame is \(\left(\frac{13L}{4c},\frac{5}{4}L\right)\) 
    Now compute the coordinate in \(C\)'s frame:
    \begin{align*}
        \begin{bmatrix}
            \gamma&\gamma\beta\\
            \gamma\beta&\gamma
        \end{bmatrix}&=
        \begin{bmatrix}
            \frac{1}{\sqrt{1-\frac{v^2}{c^2}}}&\frac{1}{\sqrt{1-\frac{v^2}{c^2}}}\frac{v}{c}\\
            \frac{1}{\sqrt{1-\frac{v^2}{c^2}}}\frac{v}{c}&\frac{1}{\sqrt{1-\frac{v^2}{c^2}}}
        \end{bmatrix}\\
        &=\begin{bmatrix}
            \frac{5}{3}&-\frac{4}{3}\\
            -\frac{4}{3}&\frac{5}{3}
        \end{bmatrix}\\
        \begin{bmatrix}
            \gamma&\gamma\beta\\
            \gamma\beta&\gamma
        \end{bmatrix}\begin{bmatrix}
            ct\\x
        \end{bmatrix}&=\begin{bmatrix}
            \frac{5}{3}&-\frac{4}{3}\\
            -\frac{4}{3}&\frac{5}{3}
        \end{bmatrix}\begin{bmatrix}
            5L\\4L
        \end{bmatrix}\\
        &=\begin{bmatrix}
            3L\\0
        \end{bmatrix}
    \end{align*}
    Thus, the table becomes:\\
    \begin{tabular}{c | c c | c c | c c|}
        &\multicolumn{2}{c|}{Ground Frame}&\multicolumn{2}{c|}{Train's back Frame}&\multicolumn{2}{c|}{C's Frame}\\
        \hline Event& $t$&$x$&$t'$&$x'$&$t_c$&$x_c$\\
        $E_1$& 0&0&0&0&0&0\\ 
        $E_2$& $\frac{5L}{c}$&$4L$&$\frac{13L}{4c}$&$\frac{5}{4}L$&$\frac{3L}{c}$&0
    \end{tabular}\\
    Now, check for values of \(c^2(\Delta t)^2-(\Delta x)^2\)\\
    \begin{align*}
        c^2(\Delta t)^2-(\Delta x)^2&=c^2\left(\frac{5L}{c}\right)-\left(4L\right)^2= 9L^2\\
        &=c^2\left(\frac{13L}{4c}\right)-\left(\frac{5}{4}L\right)^2 = \frac{13^2-5^2}{16}L^2 = 9L^2\\
        &=c^2\left(\frac{3L}{c}\right)-0 = 9L^2
    \end{align*}
    \newpage
    \item [2.]
    \begin{itemize}
        \item [a)]
        Synchronize two system at \(E_1\) when the left side of pole passing left side of barn.
        Label the space time event in barn's frame as \(S\), poles as \(S'\), coordinate of left of barn as \(B_l\), and right side as \(B_r\), for pole, it is \(P_l \) and \(P_r\).
        \begin{table}[h]
            \centering
            \begin{tabular}{c c | c c | c c }
                \multicolumn{2}{c|}{Label }&\multicolumn{2}{c|}{S}&\multicolumn{2}{c}{S'}\\ \hline
                &&                                     t&x&t'&x' \\ \hline
                \multirow{2}{*}{\(E_1\)}&\(B_l\)&0&0&0&0         \\
                                        &\(P_l\)&0&0&0&0             \\\hline
                \multirow{2}{*}{\(E_2\)}&\(B_r\)&\(t_1\)&L&\(t_2\)&L                   \\
                                        &\(P_r\)&\(t_1\)&L&\(t_2\)&L \\
            \end{tabular}
        \end{table}\\
        Now solve for \(t_1,t_2\):
        \begin{align*}
            \begin{bmatrix}
                \frac{5}{4}&-\frac{3}{4}\\
                -\frac{3}{4}&\frac{5}{4}
            \end{bmatrix}\begin{bmatrix}
                ct_1\\L
            \end{bmatrix}&=\begin{bmatrix}
                ct_2\\L
            \end{bmatrix}\\ \left\{
            \begin{matrix}
                \frac{5}{4}ct_1-\frac{3}{4}L = ct_2\\
                -\frac{3}{4}ct_1+\frac{5}{4}L = L
            \end{matrix}\right.\\\left\{
                \begin{matrix}
                    t_1 = \frac{L}{3c}\\
                    t_2 =-\frac{L}{3c}
                \end{matrix}\right.
        \end{align*}
        Thus, \(\Delta t =t_1-0=\frac{L}{3c},\ \Delta x = L-0=L\)
        \item [b)] In the red frame or double prime frame, the 2 events are all on x-axis, meaning they are happening at the same time of \(t=0\)
        \begin{figure}[h]
            \centering
            \includegraphics*[scale = 0.05]{figure/phys225_hw7_fig1.jpeg}
        \end{figure}
    \end{itemize}
\end{itemize}
\end{document}