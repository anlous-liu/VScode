\documentclass{article}
\usepackage{graphicx}
\usepackage{amsmath}
\usepackage{array}
\usepackage[font=small, labelfont={sf,bf}, margin=1cm]{caption}
\usepackage{tabularx}
\usepackage{amssymb}
\usepackage{multirow}



\date{Due:Oct 31 Edit: \today}
\title{PHYS 225 HW 8}
\author{James Liu}

\begin{document}
\maketitle
\begin{itemize}
    \item [1.]
    \begin{itemize}
        \item [a)]
        \begin{align*}
            \Delta E &= F\Delta x\\
            (\Delta E)^2 &= F^2(\Delta x)^2\\
            E^2 &= m^2c^4+p^2c^2\\
            F^2(\Delta x)^2 &=m^2c^4+p^2c^2\\
            p^2c^2 &=F^2(\Delta x)^2-m^2c^4\\
            p^2 &=\frac{F^2(\Delta x)^2-m^2c^4}{c^2}\\
            t&= p/\frac{\text{d}p}{\text{d}t}\\
            t &= p/F\\
            t &=\sqrt{\frac{F^2(\Delta x)^2-m^2c^4}{F^2c^2}}\\
        \end{align*}
        \begin{itemize}
            \item [relativistic:] when \(x\rightarrow \infty\), \(t\rightarrow\infty\), meaning that it would take infinity amount of time to go to infinitly away, meaning that 
            \item [non-relativistic:]when \(x\ll mc^2/F\), \(t = \sqrt{\frac{m^2c^4-F^2(\Delta x)^2}{F^2c^2}}\) which is finite and reasonable as er cahn push something to a close distance in finite amount of time.
        \end{itemize}
        \item [b)]
        \begin{align*}
            F^2x_m^2 &=m^2c^4+p^2c^2\\
            x_m^2&=\frac{m^2c^4}{F^2}+\frac{p^2c^2}{F^2}\\
            x_m^2&=\frac{m^2c^4}{F^2}+t^2c^2\\
            x_m &= \sqrt{\frac{m^2c^4}{F^2}+t^2c^2}\\
            x_p&=tc\\
            x_m &= tc\times \frac{1}{tc}\sqrt{\frac{m^2c^4}{F^2}+t^2c^2} \\
            &=tc\sqrt{\frac{m^2c^4}{t^2c^2F^2}+1}\\
            &\approx tc(1+\frac{m^2c^4}{2t^2c^2F^2})\\ 
            &=tc+\frac{m^2c^4}{2tcF^2}\\
            \Delta x &=\frac{m^2c^4}{2tcF^2}
        \end{align*}
    \end{itemize}
    \item [2.]
    \begin{itemize}
        \item [a)]
        \begin{align*}
            E_{ini}&=E_{final}\\
            M^2c^4 &=M^2c^4+p^2c^2+dm^2c^4+p'^2c^2\\
            p &=\gamma_u MU\\
            p' &= \gamma_w dm w\\
            p+p'&=0\\
            \frac{MdU}{\sqrt{1-U^2/c^2}}+\frac{dm w}{\sqrt{1-w^2/c^2}}&=0\\
            dU &=\frac{dm}{M} \frac{\sqrt{1-U^2/c^2}}{\sqrt{1-w^2/c^2}}\\
            \frac{M}{dm}&=\frac{\sqrt{1-U^2/c^2}}{dU \sqrt{1-w^2/c^2}}\\
            \frac{dm}{M}&=\frac{dU \sqrt{1-w^2/c^2}}{\sqrt{1-U^2/c^2}}\\
            \frac{dm}{M}&=-\frac{dU }{w\sqrt{1-U^2/c^2}}
        \end{align*}
        \item [b)]
        \begin{align*}
            \frac{dM}{M} &= -\frac{dU}{w\left(1 - \frac{U^2}{c^2}\right)}\\
            \frac{dM}{M} &= -\frac{dU}{w} \cdot \frac{1}{1 - \frac{U^2}{c^2}}\\
            \int_{M_0}^{M(U)} \frac{dM}{M} &= -\int_0^U \frac{dU}{w} \cdot \frac{1}{1 - \frac{U^2}{c^2}}\\
            \ln\left(\frac{M(U)}{M_0}\right) &= -\frac{1}{w} \int_0^U \frac{dU}{1 - \frac{U^2}{c^2}}\\
            z &= \frac{U}{c}\\
            -\frac{1}{w} \int_0^U \frac{dU}{1 - \frac{U^2}{c^2}} &= -\frac{1}{w} \int_0^z \frac{c \cdot dz}{1 - z^2} \\ &= -\frac{c}{w} \int_0^z \frac{dz}{1 - z^2}\\
            -\frac{c}{w} \left[\frac{1}{2} \ln\left|\frac{1 + z}{1 - z}\right|\right]_0^z &= -\frac{c}{w} \left[\frac{1}{2} \ln\left|\frac{1 + \frac{U}{c}}{1 - \frac{U}{c}}\right|\right]_0^U\\
            \ln\left(\frac{M(U)}{M_0}\right) &= -\frac{c}{2w} \ln\left|\frac{1 + \frac{U}{c}}{1 - \frac{U}{c}}\right|\\
            &=\ln\left[\left(\frac{1 + \frac{U}{c}}{1 - \frac{U}{c}}\right)^{-\frac{c}{2w}}\right]
            M(U) &= M_0 \left(\frac{1 + \frac{U}{c}}{1 - \frac{U}{c}}\right)^{-\frac{c}{2w}}
        \end{align*}
    \end{itemize}
\end{itemize}
\end{document}