\documentclass{article}
\usepackage{graphicx}
\usepackage{amsmath}
\usepackage{array}
\usepackage[font=small, labelfont={sf,bf}, margin=1cm]{caption}
\usepackage{tabularx}
\usepackage{amssymb}

\begin{document}
\title{\textbf{Homework \#1 }}
\author{James Liu}
\date{\ }
\maketitle
%test
\section*{Problem 1}
    
    a) \( \exists \ x\in A\) such that \(\forall \ b \in B\), satisfies \(b \le x\)\\
    b) \( \forall \ x \in A\) there \( \exists \ b\in B\) satisfies \(b \le x\)\\
    c) \( \exists x,y \in \mathbb{R}\) such that \(x\neq y\), satisfies \(f(x)=f(y)\)\\
    d) \( \exists \ \varepsilon \in P\), that \( \forall \ \delta \in P\) satisfies the
    statement that \(\exists \ x,y \in \mathbb{R}\) satisfies \(|x-y| < \delta\) and \(|f(x)-f(y)|\geq \varepsilon\)
\section*{problem 2}
    The negation of the statement is: \\
    There is a group of sets \(A\), \(B\) and \(C\) that satisfies \(A\nsubseteq B\), \(B \subseteq C\) and \(A \subseteq C\).
    Suppose there are three sets \(A=\{1,2\}, B=\{2,3\},C=\{1,2,3\}\). \\ 
    There is \(1 \in A, 1\notin B\). Therefore, \(A \nsubseteq B\).\\
    There is \(2,3 \in C\). Therefore, \(B \subseteq C\).\\
    There is \(1,2 \in C\). Therefore, \(A \subseteq C\).\\
    Thus, this group of sets satisfies \(A\nsubseteq B\), \(B \subseteq C\) and \(A \subseteq C\).\\
    The negation of the statement is true, Therefore the original statement is false.
\section*{Problem 3}
    Suppose there is a new set \(C\), that \(A\cap B = A\cup B = C\).\\
    There is\(A\cup B =C\), so every element in B and A shall be in C. \\
    Therefore\(A\subseteq C\), and \(B \subseteq C\).\\
    There is\(A\cap B =C\), so every element in \(C\) shall be in both \(A\) and \(B\).\\
    Therefore\(C\subseteq A\), and \(C \subseteq B\).\\
    There is \(C\subseteq A\), and\(A \subseteq C\), so every element in \(C\) is in \(A\) and every element in \(A\) in also in \(C\).\\
    Therefore, \(C = A\). Similarly, \(C = B\). Thus, \(A=B\).
\section*{Problem 4}
\subsection*{1. If \(A = B\), then \(A-B=B-A\)}
        There is\(A=B\), so every elements in set \(A\) is also in \(B\) and vice versa.\\
        Therefore, such element that is in set \(A\), but not in set \(B\) does not exist.\\
        Thus, \(A-B=\varnothing\), and, similarly, \(B-A = \varnothing\). 
        Therefore, \(A-B = B-A\).\\
\subsection*{2. If \(A-B=B-A\), then \(A = B\)}
        Suppose \(A\neq B\), so there shall be\(x \in A, x\notin B\). Therefore, \(x\in A-B\)\\
        Therefore, \(A-B \neq \varnothing\). Similarly, \(B-A \neq \varnothing\).\\
        Therefore, suppose \(a \in A-B, b \in B-A\), then \(a \in A, a\notin B, b\in B, b\notin A\).\\
        If the statement:" If \(A-B=B-A\), then \(A\neq B\)." is true, then there should be a pair of \(a=b\).\\
        There is \(a \in A, b \notin A\). Therefore \(a\neq b\). Thus, the statement is false.\\
        The statement is the negation of the statemen:"If \(A-B=B-A\), then \(A = B\)".\\
        Therefore, the statement is true.\\
        
        Thus, the original statement is true

\section*{Problem 5}
    There is, \(0\) times every other real number equals zero. \\
    Therefore, If \(x ={a_1}\), \((x-a_1)(x-a_2)\cdots(x-a_n)=0\). Therefore, \(x\neq a_1\).\\
    Similarly,\(x\neq a_1, a_2, a_3, \cdots, a_n\). There is that odd number of negative numbers times together results in a negative number and even numbers of them gives a positive number.
    Therefore, for the inequality to be negative, there shall be odd numbers of terms that are negative.
    When \(a_{n-1}<x<a_n\), there will be \(n-1\) terms being positive and 1 term being negative which is the last term. 
    Similarly, if \(a_{n-2}<x<a_{n-1}\), there will be two terms being negative resulting the inequality to be false.
    Following the same pattern, the answer will be different considering different property of \(n\).

    \subsection*{When n is an odd number}
        There is \(n\) is odd. Therefore, \(\forall \ i\in (\mathbb{N}\cup \{0\})
        ,\ n-2i \) is still odd, but \\ 
        \(n-1-2i\) is even.
        Therfore, when \(a_1<x<a_2\), there are \(n-1\) therms that are negative resulting in the expression to be positive.
        Similarly, intervals \((a_3,a_4),(a_5,a_6),\ \dots, \ (a_{n-2},a_{n-1})\) will result in similar effact.
        Therefore, the other interval will result in odd numbers of negative numbers timing each other, wich will satisfy the inequality.
        Therefore the answer shall be a set \(A = (-\infty,a_1)\cup (a_2,a_3)\cup (a_4,a_5)\cup \dots \cup (a_{n-1},a_n)\).

    \subsection*{When n is an even number}
        There is \(n\) is even. Therefore, \(\forall \ i\in (\mathbb{N}\cup \{0\})
        ,\ n-2i \) is still even, but \\
        \(n-1-2i\) is odd.
        Therfore, when \(-\infty <x<a_1\), there are \(n\) therms that are negative resulting in the expression to be positive.
        Similarly, intervals \((a_2,a_3),(a_4,a_5),\ \dots, \ (a_{n-1},a_{n})\) will result in similar effact.
        Therefore, the other interval will result in odd numbers of negative numbers timing each other, wich will satisfy the inequality.
        Therefore the answer shall be a set \(A = (a_1,a_2)\cup (a_3,a_4)\cup (a_5,a_6)\cup \dots \cup (a_{n-1},a_n)\).
    \\
    
    Therefore the answer shall be: 
    \begin{equation*}
        A = 
        \begin{cases}
            (-\infty,a_1)\cup (a_2,a_3)\cup (a_4,a_5)\cup \dots \cup (a_{n-1},a_n), \text{If}\  \frac{n}{2} \in \mathbb{Z}\\
            (a_1,a_2)\cup (a_3,a_4)\cup (a_5,a_6)\cup \dots \cup (a_{n-1},a_n), \text{If}\  \frac{n}{2} \notin \mathbb{Z}
        \end{cases}
    \end{equation*}

\end{document}
