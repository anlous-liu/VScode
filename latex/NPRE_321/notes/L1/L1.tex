\documentclass{article}
\usepackage{graphicx}
\usepackage{amsmath}
\usepackage{array}
\usepackage[font=small, labelfont={sf,bf}, margin=1cm]{caption}
\usepackage{tabularx}
\usepackage{amssymb}



\date{Edit: \today, Class: Aug 26}
\title{NPRE 321 Lecture 1 Note}
\author{James Liu}

\begin{document}
\maketitle
\subsection*{What is a plasma?}

\begin{itemize}
    \item \bf{ionized gas/matter}
    \begin{description}
        \item[] plasma are not nessesarily gasses in atmosphere of sun it could be as dense as 1/2 of water.
    \end{description}
    \item quasi neutral
    \begin{description}
        \item[] it is electricly neutral macroly and not nessesarily neutral\\ microly 
    \end{description}
    \item Hot
    
\end{itemize}

\subsection*{Fun Facts}
room temperature measured in \(eV\) is about \(\frac{1}{40} eV\). Pressure plasma is about 1-2 \(eV\).
Fusion plasma at \(10k eV\).\\
In cases of fusion, plasmas are in stats of:
\begin{itemize}
    \item low density (\(10^{-5}\) tor *as height of mercury)
    \item high temperature (\(10keV\))
    \item energy losses are majorly because of photons
\end{itemize}
\newpage
\subsection*{Gauess Law}
\[F=qE\],\[E=\frac{q}{4\pi\varepsilon_0r^2}\]
\[F=\frac{q_1q_2}{4\pi\varepsilon_0r^2}\]
\\ note that \[F=ma\]
Therefore \[a=F/m=\frac{q_1q_2}{4\pi\varepsilon_0r^2m}\]
Thus, for a point charge moving in a electric field \(E\), there is:
\[y = y_0 +v_{0y}\cdot t + \frac{1}{2}a t2\]
\end{document}